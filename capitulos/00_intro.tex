% ####################################################################
% #                     CAPA CUSTOMIZADA (Titlepage)                 #
% ####################################################################
%
% Esta é a estrutura da sua capa, usando as definições do preâmbulo.
% Você pode colocar isso no início do seu \begin{document}.

\begin{titlepage}
    \centering
    \makeatletter % Permite o uso de comandos com '@' (ex: \@title)

    % --- CABEÇALHO DA UNIVERSIDADE ---
    {\fontsize{16}{18}\selectfont\bfseries UNIVERSIDADE DE SÃO PAULO\par}
    {\fontsize{14}{16}\selectfont ESCOLA DE ENGENHARIA DE SÃO CARLOS\par}
    {\fontsize{12}{14}\selectfont Departamento de Engenharia Elétrica e de Computação\par}
    
    \vspace{2.5cm}
    
    % --- TÍTULO DO RELATÓRIO ---
    % Usa o \title definido no preâmbulo
    {\huge\bfseries \@title\par}
    
    \vfill % Adiciona espaço flexível
    

    % --- INFORMAÇÕES DO PROFESSOR ---
    \begin{minipage}{0.9\textwidth}
        \large
        \begin{flushleft}
            \textbf{Professor:} \\
            \professor
        \end{flushleft}
    \end{minipage}

    \vspace{1.5cm}

    % --- INFORMAÇÕES DOS ALUNOS ---
    \begin{minipage}{0.9\textwidth}
        \large
        \begin{flushleft}
            \textbf{Aluno:} \\
            \@author
        \end{flushleft}
    \end{minipage}
    
    \vfill % Adiciona mais espaço flexível
    
    % --- LOCAL E DATA ---
    % Usa o \dataEntrega definido no preâmbulo
    {\large São Carlos - SP\par}
    {\large \dataEntrega\par}
    
    \makeatother % Restaura o comportamento normal do '@'
\end{titlepage}


% ####################################################################
% #                          ÍNDICE (Sumário)                        #
% ####################################################################
\newpage
\tableofcontents
\thispagestyle{empty} % Remove cabeçalho e rodapé da pág. do índice

\newpage

% ####################################################################
% #                    INÍCIO DO CONTEÚDO DO DOCUMENTO               #
% ####################################################################


\newpage