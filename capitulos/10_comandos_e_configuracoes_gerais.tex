\section{Comandos e Configurações Gerais}


\subsection{Configuração do Ambiente}

\subsubsection{Inicialização dos Ambientes}
\begin{lstlisting}[language=bash, caption={Terminal 1 - Para ICStation (Mentor)}]
cd /local/users/cad/
source .cshrc-mentor
\end{lstlisting}

\begin{lstlisting}[language=bash, caption={Terminal 2 - Para ELDO (Anacad)}]
cd /local/users/cad/
source .cshrc-anacad
\end{lstlisting}

\textbf{Importante}: Executar em shells separados para evitar conflitos.

\subsubsection{Criação e Acesso a Projetos}
\begin{lstlisting}[language=bash]
# Novo projeto
ams_ics -project nome_projeto -t c35b4c3

# Projeto existente (ex: ams_ics -p nome)
ams_ics -p nome_projeto

# Exemplo de prova
ams_ics -p prova24brbtv -t -c35b4c3
\end{lstlisting}

\subsubsection{Diretórios de Trabalho}
\begin{lstlisting}[language=bash, frame=none]
# Diretorio padrao de trabalho
cd /local/users/cad/workavdl

# Estrutura de projetos
nome_projeto.proj/
|-- cell.lib/
|   `-- default.group/
|       |-- logic.views/ (Esquematicos, Simbolos)
|       `-- layout.views/ (Layouts, pasta .cal com PEX)
`-- ...
\end{lstlisting}

\subsection{Comandos de Terminal Essenciais}

\subsubsection{Gerenciamento de Arquivos}
\begin{lstlisting}[language=bash]
# Criar arquivo de simulacao
touch circuito.cir

# Ver conteudo
cat circuito.cir

# Editar
vim circuito.cir
# ou
nano circuito.cir

# Copiar seguranca
cp projeto.proj projeto_backup.proj
\end{lstlisting}

\subsubsection{Execução de Simulações}
\begin{lstlisting}[language=bash]
# Simulacao ELDO
eldo circuito.cir

# Abrir resultados
ezwave circuito.wdb

# Executar em background
eldo circuito.cir > log.txt 2>&1 &
\end{lstlisting}

\subsubsection{Processamento de Imagens (Screenshots)}
\begin{lstlisting}[language=bash]
# Capturar a janela (mouse vira uma mira)
xwd > layout.xwd

# Converter para formato util (inverte e limpa)
convert -white-threshold 1 -negate layout.xwd layout.tif
convert -white-threshold 1 -negate layout.xwd layout.png
\end{lstlisting}

\subsection{Paths e Diretórios Importantes}

\subsubsection{Modelos e Regras}
\begin{lstlisting}[language=bash, frame=none]
# Pasta de Modelos de transistores
/local/tools/dkit/ams_3.70_mgc/eldo/c35/

# Modelo Tipico (exemplo de arquivo)
/local/tools/dkit/ams_3.70_mgc/eldo/c35/cmos53tm.mod

#  Modelo de Resistor
/local/tools/dkit/ams_3.70_mgc/eldo/c35/restm.mod

# Regras Calibre
/local/users/cad/work/rules/cac35b4rules_all.run
/local/users/cad/Calibre_rules/cac35b4rules_all.run

# Manuais
/local/tools/dkit/ams_3.70_mgc/docs/ENG-182_rev2.pdf
/local/tools/dkit/ams_3.70_mgc/docs/ENG-183_rev2.pdf
/local/tools/mentor/shared/pdfdocs/eldo_ur.pdf
\end{lstlisting}

\subsubsection{Arquivos de Projeto}
\begin{lstlisting}[language=bash, frame=none]
# Netlists PEX (localizacao)
/local/users/cad/work/<projeto>.proj/cell.lib/default.group/
layout.views/<celula>/<celula>.cal/

# ViewPoints (localizacao)
/local/users/cad/work/<projeto>.proj/cell.lib/default.group/
logic.views/<celula>/vpt_c35b4_device
\end{lstlisting}

\subsection{Configurações do ICStation}

\subsubsection{Hotkeys e Softkeys}
\begin{itemize}
\item \textbf{Other $\rightarrow$ Hotkeys $\rightarrow$ Enable} 
\item \textbf{Other $\rightarrow$ Hotkeys $\rightarrow$ Load} 
\item \textbf{Other $\rightarrow$ Hotkeys $\rightarrow$ Report} (ver hotkeys ativas) 
\item \textbf{MGC $\rightarrow$ Setup $\rightarrow$ Show Softkeys} 
\end{itemize}

\subsubsection{Configurações de Grid}
\begin{itemize}
\item \textbf{Other $\rightarrow$ Window $\rightarrow$ Set Grid} 
\item \textbf{X = 0.025}, \textbf{Y = 0.025} 
\item \textbf{Minor = 0.1}, \textbf{Major = 1} 
\end{itemize}

\subsubsection{Camadas e Visualização}
\begin{itemize}
\item \textbf{Other $\rightarrow$ Layers $\rightarrow$ Show layer palette $\rightarrow$ Append $\rightarrow$ all} 
\item \textbf{View $\rightarrow$ Visible Layers} (configurar visibilidade) 
\item \textbf{Setup $\rightarrow$ Select Filter $\rightarrow$ Properties} (para selecionar textos) 
\end{itemize}

\subsubsection{Reserva de Células}
\begin{itemize}
\item \textbf{Verificar status}: "Context: nome\_celula(GE-R-0)" = Não reservado 
\item \textbf{Reservar}: \textbf{File $\rightarrow$ Cell $\rightarrow$ Reserve $\rightarrow$ Current Context} 
\item \textbf{Liberar}: \textbf{File $\rightarrow$ Cell $\rightarrow$ Unreserve}
\end{itemize}

\newpage

\subsection{Configurações ELDO/SPICE}

\subsubsection{ Sintaxe de Componentes (ELDO)}
\begin{itemize}
    \item \textbf{Transistor (MOSFET):} 
    \begin{lstlisting}[language=pspice, frame=none]
M<nome> <dreno> <gate> <source> <bulk> <modelo>
+ w=<width> l=<length>
+ Ad=<area_dreno> Pd=<perim_dreno>
+ As=<area_source> Ps=<perim_source>
    \end{lstlisting}
    \item \textbf{Resistor:} \texttt{R<nome> <no1> <no2> <valor>} 
    \item \textbf{Capacitor:} \texttt{C<nome> <no1> <no2> <valor>} 
    \item \textbf{Subcircuito:} \texttt{x<nome> <nos...> <nome\_modelo>} 
    \item \textbf{Fonte de Tensão (V):} 
    \begin{lstlisting}[language=pspice, frame=none]
V<nome> <no+> <no-> <valor_dc>
V<nome> <no+> <no-> PULSE(<v_min> <v_max> <delay> <t_rise> <t_fall> <t_pulse> <periodo>)
V<nome> <no+> <no-> SIN(<offset> <amplitude> <freq> <delay> <amortec> <fase>)
    \end{lstlisting}
\end{itemize}

\subsubsection{Parâmetros Comuns}
\begin{lstlisting}[language=pspice]
* Tensao de alimentacao
.param VDD_value='3.3V'
VDD VDD 0 DC VDD_value

* Parametros de tempo e frequencia
.param freq='10MEG'
.param T='1/freq'

* Capacitancia de carga
.param Cload='50f'
CL OUT 0 Cload
\end{lstlisting}

\subsubsection{Includes de Modelos}
\begin{lstlisting}[language=pspice]
* Modelos para diferentes condicoes
.include "/local/tools/dkit/ams_3.70_mgc/eldo/c35/modeloWP" * Worst Power
.include "/local/tools/dkit/ams_3.70_mgc/eldo/c35/modeloWS" * Worst Speed
.include "/local/tools/dkit/ams_3.70_mgc/eldo/c35/modeloMOD" * Typical

* Modelos de resistor
.include "/local/tools/dkit/ams_3.70_mgc/eldo/c35/restm.mod"
\end{lstlisting}

\subsubsection{Comandos de Análise e Medição}
\begin{lstlisting}[language=pspice]
* Ponto de Operacao DC
.op

* Analise DC (varredura)
.DC <fonte> <inicio> <fim> <passo>

* Analise transiente
.tran <passo_print> <tempo_final> <tempo_inicial> <passo_max>

* Analise com varredura (ex: varrer T)
.tran 0 80n 0 0.01n sweep T 0.4n 0.6n 0.01n

* Condicoes iniciais
.ic V(no)=valor

* Salvar formas de onda
.probe DC V(no1) V(no2)
.probe tran V(no1) I(Vfonte)

* Medir tempos de propagacao (tphl/tplh)
.meas tran <nome> trig V(<in>) val='VDD/2' fall=1 
+                  targ V(<out>) val='VDD/2' rise=1

* Medir potencia media
.meas tran <current> AVG I(Vdd) FROM=<inicio> TO=<fim>
.meas tran <pot> param='<current> * VDD_value'
\end{lstlisting}

\subsubsection{ Dicas Críticas de Simulação (ELDO)}
\begin{itemize}
    \item \textbf{Primeira Linha:} Comente a primeira linha do arquivo \texttt{.cir} (título), pois ela pode não ser lida corretamente.
    \item \textbf{Nomes de Modelo:} Use \texttt{MODN} e \texttt{MODP} (definidos por \texttt{.defmod}) ao invés de \texttt{NMOS4} e \texttt{PMOS4} no netlist, a menos que esteja incluindo os arquivos de modelo completos (como \texttt{cmos53tm.mod}).
    \item \textbf{Netlist PEX:} Lembre-se de editar o netlist PEX para corrigir a linha \texttt{.subckt} e os nomes dos modelos.
\end{itemize}

\subsection{Scripts e Automação}

\subsubsection{Script de Inicialização Rápida}
\begin{lstlisting}[language=bash]
#!/bin/csh
# init_project.csh
cd /local/users/cad/
source .cshrc-mentor
cd /local/users/cad/workavdl
ams_ics -p $1
\end{lstlisting}

\subsubsection{Script de Simulação Automática}
\begin{lstlisting}[language=bash]
#!/bin/csh
# run_simulation.csh
cd /local/users/cad/
source .cshrc-anacad
cd /local/users/cad/workavdl/$1
eldo circuito.cir
ezwave circuito.wdb
\end{lstlisting}

\subsubsection{Comandos TCL para EZWave}
\begin{lstlisting}[language=bash]
# config_ezwave.tcl
wave add V(IN) V(OUT)
xaxis limit 0 100n
yaxis limit -0.5 3.5
cursor add
cursor add
\end{lstlisting}

\subsection{Constantes e Parâmetros Técnicos}

\subsubsection{Mobilidades e Constantes}
\begin{itemize}
\item $\mu_n = 370\ cm^2/V\cdot s$ 
\item $\mu_p = 126\ cm^2/V\cdot s$ 
\item $r = \mu_n/\mu_p \approx 2.94$
\item $C_{ox} \approx 4.6\ fF/\mu m^2$ 
\item $\epsilon_{ox} = 3.5 \times 10^{-13}\ F/cm$ 
\item $t_{ox} = 7.6\ nm$ 
\end{itemize}

\subsubsection{Modelos de Transistores (Valores U0)}
\begin{itemize}
\item \textbf{Típico (tm)}:
 \begin{itemize}
 \item NMOS: U0 = 4.758e+02 
 \item PMOS: U0 = 1.482e+02 
 \item $r \approx 3.21$
 \end{itemize}
\item \textbf{Worst Case Power (wp)}:
 \begin{itemize}
 \item NMOS: U0 = 5.002e+02 
 \item PMOS: U0 = 1.581e+02 
 \item $r \approx 3.16$
 \end{itemize}
\item \textbf{Worst Case Speed (ws)}:
 \begin{itemize}
 \item NMOS: U0 = 4.671e+02 
 \item PMOS: U0 = 1.314e+02 
 \item $r \approx 3.55$
 \end{itemize}
\end{itemize}

\subsubsection{Distâncias Críticas e Cálculos}
\begin{itemize}
\item \textbf{POLY-POLY}: 0.45 $\mu m$ 
\item \textbf{RES-POLY}: 0.35 $\mu m$ 
\item \textbf{DIFF-NTUB}: 1.2 $\mu m$ 
\item \textbf{NTUB enclosure}: 1.2 $\mu m$ 
\item \textbf{MET1 largura VDD/VSS}: 1.8 $\mu m$ (recomendado) 
\item \textbf{ Área/Perímetro (AD/PD):} 
    \begin{itemize}
        \item $AD = 0.85 \times W$
        \item $PD = W + (2 \times 0.85)$ (ou $PD = W + 1.7$)
    \end{itemize}
\end{itemize}

\subsection{Solução de Problemas Comuns}

\subsubsection{Problemas de ICStation}
\begin{itemize}
\item \textbf{Mouse comporta-se como Ctrl pressionado}:
 \begin{itemize}
 \item Solução: Pressione \textbf{Ctrl+Shift} e use as setas do teclado; isso deve normalizar .
 \end{itemize}
\item \textbf{ICStation Travado}:
 \begin{itemize}
 \item Causa: Clicar repetidamente no scroll do mouse.
 \item Solução: Fechar e reabrir.
 \end{itemize}
\item \textbf{Células desaparecem da library}:
 \begin{itemize}
 \item Solução: Crie um novo projeto com o nome antigo e copie a pasta \texttt{default.group} do projeto antigo/backup para a nova pasta do projeto .
 \end{itemize}
\item \textbf{Não consegue colocar ports}:
 \begin{itemize}
 \item Solução: Tente \textbf{DLA Layout $\rightarrow$ Open}.
 \end{itemize}
\end{itemize}

\subsubsection{Problemas de ELDO}
\begin{itemize}
\item \textbf{Erro de modelo não encontrado}:
 \begin{itemize}
 \item Verificar includes dos modelos.
 \item Substituir \texttt{NMOS4/PMOS4} por \texttt{MODN/MODP} ou vice-versa.
 \end{itemize}
\item \textbf{Netlist PEX com erros}:
 \begin{itemize}
 \item Verificar conexões: \texttt{.connect VSS N\_VSS\_...} .
 \item Corrigir resistores: \texttt{rR0} $\rightarrow$ \texttt{XR0}.
 \item Remover parâmetro \texttt{AREA} de bipolares.
 \end{itemize}
\item \textbf{Simulação não converge}:
 \begin{itemize}
 \item Aumentar HMAX.
 \item Adicionar \texttt{.ic} com condições iniciais.
 \item Verificar fontes de alimentação.
 \end{itemize}
\end{itemize}

\subsubsection{Problemas de Calibre}
\begin{itemize}
\item \textbf{Rules não carregam}:
 \begin{itemize}
 \item Verificar path completo do arquivo \texttt{.run}.
 \item \textbf{Load} obrigatório após selecionar.
 \end{itemize}
\item \textbf{LVS não encontra schematic}:
 \begin{itemize}
 \item Verificar se ViewPoint foi criado .
 \item Verificar path do \texttt{vpt\_c35b4\_device} .
 \end{itemize}
\item \textbf{Erros de port}:
 \begin{itemize}
 \item Executar \textbf{Add Text on Ports} novamente .
 \item Verificar camada \texttt{M1NET} .
 \end{itemize}
\end{itemize}

\subsection{Comandos de Debug e Verificação}

\subsubsection{Verificação de Layout}
\begin{lstlisting}[language=bash, frame=none]
# Relatorio de camadas
Report -> Layer -> Design Layers

# Relatorio de selecao
Report -> Selected

# Tamanho da celula
Report -> Windows
\end{lstlisting}

\subsubsection{Verificação de Hierarquia}
\begin{itemize}
\item \textbf{Context $\rightarrow$ Hierarchy $\rightarrow$ Peek $\rightarrow$ 2-4 levels} 
\item \textbf{Edit $\rightarrow$ Flatten} (para desmontar célula) 
\item \textbf{Objects $\rightarrow$ Make $\rightarrow$ Cell} (para definir célula) 
\end{itemize}

\subsubsection{Comandos de History}
\begin{lstlisting}[language=bash]
# Ver comandos recentes
history

# Reexecutar comando
!numero

# Encontrar arquivos
find -name "*.cir"
\end{lstlisting}

\subsection{Fluxos de Trabalho Otimizados}

\subsubsection{Fluxo Completo do Projeto}
\begin{enumerate}
\item \textbf{Terminal}: \texttt{source .cshrc-mentor} + \texttt{ams\_ics}
\item \textbf{Schematic}: Criar e verificar esquemático
\item \textbf{Symbol}: Gerar e configurar símbolo
\item \textbf{ViewPoint}: Criar para netlist
\item \textbf{Layout}: AutoInst + otimizacao + routing
\item \textbf{DRC}: Verificar e corrigir erros
\item \textbf{LVS}: Comparar com schematic
\item \textbf{PEX}: Extrair parasitas
\item \textbf{ELDO}: Preparar e executar simulacao
\item \textbf{EZWave}: Analisar resultados
\end{enumerate}

\subsubsection{Fluxo Rapido para Modificacoes}
\begin{enumerate}
\item Modificar schematic
\item Check schematic
\item Atualizar layout (se necessario)
\item Run DRC + LVS
\item Run PEX
\item Simular e analisar
\end{enumerate}

\subsection{Backup e Versionamento}

\subsubsection{Estrategias de Backup}
\begin{lstlisting}[language=bash]
# Backup completo do projeto
cp -r projeto.proj projeto_backup_date.proj

# Backup incremental
tar -czf projeto_$(date +%Y%m%d).tar.gz projeto.proj

# Backup seletivo (apenas celulas importantes)
cp -r projeto.proj/cell.lib/default.group/logic.views/celula_chave
\end{lstlisting}

\subsubsection{Controle de Versões Manual}
\begin{itemize}
\item \textbf{Nomear versões}: projeto\_v1, projeto\_v2, etc.
\item \textbf{Documentar mudanças}: Arquivo CHANGES.txt
\item \textbf{Backup antes de modificações grandes}
\item \textbf{Testar cada versão} antes de prosseguir
\end{itemize}

\subsection{Dicas de Produtividade}

\subsubsection{Atalhos e Boas Práticas}
\begin{itemize}
\item \textbf{Salvar frequentemente} em todas as ferramentas
\item \textbf{Usar nomes descritivos} para células e sinais
\item \textbf{Documentar} configurações importantes
\item \textbf{Testar incrementalmente} cada parte do circuito
\item \textbf{Manter organização} nos diretórios e arquivos
\end{itemize}

\subsubsection{Otimização de Performance}
\begin{itemize}
\item \textbf{Limitar complexidade} do layout quando possível
\item \textbf{Usar extrações mais simples} (C+CC ao invés de R+C+CC) para debug
\item \textbf{Fechar ferramentas} não utilizadas para liberar memória
\item \textbf{Limitar número de sinais} no EZWave para melhor performance
\end{itemize}

\subsubsection{Workflow em Equipe}
\begin{itemize}
\item \textbf{Comunicar mudanças} em células compartilhadas
\item \textbf{Reservar células} quando estiver trabalhando
\item \textbf{Documentar interfaces} entre diferentes blocos
\item \textbf{Manter padrões consistentes} de nomenclatura
\end{itemize}

\subsection{Comandos de Emergência}

\subsubsection{Recuperação de Projeto}
\begin{lstlisting}[language=bash]
# Se cElulas desaparecerem 
ams_ics -p projeto_copy -t c35b4c3
# Copiar default.group do projeto antigo para o novo

# Se arquivo corrompido
cp backup/projeto_backup.proj ./
\end{lstlisting}

\subsubsection{Limpeza de Arquivos Temporários}
\begin{lstlisting}[language=bash]
# Limpar arquivos de simulaCAOo temporArios
rm -f *.wdb *.log *.out *.err

# Limpar netlists antigos
rm -rf *.cal/
\end{lstlisting}

\subsubsection{Reinicialização de Ambiente}
\begin{lstlisting}[language=bash]
# Se problemas com variaveis de ambiente
exit
# Re-login e re-executar source commands
\end{lstlisting}
\newpage
\newpage
