\section{Tópicos Específicos}


\subsection{Células da CORELIB}

\subsubsection{Células Disponíveis (Exemplos)}
\begin{itemize}
\item \textbf{inv0, inv1}: Inversores básicos.
\item \textbf{df1, df3}: Flip-flops D.
\item \textbf{dl1}: Latch D.
\item \textbf{nand2, nand21, nand22, nand23}: Portas NAND .
\item \textbf{nor2, nor21, nor23}: Portas NOR .
\item \textbf{xor2}: Porta XOR.
\item \textbf{and2}, \textbf{or2}.
\end{itemize}

\subsubsection{Uso no Schematic}
\begin{enumerate}
\item \textbf{Add $\rightarrow$ Instance $\rightarrow$ Choose Symbol} .
\item \textbf{Biblioteca}: CORELIB.
\item Selecionar célula desejada (ex: \texttt{df1}).
\item Conectar conforme necessário.
\item \textbf{Verificar}: Check Schematic (0 errors, 0 warnings).
\end{enumerate}

\subsubsection{Uso no Layout}
\begin{itemize}
\item \textbf{ICStudio $\rightarrow$ Library $\rightarrow$ CORELIB $\rightarrow$ célula}.
\item Já inclui layout pré-validado (célula-padrão).
\item \textbf{DRC/LVS}: Geralmente já verificado.
\end{itemize}

\subsection{Adição de PADs}

\subsubsection{Procedimento Completo}
\begin{enumerate}
\item \textbf{Objects $\rightarrow$ Add $\rightarrow$ Cell}.
\item \textbf{Biblioteca}: \textbf{IOLIB\_4M} .
\item \textbf{Célula}: \textbf{g\_padonly} (último item).
\item Colocar um PAD para VDD e um para VSS.
\item \textbf{Conectar com shapes}: 
 \begin{itemize}
 \item Puxar shapes de metal dos PADs até os pontos de conexão.
 \item \textbf{Grossura}: O mais grosso possível (reduzir se der erro DRC).
 \item \textbf{Materiais}: Usar MET1 ou MET2.
 \end{itemize}
\item \textbf{Verificar DRC}: Corrigir erros de espaçamento.
\end{enumerate}

\subsubsection{Configuração de PADs}
\begin{itemize}
\item \textbf{Orientação}: Posicionar nas bordas do chip.
\item \textbf{Alimentação}: PAD VDD e VSS devem ser robustos.
\item \textbf{Sinais}: PADs para entradas/saídas importantes.
\item \textbf{ESD}: Proteção geralmente incluída nos PADs.
\end{itemize}

\subsection{Criação de Hierarquia}

\subsubsection{Hierarquia no Schematic}
\begin{enumerate}
\item Criar schematic de bloco inferior (ex: \texttt{inversor}).
\item Gerar símbolo e configurar a propriedade \textbf{phy\_comp} (ver abaixo).
\item No schematic superior: \textbf{Add $\rightarrow$ Instance $\rightarrow$ Choose Symbol}.
\item Selecionar símbolo do bloco inferior (ex: \texttt{inversor}).
\item Conectar ports e alimentação.
\item \textbf{Check Schematic} (0 errors, 0 warnings).
\end{enumerate}

\subsubsection{ Ligação Símbolo-Layout (phy\_comp)}
Esta é a etapa crucial que conecta o símbolo do esquemático ao seu layout físico.
\begin{enumerate}
    \item \textbf{Copiar Localização do Layout:}
    \begin{itemize}
        \item No \textbf{ICStation}, clique com o \textbf{botão direito} na \textit{view} do seu layout.
        \item \textbf{Properties $\rightarrow$ Location}.
        \item Copie o caminho exibido (ex: \texttt{\$minha\_lib/default.group/layout.views/inv}) .
    \end{itemize}
    \item \textbf{Adicionar Propriedade ao Símbolo:}
    \begin{itemize}
        \item Volte ao \textbf{Design Architect} e abra o \textbf{SÍMBOLO} (\texttt{File $\rightarrow$ Open $\rightarrow$ Symbol}).
        \item \textbf{Comando:} \textbf{Add $\rightarrow$ Properties}.
        \item \textbf{Property Name}: \texttt{phy\_comp}.
        \item \textbf{Property Value}: Cole a \textbf{Location} copiada .
    \end{itemize}
    \item \textbf{Ajustar Texto e Verificar (A "Parte Mágica"):}
    \begin{itemize}
        \item \textbf{Setup $\rightarrow$ Select Filter $\rightarrow$ Properties} $\rightarrow$ OK .
        \item Mude a altura do texto (ex: \textbf{Change Height $\rightarrow$ Specified $\rightarrow$ 0.2}) .
        \item \textbf{File $\rightarrow$ Check Symbol} (1 warning é esperado) .
        \item \textbf{File $\rightarrow$ Save Symbol}.
        \item \textbf{File $\rightarrow$ Check Symbol} (Agora 0 warnings são esperados) .
    \end{itemize}
\end{enumerate}

\subsubsection{Hierarquia no Layout}
\begin{enumerate}
\item Criar layout do bloco inferior (com DRC/LVS limpos).
\item No layout superior: \textbf{Objects $\rightarrow$ Add $\rightarrow$ Cell}.
\item Selecionar célula do bloco inferior (que agora está ligada ao símbolo).
\item Posicionar e conectar (usando \textbf{Place \& Route} para Standard Cells ou manualmente).
\item \textbf{Verificar}: DRC e LVS hierárquicos.
\end{enumerate}

\subsubsection{Contexto Hierárquico}
\begin{itemize}
\item \textbf{Context $\rightarrow$ Hierarchy $\rightarrow$ Peek $\rightarrow$ N levels}: Ver "dentro" do layout do bloco.
\item \textbf{Edit $\rightarrow$ Flatten}: Desmontar hierarquia (cuidado).
\item \textbf{Objects $\rightarrow$ Make $\rightarrow$ Cell}: Criar célula a partir de shapes.
\end{itemize}

\subsection{Resistores em Poly}

\subsubsection{Adição no Schematic}
\begin{enumerate}
\item \textbf{HIT-KIT Utilities $\rightarrow$ Devices $\rightarrow$ Resistors}.
\item \textbf{rpolyh}: Resistor poly de alta resistividade.
\item \textbf{Parâmetros}:
 \begin{itemize}
 \item \textbf{R}: Valor da resistência.
 \item \textbf{L}: Comprimento (calculado automaticamente).
 \item \textbf{W}: Largura.
 \item \textbf{Bends}: Número de dobras (para layout compacto).
 \end{itemize}
\end{enumerate}

\subsubsection{Layout de Resistores}
\begin{itemize}
\item \textbf{AutoInst}: Gera layout automaticamente.
\item \textbf{Dobramento}: \textbf{Object $\rightarrow$ Change $\rightarrow$ Device $\rightarrow$ Bend}.
\item \textbf{Distâncias}: Respeitar espaçamento POLY-RES (0.35$\mu$m).
\item \textbf{Contatos}: Usar MET1 para conexões.
\end{itemize}

\subsubsection{Modelo ELDO}
\begin{lstlisting}[language=pspice]
* Incluir modelo de resistor
.include "/local/tools/dkit/ams_3.70_mgc/eldo/c35/restm.mod"

* Instancia no PEX (vem como 'rR0', trocar para 'XR0')
XR0 n1 n2 rpolyh r=1k
\end{lstlisting}

\subsection{Fontes de Corrente e Espelhos}

\subsubsection{Implementação Básica}
\begin{itemize}
\item \textbf{Transistores em diodo}: Para referência.
\item \textbf{Espelho simples}: Copiar corrente.
\item \textbf{Espelho cascode} ou \textbf{Wilson}: Maior impedância de saída, melhor topologia para fontes de corrente (conforme Prova 2016b, Q9) .
\item \textbf{Polarização}: Usar resistores ou divisores.
\end{itemize}

\subsubsection{Dimensionamento}
\begin{itemize}
\item \textbf{Corrente de referência}: Definida por resistor ou fonte.
\item \textbf{Relação W/L}: Define razão de espelhamento.
\item \textbf{Tensão Early}: Considerar para precisão.
\item \textbf{Matching}: Transistores próximos no layout.
\end{itemize}

\subsubsection{Exemplo ELDO}
\begin{lstlisting}[language=pspice]
* Espelho de corrente simples 1:2
Mref ref ref VSS VSS MODN w=10u l=1u
Mout out ref VSS VSS MODN w=20u l=1u
Iref VDD ref DC 100uA
\end{lstlisting}

\subsection{Circuitos C2MOS}

\subsubsection{Características}
\begin{itemize}
\item \textbf{Latch dinâmico}: Mantém estado quando clock desabilitado.
\item \textbf{Clock = Alto (3V)}: $M_{P1}$ e $M_{N1}$ estão conduzindo e o circuito se comporta como um inversor.
\item \textbf{Clock = Baixo (0V)}: $M_{P1}$ e $M_{N1}$ estão cortados e a carga é mantida no capacitor de saída (ex: gate de outra porta).
\item \textbf{Aplicações}: Registradores, pipelines.
\end{itemize}

\subsubsection{Implementação}
\begin{itemize}
\item \textbf{Transistores de passagem} (em série): $M_{P1}$ e $M_{N1}$ controlados pelo clock.
\item \textbf{Inversor principal}: $M_{P2}$ e $M_{N2}$ controlados pela entrada \texttt{In}.
\item \textbf{Bulk}: Conectar corretamente (PMOS em VDD, NMOS em VSS).
\end{itemize}

\subsubsection{Dimensionamento}
\begin{itemize}
\item \textbf{Transistores de passagem}: L mínimo para reduzir carga.
\item \textbf{Inversor}: Dimensionar para tempos equivalentes (subida/descida).
\item \textbf{Clock}: Considerar sobreposição (non-overlap).
\end{itemize}

\subsection{Osciladores em Anel}

\subsubsection{Configuração Básica}
\begin{itemize}
\item \textbf{N estágios}: Número \textbf{ímpar} de inversores.
\item \textbf{Frequência}: $f = \frac{1}{2 \cdot N \cdot t_{pd}}$ (onde $t_{pd}$ é o atraso médio do inversor).
\item \textbf{Condição de oscilação}: Ganho > 1 por estágio.
\item \textbf{Alimentação}: Estável para frequência constante.
\end{itemize}

\subsubsection{Implementação}
\begin{itemize}
\item \textbf{Célula básica}: Inversor ou porta CMOS (ex: NAND com uma entrada em VDD).
\item \textbf{Controle de frequência}: Tensão de alimentação ou capacidades de carga.
\item \textbf{Saída}: Usar um buffer para evitar que a carga de medição afete a oscilação.
\end{itemize}

\subsubsection{Simulação}
\begin{itemize}
\item \textbf{Condições iniciais}: \texttt{.ic} pode ser necessário para "chutar" a oscilação.
\item \textbf{Tempo de simulação}: Múltiplos períodos.
\item \textbf{Medição}: Período e frequência.
\end{itemize}

\subsection{Camadas Especiais}

\subsubsection{HRES (High Resistance)}
\begin{itemize}
\item \textbf{Função}: Usada para construir resistores de polisilício.
\item \textbf{Processo}: Define uma área que isola o \texttt{POLY1} e impede que ele seja dopado (aumentando sua resistividade) .
\item \textbf{Uso}: Definir o corpo do resistor \texttt{rpolyh}.
\item \textbf{DRC}: Verificar espaçamentos (ex: \texttt{RES-POLY}).
\end{itemize}

\subsubsection{NLDD e FIMP}
\begin{itemize}
\item \textbf{NLDD (N-Lightly Doped Drain)}: Reduz campos elétricos na junção dreno-bulk.
\item \textbf{FIMP}: Implantação para ajuste fino de $V_{th}$ (Threshold Voltage).
\item \textbf{Geração}: \textbf{HIT-Kit Utilities $\rightarrow$ Generated Layers} .
\item \textbf{Verificação}: Rodar DRC novamente após a geração .
\end{itemize}

\subsubsection{ Camadas Físicas (Exemplos)}
Os nomes das camadas no processo :
\begin{itemize}
    \item \textbf{ntub}: Poço N (Well) para PMOS.
    \item \textbf{diff}: Região ativa (onde dreno/source são formados).
    \item \textbf{nplus / pplus}: Implantações N+ e P+ para Dreno/Source.
    \item \textbf{poly1}: Primeira camada de Polisilício (Gates).
    \item \textbf{cont}: Contato entre \texttt{diff} ou \texttt{poly1} e \texttt{met1}.
    \item \textbf{met1}: Primeira camada de Metal.
    \item \textbf{via1}: Contato entre \texttt{met1} e \texttt{met2}.
    \item \textbf{pad}: Abertura na passivação para contato externo (Bonding pad).
\end{itemize}

\subsection{Técnicas de Layout Avançadas}

\subsubsection{Match de Transistores (Casamento)}
Essencial para circuitos analógicos (espelhos de corrente, pares diferenciais).
\begin{itemize}
\item \textbf{Orientação comum}: Todos os transistores na mesma direção.
\item \textbf{Proximidade}: Transistores pareados devem ser colocados próximos.
\item \textbf{Interdigitação}: Alternar "dedos" de transistores (ex: A-B-A-B).
\item \textbf{Comum centroid (Centro Comum)}: Layout simétrico ao redor de um ponto central.
\end{itemize}

\subsubsection{Redução de Parasitas}
\begin{itemize}
\item \textbf{Metais curtos}: Minimizar comprimento de interconexões.
\item \textbf{Shielding}: Blindar sinais sensíveis (ex: analógicos) com linhas de VDD/VSS.
\item \textbf{Vias múltiplas}: Usar várias vias para conexões de alta corrente (reduz R e melhora confiabilidade).
\item \textbf{Orientação}: Evitar cruzamentos desnecessários; se inevitável, cruzar em camadas diferentes (ex: MET1 e MET2).
\end{itemize}

\subsubsection{Otimização de Área}
\begin{itemize}
\item \textbf{Merge agressivo}: Juntar drenos/sources sempre que possível .
\item \textbf{Dobramento (Fold)}: Usar \texttt{fold} para transistores largos .
\item \textbf{Compactação}: Mover blocos para o mais próximo possível (respeitando o DRC).
\item \textbf{Roteamento inteligente}: Planejar o roteamento para minimizar a área de "fios".
\end{itemize}

\subsection{Fluxos Especiais}

\subsubsection{Fluxo Standard Cells}
\begin{enumerate}
\item \textbf{Place \& Route $\rightarrow$ Autofp}.
\item \textbf{Aspect Ratio}: 2 em Upper.
\item \textbf{StdCell}: Selecionar área.
\item \textbf{Route}: Configurar net classes para VDD/VSS (ex: 1.8$\mu$m) .
\item \textbf{Verificação}: DRC e LVS hierárquico.
\end{enumerate}

\subsubsection{Fluxo com Células Personalizadas}
\begin{enumerate}
\item Criar células personalizadas (ex: porta lógica) com DRC/LVS limpos.
\item Adicionar ao schematic superior como instâncias.
\item Gerar layout hierárquico (os blocos aparecerão como caixas).
\item Conectar as células com roteamento (manual ou automático).
\item Verificação completa (DRC, LVS).
\end{enumerate}

\subsubsection{Fluxo Analógico-Digital}
\begin{enumerate}
\item \textbf{Blocos analógicos}: Layout manual com \textit{matching} (interdigitação, etc.).
\item \textbf{Blocos digitais}: Standard cells ou auto place \& route.
\item \textbf{Interface}: Cuidado com acoplamento de ruído digital para o analógico.
\item \textbf{Alimentação}: Idealmente, separar VDD/VSS analógico e digital.
\item \textbf{Verificação}: DRC, LVS e extração PEX completa.
\end{enumerate}

\subsection{Problemas Específicos e Soluções}

\subsubsection{Latch-up}
\begin{itemize}
\item \textbf{Causa}: Acionamento de um SCR (tiristor) parasitário (PNPN) entre VDD e VSS, causando um curto-circuito.
\item \textbf{Prevenção}:
 \begin{itemize}
 \item \textbf{Guard rings}: Anéis de proteção (N+ em poço N, P+ em substrato P) ao redor de blocos.
 \item \textbf{Espaçamento}: Distância adequada entre NMOS (PTUB) e PMOS (NTUB).
 \item \textbf{Substrate contacts}: Usar muitos contatos de poço (bulks) para "amarrar" o potencial do substrato ao VDD/VSS.
 \end{itemize}
\item \textbf{Verificação}: DRC para espaçamentos críticos (regra NTUB/PTUB).
\end{itemize}

\subsubsection{Electromigration}
\begin{itemize}
\item \textbf{Causa}: Corrente excessiva "empurra" os átomos de metal em trilhas finas, criando aberturas (falhas).
\item \textbf{Prevenção}:
 \begin{itemize}
 \item \textbf{Largura de metal}: Usar trilhas largas para VDD/VSS (ex: 1.8$\mu$m) .
 \item \textbf{Vias múltiplas}: Usar várias vias para conexões de alta corrente.
 \end{itemize}
\item \textbf{Verificação}: DRC verifica regras de largura mínima de metal (width rules).
\end{itemize}

\subsubsection{Cross-talk (Acoplamento)}
\begin{itemize}
\item \textbf{Causa}: Acoplamento capacitivo entre sinais (trilhas) adjacentes.
\item \textbf{Prevenção}:
 \begin{itemize}
 \item \textbf{Shielding}: Blindar sinais sensíveis (ex: analógicos, clock) com linhas de VDD/VSS ao lado.
 \item \textbf{Espaçamento}: Aumentar distância entre sinais críticos.
 \item \textbf{Orientação}: Cruzar sinais em ângulos retos (usando camadas diferentes, ex: MET1 vs MET2).
 \end{itemize}
\item \textbf{Verificação}: PEX com extração \textbf{C+CC} (Capacitância de Acoplamento) .
\end{itemize}

\subsection{Exemplos de Células Complexas}

\subsubsection{Flip-flop D Master-Slave}
\begin{itemize}
\item \textbf{Composição}: 2 latches (ex: C2MOS ou com realimentação) em série.
\item \textbf{Clock}: Clocks $\phi$ e $\bar{\phi}$ (não sobrepostos) para evitar \textit{race conditions}.
\item \textbf{Reset}: Pode ser assíncrono ou síncrono.
\item \textbf{Layout}: Cuidado com o roteamento e distribuição do(s) sinal(is) de clock.
\end{itemize}

\subsubsection{Buffer de Potência (Driver)}
\begin{itemize}
\item \textbf{Função}: Dirigir cargas capacitivas grandes (ex: pinos de I/O, linhas de clock longas).
\item \textbf{Estágios múltiplos}: Uma cadeia de inversores com W/L progressivamente maiores.
\item \textbf{Dimensionamento}: Razão de "tapering" constante (fator 'f') entre os estágios.
\item \textbf{Layout}: Transistores de saída muito largos (usando \textit{fold}) e múltiplas vias para VDD/VSS.
\end{itemize}

\subsubsection{Comparador Analógico}
\begin{itemize}
\item \textbf{Amplificador diferencial}: Estágio de entrada (alta impedância, rejeição de modo comum).
\item \textbf{Estágio de ganho}: Amplificação do pequeno sinal de diferença.
\item \textbf{Latch de saída}: Converte o sinal analógico amplificado em um sinal digital (0 ou 1).
\item \textbf{Layout}: \textit{Matching} (casamento) é crítico no par diferencial de entrada.
\end{itemize}

\subsection{Técnicas de Verificação Avançada}

\subsubsection{LVS com Células Personalizadas (Hierarquia)}
\begin{itemize}
\item \textbf{phy\_comp}: A propriedade é essencial. O LVS usará o layout apontado por \texttt{phy\_comp} em vez de procurar transistores dentro do símbolo .
\item \textbf{Hierarquia}: Verificar todos os níveis (LVS hierárquico).
\item \textbf{Interfaces}: Conferir se os ports (VDD, VSS, A, B) do bloco correspondem ao esquemático.
\item \textbf{Problemas comuns}: Bulk não conectado dentro do bloco, ports trocados.
\end{itemize}

\subsubsection{ERC (Electrical Rule Checking)}
O DRC moderno (Calibre) já inclui muitas verificações de ERC:
\begin{itemize}
\item \textbf{Gates flutuantes} (ex: erro \texttt{ILL\_FLOATING\_GATE\_ERC}).
\item \textbf{Bulk connections} (ex: erro \texttt{NWELL TOO HOT}).
\item \textbf{Short circuits} (curtos entre VDD e VSS).
\item \textbf{Open circuits} (nets não conectados).
\end{itemize}

\subsubsection{Análise de Performance}
\begin{itemize}
\item \textbf{Atraso}: Extrair $t_{pd}$ (média de $t_{phl}$ e $t_{plh}$).
\item \textbf{Potência}: Consumo estático (vazamento) e dinâmico ($C \cdot V^2 \cdot f$).
\item \textbf{Margem de ruído}: Imunidade a interferências.
\item \textbf{Temperatura}: Efeito na performance (simulação \texttt{.DC TEMP}).
\end{itemize}

\subsection{Scripts e Automação Avançada}

\subsubsection{Batch DRC/LVS}
\begin{lstlisting}[language=bash]
# Script para verificacao em lote
foreach cell (cell1 cell2 cell3)
  calibre -drc $cell
  calibre -lvs $cell
end
\end{lstlisting}

\subsubsection{Extração Automática de Parâmetros}
\begin{lstlisting}[language=bash]
# Extrair W/L de todos os transistores
grep "M.*MOD" schematic.netlist | awk '{print $8, $9}'
\end{lstlisting}

\subsubsection{Geração de Relatórios}
\begin{lstlisting}[language=bash, frame=none]
# (Exemplo de scripts hipoteticos)
report_area.tcl
report_performance.tcl
\end{lstlisting}

\subsection{Migração de Tecnologia}

\subsubsection{Ajustes Necessários}
\begin{itemize}
\item \textbf{Regras de design}: Novas distâncias mínimas.
\item \textbf{Modelos de transistor}: Novos parâmetros (ex: $V_{th}$, $C_{ox}$).
\item \textbf{Capacitâncias}: Valores diferentes.
\item \textbf{Resistências}: Novos valores de \textit{sheet resistance}.
\end{itemize}

\subsubsection{Procedimento}
\begin{enumerate}
\item \textbf{Revisar regras}: Novas regras DRC.
\item \textbf{Ajustar dimensões}: W/L se necessário.
\item \textbf{Re-rotear}: Se necessário para novas regras.
\item \textbf{Re-verificar}: DRC, LVS, simulação.
\end{enumerate}

\subsection{Documentação e Manutenção}

\subsubsection{Documentação de Células}
\begin{itemize}
\item \textbf{Esquemático}: Diagrama limpo e legível.
\item \textbf{Símbolo}: Pins organizados logicamente.
\item \textbf{Layout}: Com medidas críticas anotadas.
\item \textbf{Performance}: Parâmetros extraídos (atrasos, potência).
\end{itemize}

\subsubsection{Versionamento}
\begin{itemize}
\item \textbf{Nomenclatura}: celula\_v1, celula\_v2.
\item \textbf{Log de mudanças}: O que foi modificado e porquê.
\item \textbf{Backup}: Versões estáveis arquivadas.
\item \textbf{Testes}: Resultados de simulação documentados.
\end{itemize}

\subsubsection{Manutenção}
\begin{itemize}
\item \textbf{Atualização}: Ajustes para novas versões das ferramentas.
\item \textbf{Otimização}: Melhorias contínuas de performance.
\item \textbf{Bug fixes}: Correção de problemas identificados.
\item \textbf{Documentação}: Manter documentação atualizada.
\end{itemize}


