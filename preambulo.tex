% !TeX root = main.tex
% !TeX program = lualatex
% ####################################################################
% #                     PREÂMBULO DO DOCUMENTO                       #
% ####################################################################
\pdfvariable minorversion=7
\documentclass[a4paper, 11pt]{article}

% --- PACOTES BÁSICOS ---
\usepackage{multirow}
\usepackage{subcaption}
\usepackage[brazil]{babel}
\usepackage{amsmath, amssymb}       % Pacotes de matemática
\usepackage{graphicx}
\usepackage[margin=1.5cm]{geometry} % Margens conforme originais
\usepackage{parskip}                % Espaçamento entre parágrafos
\usepackage{float}                  % Para usar [H]

% --- FONTES (ESSENCIAL PARA LUALATEX) ---
\usepackage{fontspec} 

\setmonofont{DejaVu Sans Mono}[Scale=0.9]

% --- PACOTES DE TIPOGRAFIA E MELHORIAS ---
\usepackage{microtype} % Melhora justificação e espaçamento
\usepackage{fancyhdr}  % Cabeçalhos e rodapés
\usepackage{booktabs}  % Tabelas profissionais
\usepackage{array}     % Controle avançado de tabelas

% --- PACOTES PARA LISTAGEM DE CÓDIGO ---
\usepackage{xcolor}
\usepackage{listings}

% --- PACOTES DE NAVEGAÇÃO (CARREGAR POR ÚLTIMO) ---
\usepackage[hidelinks]{hyperref}

% --- DEFINIÇÃO DAS CORES PARA O CÓDIGO ---
\definecolor{codegray}{rgb}{0.95,0.95,0.95}
\definecolor{commentgreen}{rgb}{0,0.5,0}
\definecolor{keywordblue}{rgb}{0.0,0.0,0.6}
\definecolor{stringpurple}{rgb}{0.5,0.0.5}
\definecolor{bashcommand}{rgb}{0.6,0.1,0.1}

% --- CONFIGURAÇÃO GLOBAL DO `listings` ---
\lstset{
    backgroundcolor=\color{white},
    commentstyle=\color{commentgreen},
    keywordstyle=\bfseries\color{keywordblue},
    stringstyle=\color{stringpurple},
    basicstyle=\small\ttfamily,      % \small ajuda a caber linhas longas
    showstringspaces=false,
    breaklines=true,
    frame=single,
    framerule=0.5pt,
    rulecolor=\color{black},
    tabsize=4,
    numbers=none,
    numberstyle=\tiny\color{gray},
    % --- CONFIGURAÇÃO LUALATEX ---
    % Não usamos inputencoding nem extendedchars aqui.
    % Apenas tratamos símbolos especiais que não estão na fonte mono:
    literate={Ω}{{$\Omega$}}1 {µ}{{$\mu$}}1 {°}{{$^\circ$}}1,
}

% --- DEFINIÇÃO DE LINGUAGENS CUSTOMIZADAS ---

% 1. Linguagem para PSpice (SPICE)
\lstdefinelanguage{pspice}{%
    sensitive=false,
    morekeywords={.param,.global,.subckt,.ends,.lib,.include,.tran,.dc,.ac,.meas,.mc,.option,.plot,.probe,.ic,.nodeset,
    VDD,VSS,PULSE,sweep,trig,targ,rise,fall,param,abs,
    WHEN,FIND,AVG,STDDEV,MAX,MIN,INCR,
    MODN,MODP,RPOLYH,VERT10,
    NMOS4,PMOS4,V,I,Is,Vout,Vdd},
    morecomment=[l]{*},
}

% 2. Linguagem para Bash/Terminal
\lstdefinelanguage{bash}{%
    sensitive=true,
    morekeywords={cd, source, ams_ics, eldo, ezwave, calibre, xwd, convert,
    grep, find, cat, touch, vim, nano, cp, rm, mkdir,
    DRC, LVS, PEX, history, exit, python3, latexmk},
    keywordstyle=\bfseries\color{bashcommand},
    morecomment=[l]{\#},
    morestring=[b]",
    morestring=[b]',
}

% --- CONFIGURAÇÃO DO CABEÇALHO E RODAPÉ ---
\pagestyle{fancy}
\fancyhf{}
\lhead{\footnotesize \nouppercase{\leftmark}} 
\rhead{\footnotesize \nouppercase{\rightmark}} 
\cfoot{\footnotesize \thepage}
\renewcommand{\headrulewidth}{0.4pt}

% Ajuste de marcas
\renewcommand{\sectionmark}[1]{\markboth{#1}{}}
\renewcommand{\subsectionmark}[1]{\markright{#1}}

% ####################################################################
% #           INFORMAÇÕES DA CAPA (DEFINIÇÕES GLOBAIS)               #
% ####################################################################

\title{Material de Consulta\\ \large SEL0621 - Projeto de Circuitos Integrados Digitais I}

\author{
    Felipi Adenildo Soares Sousa \\ felipiadenildo@usp.br
}

\newcommand{\professor}{Prof. Dr. João Navarro Soares Júnior \\ navarro@sc.usp.br}
\newcommand{\dataEntrega}{13 de novembro de 2025}

% ####################################################################
% #                        FIM DO PREÂMBULO                          #
% ####################################################################

\graphicspath{{./imagens/}}
\usepackage{pdfpages}