
% ####################################################################
% #                     PREÂMBULO DO DOCUMENTO                       #
% ####################################################################

\documentclass[a4paper, 11pt]{article}
% regex \

% --- PACOTES BÁSICOS ---
\usepackage{multirow}
\usepackage{subcaption}
\usepackage[utf8]{inputenc}
\usepackage[T1]{fontenc}
\usepackage[brazil]{babel}
\usepackage{amsmath, amssymb} % Pacotes de matemática
\usepackage{graphicx}
\usepackage[margin=1.5cm]{geometry} % Margens conforme originais
\usepackage{parskip}                % Espaçamento entre parágrafos (sem indentação)
\usepackage{float}                  % Para usar [H]

% --- PACOTES DE TIPOGRAFIA E MELHORIAS ---
\usepackage{microtype} % Melhora sutilmente a justificação e o espaçamento
\usepackage{fancyhdr}  % Para cabeçalhos e rodapés
\usepackage{booktabs}  % Para tabelas profissionais
\usepackage{array}     % Para controle avançado de colunas em tabelas

% --- PACOTES PARA LISTAGEM DE CÓDIGO ---
\usepackage{xcolor}   % Necessário para definir cores
\usepackage{listings} % O pacote de code listing

% --- PACOTES DE NAVEGAÇÃO (CARREGAR POR ÚLTIMO) ---
\usepackage[hidelinks]{hyperref} % Para índice clicável e links internos

% --- DEFINIÇÃO DAS CORES PARA O CÓDIGO ---
\definecolor{codegray}{rgb}{0.95,0.95,0.95}
\definecolor{commentgreen}{rgb}{0,0.5,0}
\definecolor{keywordblue}{rgb}{0.0,0.0,0.6}
\definecolor{stringpurple}{rgb}{0.5,0.0.5}
\definecolor{bashcommand}{rgb}{0.6,0.1,0.1} % Nova cor para comandos bash

% --- CONFIGURAÇÃO GLOBAL DO `listings` ---
% Esta é a aparência base para TODOS os blocos de código
\lstset{
 backgroundcolor=\color{white},         % Fundo branco
 commentstyle=\color{commentgreen},     % Cor dos comentários
 keywordstyle=\bfseries\color{keywordblue},% Estilo das palavras-chave
 stringstyle=\color{stringpurple},        % Cor das strings
 basicstyle=\ttfamily,                  % Fonte de máquina de escrever simples
 showstringspaces=false,                % Não mostrar espaços em strings
 breaklines=true,                       % Quebra de linha automática
 frame=single,                          % Borda simples
 framerule=0.5pt,
 rulecolor=\color{black},
 tabsize=4,
 numbers=none,                          % Desligar números de linha por padrão
 numberstyle=\tiny\color{gray},
 inputencoding=utf8,
 extendedchars=true,
 literate={Ω}{{$\Omega$}}1 {µ}{{$\mu$}}1, % Adicione esta linha!
}

% --- DEFINIÇÃO DE LINGUAGENS CUSTOMIZADAS ---

% 1. Linguagem para PSpice (SPICE)
\lstdefinelanguage{pspice}{%
 sensitive=false, % pspice não é case-sensitive
 morekeywords={.param, .global, .subckt, .ends, .lib, .include, .tran,
 .dc, .ac, .meas, .mc, .option, .plot, .probe, .ic, .nodeset,
 VDD, VSS, PULSE, sweep, trig, targ, rise, fall, param, abs,
 WHEN, FIND, AVG, STDDEV, MAX, MIN, INCR,
 MODN, MODP, RPOLYH, VERT10,
 NMOS4, PMOS4, V, I, Is, Vout, Vdd},
 morecomment=[l]{*}, % Comentários começam com *
}

% 2. Linguagem para Bash/Terminal
\lstdefinelanguage{bash}{%
 sensitive=true,
 morekeywords={cd, source, ams_ics, eldo, ezwave, calibre, xwd, convert,
 grep, find, cat, touch, vim, nano, cp, rm, mkdir,
 DRC, LVS, PEX, history, exit, touch,},
 keywordstyle=\bfseries\color{bashcommand},
 morecomment=[l]{\#}, % Comentários começam com #
 morestring=[b]", % Strings em aspas duplas
 morestring=[b]', % Strings em aspas simples
}

% --- CONFIGURAÇÃO DO CABEÇALHO E RODAPÉ ---
\pagestyle{fancy}
\fancyhf{}
\lhead{\footnotesize \nouppercase{\leftmark}} % Mostra a Seção atual
\rhead{\footnotesize \nouppercase{\rightmark}} % Mostra a Subseção atual
\cfoot{\footnotesize \thepage}
\renewcommand{\headrulewidth}{0.4pt}

% --- AJUSTE PARA \leftmark E \rightmark FUNCIONAREM COM \section ---
\renewcommand{\sectionmark}[1]{\markboth{#1}{}}
\renewcommand{\subsectionmark}[1]{\markright{#1}}


% ####################################################################
% #           INFORMAÇÕES DA CAPA (DEFINIÇÕES GLOBAIS)               #
% ####################################################################
%
% Estas definições alimentam a \begin{titlepage} abaixo.

\title{Material de Consulta\\ \large SEL0621 - Projeto de Circuitos Integrados Digitais I}

\author{
    Felipi Adenildo Soares Sousa \\ felipiadenildo(at)usp.br
}

\newcommand{\professor}{Prof. Dr. João Navarro Soares Júnior \\ navarro(at)sc.usp.br}
\newcommand{\dataEntrega}{13 de novembro de 2025}

% ####################################################################
% #                        FIM DO PREÂMBULO                          #
% ####################################################################



% Configuração adicionada pelo script
\graphicspath{{./imagens/}}
\usepackage{pdfpages}
