% ####################################################################
% #                     PREÂMBULO DO DOCUMENTO                       #
% ####################################################################

\documentclass[a4paper, 11pt]{article}
% regex \

% --- PACOTES BÁSICOS ---
\usepackage{multirow}
\usepackage{subcaption}
\usepackage[utf8]{inputenc}
\usepackage[T1]{fontenc}
\usepackage[brazil]{babel}
\usepackage{amsmath, amssymb} % Pacotes de matemática
\usepackage{graphicx}
\usepackage[margin=1.5cm]{geometry} % Margens conforme originais
\usepackage{parskip}                % Espaçamento entre parágrafos (sem indentação)
\usepackage{float}                  % Para usar [H]

% --- PACOTES DE TIPOGRAFIA E MELHORIAS ---
\usepackage{microtype} % Melhora sutilmente a justificação e o espaçamento
\usepackage{fancyhdr}  % Para cabeçalhos e rodapés
\usepackage{booktabs}  % Para tabelas profissionais
\usepackage{array}     % Para controle avançado de colunas em tabelas

% --- PACOTES PARA LISTAGEM DE CÓDIGO ---
\usepackage{xcolor}   % Necessário para definir cores
\usepackage{listings} % O pacote de code listing

% --- PACOTES DE NAVEGAÇÃO (CARREGAR POR ÚLTIMO) ---
\usepackage[hidelinks]{hyperref} % Para índice clicável e links internos

% --- DEFINIÇÃO DAS CORES PARA O CÓDIGO ---
\definecolor{codegray}{rgb}{0.95,0.95,0.95}
\definecolor{commentgreen}{rgb}{0,0.5,0}
\definecolor{keywordblue}{rgb}{0.0,0.0,0.6}
\definecolor{stringpurple}{rgb}{0.5,0.0.5}
\definecolor{bashcommand}{rgb}{0.6,0.1,0.1} % Nova cor para comandos bash

% --- CONFIGURAÇÃO GLOBAL DO `listings` ---
% Esta é a aparência base para TODOS os blocos de código
\lstset{
 backgroundcolor=\color{white},         % Fundo branco
 commentstyle=\color{commentgreen},     % Cor dos comentários
 keywordstyle=\bfseries\color{keywordblue},% Estilo das palavras-chave
 stringstyle=\color{stringpurple},        % Cor das strings
 basicstyle=\ttfamily,                  % Fonte de máquina de escrever simples
 showstringspaces=false,                % Não mostrar espaços em strings
 breaklines=true,                       % Quebra de linha automática
 frame=single,                          % Borda simples
 framerule=0.5pt,
 rulecolor=\color{black},
 tabsize=4,
 numbers=none,                          % Desligar números de linha por padrão
 numberstyle=\tiny\color{gray},
}

% --- DEFINIÇÃO DE LINGUAGENS CUSTOMIZADAS ---

% 1. Linguagem para PSpice (SPICE)
\lstdefinelanguage{pspice}{%
 sensitive=false, % pspice não é case-sensitive
 morekeywords={.param, .global, .subckt, .ends, .lib, .include, .tran,
 .dc, .ac, .meas, .mc, .option, .plot, .probe, .ic, .nodeset,
 VDD, VSS, PULSE, sweep, trig, targ, rise, fall, param, abs,
 WHEN, FIND, AVG, STDDEV, MAX, MIN, INCR,
 MODN, MODP, RPOLYH, VERT10,
 NMOS4, PMOS4, V, I, Is, Vout, Vdd},
 morecomment=[l]{*}, % Comentários começam com *
}

% 2. Linguagem para Bash/Terminal
\lstdefinelanguage{bash}{%
 sensitive=true,
 morekeywords={cd, source, ams_ics, eldo, ezwave, calibre, xwd, convert,
 grep, find, cat, touch, vim, nano, cp, rm, mkdir,
 DRC, LVS, PEX, history, exit, touch,},
 keywordstyle=\bfseries\color{bashcommand},
 morecomment=[l]{\#}, % Comentários começam com #
 morestring=[b]", % Strings em aspas duplas
 morestring=[b]', % Strings em aspas simples
}

% --- CONFIGURAÇÃO DO CABEÇALHO E RODAPÉ ---
\pagestyle{fancy}
\fancyhf{}
\lhead{\footnotesize \nouppercase{\leftmark}} % Mostra a Seção atual
\rhead{\footnotesize \nouppercase{\rightmark}} % Mostra a Subseção atual
\cfoot{\footnotesize \thepage}
\renewcommand{\headrulewidth}{0.4pt}

% --- AJUSTE PARA \leftmark E \rightmark FUNCIONAREM COM \section ---
\renewcommand{\sectionmark}[1]{\markboth{#1}{}}
\renewcommand{\subsectionmark}[1]{\markright{#1}}


% ####################################################################
% #           INFORMAÇÕES DA CAPA (DEFINIÇÕES GLOBAIS)               #
% ####################################################################
%
% Estas definições alimentam a \begin{titlepage} abaixo.

\title{Material de Consulta\\ \large SEL0621 - Projeto de Circuitos Integrados Digitais I}

\author{
    Felipi Adenildo Soares Sousa \\ felipiadenildo(at)usp.br
}

\newcommand{\professor}{Prof. Dr. João Navarro Soares Júnior \\ navarro(at)sc.usp.br}
\newcommand{\dataEntrega}{13 de novembro de 2025}

% ####################################################################
% #                        FIM DO PREÂMBULO                          #
% ####################################################################


\begin{document}

% ####################################################################
% #                     CAPA CUSTOMIZADA (Titlepage)                 #
% ####################################################################
%
% Esta é a estrutura da sua capa, usando as definições do preâmbulo.
% Você pode colocar isso no início do seu \begin{document}.

\begin{titlepage}
    \centering
    \makeatletter % Permite o uso de comandos com '@' (ex: \@title)

    % --- CABEÇALHO DA UNIVERSIDADE ---
    {\fontsize{16}{18}\selectfont\bfseries UNIVERSIDADE DE SÃO PAULO\par}
    {\fontsize{14}{16}\selectfont ESCOLA DE ENGENHARIA DE SÃO CARLOS\par}
    {\fontsize{12}{14}\selectfont Departamento de Engenharia Elétrica e de Computação\par}
    
    \vspace{2.5cm}
    
    % --- TÍTULO DO RELATÓRIO ---
    % Usa o \title definido no preâmbulo
    {\huge\bfseries \@title\par}
    
    \vfill % Adiciona espaço flexível
    

    % --- INFORMAÇÕES DO PROFESSOR ---
    \begin{minipage}{0.9\textwidth}
        \large
        \begin{flushleft}
            \textbf{Professor:} \\
            \professor
        \end{flushleft}
    \end{minipage}

    \vspace{1.5cm}

    % --- INFORMAÇÕES DOS ALUNOS ---
    \begin{minipage}{0.9\textwidth}
        \large
        \begin{flushleft}
            \textbf{Aluno:} \\
            \@author
        \end{flushleft}
    \end{minipage}
    
    \vfill % Adiciona mais espaço flexível
    
    % --- LOCAL E DATA ---
    % Usa o \dataEntrega definido no preâmbulo
    {\large São Carlos - SP\par}
    {\large \dataEntrega\par}
    
    \makeatother % Restaura o comportamento normal do '@'
\end{titlepage}


% ####################################################################
% #                          ÍNDICE (Sumário)                        #
% ####################################################################
\newpage
\tableofcontents
\thispagestyle{empty} % Remove cabeçalho e rodapé da pág. do índice

\newpage

% ####################################################################
% #                    INÍCIO DO CONTEÚDO DO DOCUMENTO               #
% ####################################################################


\newpage
\section{Projeto de Portas CMOS Estáticas}

\subsection{Da Lógica ao Esquemático}
Uma porta CMOS estática é composta por duas redes de transistores que são o dual uma da outra .

\begin{itemize}
    \item \textbf{Rede Pull-Up (PUN):}
    \begin{itemize}
        \item \textbf{Transistores:} \textbf{PMOS} (ficam "em cima").
        \item \textbf{Conecta:} A saída (Y) ao \textbf{VDD}.
    \end{itemize}
    \item \textbf{Rede Pull-Down (PDN):}
    \begin{itemize}
        \item \textbf{Transistores:} \textbf{NMOS} (ficam "em baixo").
        \item \textbf{Conecta:} A saída (Y) ao \textbf{VSS (GND)}.
    \end{itemize}
\end{itemize}

\subsubsection{ Regras de Dualidade (Teorema de De Morgan)}
A rede PDN implementa a lógica \textit{diretamente} (sem a negação final), enquanto a rede PUN implementa o DUAL. Esta dualidade é uma consequência direta do Teorema de De Morgan .

\begin{itemize}
    \item \textbf{Rede PUN (PMOS):}
    \begin{itemize}
        \item Operação \textbf{AND ($\cdot$)} $\rightarrow$ Transistores em \textbf{Paralelo}.
        \item Operação \textbf{OR (+)} $\rightarrow$ Transistores em \textbf{Série}.
    \end{itemize}
    \item \textbf{Rede PDN (NMOS):}
    \begin{itemize}
        \item Operação \textbf{AND ($\cdot$)} $\rightarrow$ Transistores em \textbf{Série}.
        \item Operação \textbf{OR (+)} $\rightarrow$ Transistores em \textbf{Paralelo}.
    \end{itemize}
\end{itemize}

\subsection{Exemplo Passo-a-Passo: OAI21 - $Y = \overline{(A+B) \cdot C}$}
Vamos projetar a porta "OR-AND-Invert" (OAI21), cuja lógica é $\neg(b(a+c))$ na Prova 2010.

\begin{enumerate}
    \item \textbf{Rede Pull-Up (PUN - PMOS):}
    \begin{itemize}
        \item A lógica é o DUAL da PDN: $(A \cdot B) + C$.
        \item (A $\parallel$ B) na PDN $\rightarrow$ Transistores PMOS A e B ficam em \textbf{SÉRIE}.
        \item em SÉRIE com C na PDN $\rightarrow$ O bloco (A $\cdot$ B) fica em \textbf{PARALELO} com o transistor PMOS C.
    \end{itemize}

    \item \textbf{Rede Pull-Down (PDN - NMOS):}
    \begin{itemize}
        \item A lógica (sem a inversão) é: $(A+B) \cdot C$.
        \item \textbf{$(A+B)$ (OR):} Transistores NMOS A e B ficam em \textbf{PARALELO}.
        \item \textbf{$\cdot C$ (AND):} O bloco (A $\parallel$ B) fica em \textbf{SÉRIE} com o transistor NMOS C.
    \end{itemize}

    
    \item \textbf{Circuito Completo:}
    \begin{itemize}
        \item As entradas (A, B, C) são conectadas aos gates dos pares NMOS/PMOS.
        \item A PUN é conectada ao VDD; a PDN é conectada ao VSS.
        \item A Saída (Y ou 'out') é retirada da junção entre a PUN e a PDN.
    \end{itemize}
\end{enumerate}

\newpage

\subsection{Princípio Fundamental do Dimensionamento}
O objetivo do dimensionamento (definir $W_n$ e $W_p$) é igualar as "forças" das redes pull-up e pull-down para que os tempos de subida e descida da porta sejam simétricos.

\subsubsection{ Equações de Tempo de Propagação}
Os tempos de subida e descida são dados aproximadamente por :
$$ t_{phl} = \frac{1.6 \cdot C_L}{\mu_n C_{ox} (W/L)_{n,eff} V_{DD}} $$
$$ t_{plh} = \frac{1.6 \cdot C_L}{\mu_p C_{ox} (W/L)_{p,eff} V_{DD}} $$
Para igualar os tempos ($t_{phl} = t_{plh}$), os termos $1.6 \cdot C_L / (C_{ox} V_{DD})$ cancelam, resultando em:
$$ \frac{1}{\mu_n (W/L)_{n,eff}} = \frac{1}{\mu_p (W/L)_{p,eff}} $$
Assumindo que $L_n = L_p$ (comprimentos iguais), a condição de simetria torna-se:
\begin{equation}
\mu_n \cdot W_{n,eff} = \mu_p \cdot W_{p,eff}
\end{equation}

\subsubsection{ Razão de Mobilidade ($r$) e Modelos}
Definimos a razão de mobilidade $r = \mu_n / \mu_p$. No entanto, este valor \textbf{não é constante}; ele depende do modelo de simulação (típico, pior velocidade, etc.) .

\begin{itemize}
    \item $\mu_n$: Mobilidade dos elétrons (NMOS).
    \item $\mu_p$: Mobilidade das lacunas (PMOS).
\end{itemize}
Como $\mu_n > \mu_p$, os transistores PMOS precisam ser fisicamente \textbf{mais largos ($W_p$)} que os NMOS ($W_n$) para ter a mesma "força".

\begin{table}[H]
\centering
\caption{Valores de Mobilidade (U0) e Razão (r) para Diferentes Modelos}
\begin{tabular}{@{}lccc@{}}
\toprule
\textbf{Modelo} & \textbf{U0 NMOS ($\mu_n$)} & \textbf{U0 PMOS ($\mu_p$)} & \textbf{Razão $r = \mu_n / \mu_p$} \\
\midrule
Constantes Gerais & 370  & 126  & $\approx 2.94$ \\
Modelo Típico (tm) & 475.8  & 148.2  & $\approx 3.21$  \\
Worst Power (wp) & 500.2  & 158.1  & $\approx 3.16$  \\
Worst Speed (ws) & 467.1  & 131.4  & $\approx 3.55$  \\
\bottomrule
\end{tabular}
\end{table}

\textbf{Relação de Largura para Força Equivalente:}
\begin{equation}
W_{p,eff} = r \cdot W_{n,eff}
\end{equation}

\subsection{Análise de Tempos (Melhor e Pior Caso)}
A largura efetiva ($W_{eff}$) depende de quais transistores estão conduzindo.

\begin{itemize}
    \item $t_{PHL}$ (Tempo High-to-Low): Tempo de \textbf{descida}. Controlado pela \textbf{PDN (NMOS)}.
    \item $t_{PLH}$ (Tempo Low-to-High): Tempo de \textbf{subida}. Controlado pela \textbf{PUN (PMOS)}.
\end{itemize}

\begin{itemize}
    \item \textbf{Pior Caso (Worst Case):} Ocorre no caminho de \textbf{maior resistência} (mais transistores em série e menos em paralelo) . Resulta no tempo \textbf{mais longo}.
    \item \textbf{Melhor Caso (Best Case):} Ocorre no caminho de \textbf{menor resistência} (todos transistores em paralelo ligados). Resulta no tempo \textbf{mais curto}.
\end{itemize}

\subsubsection{ Cálculo da Largura Efetiva ($W_{eff}$)}
Assumindo transistores idênticos ($W, L$) :
\begin{itemize}
    \item \textbf{N Transistores em Série:} A resistência total aumenta. A largura efetiva é $N$ vezes menor.
    $$ W_{eff} = \frac{W}{N} $$
    \item \textbf{N Transistores em Paralelo:} A resistência total diminui. A largura efetiva é $N$ vezes maior.
    $$ W_{eff} = N \cdot W $$
\end{itemize}

\newpage

\subsection{Dimensionamento da Porta OAI21 - $Y = \overline{(A+B) \cdot C}$}
Usamos as redes que derivamos acima e assumimos $W_{nA} = W_{nB} = W_{nC} = W_n$ e $W_{pA} = W_{pB} = W_{pC} = W_p$.

\begin{table}[H]
\centering
\caption{Cálculo da Largura Efetiva ($W_{eff}$) para a porta OAI21 }
\begin{tabular}{@{}lll@{}}
\toprule
\textbf{Caso} & \textbf{Rede e Caminho Ativo} & \textbf{$W_{eff}$} \\
\midrule
\textbf{Pior Subida} ($t_{PLH,worst}$) & PUN: Caminho por A e B em série. & $W_{p,eff} = W_p / 2$ \\
\textbf{Melhor Subida} ($t_{PLH,best}$) & PUN: Caminho por C (em paralelo). & $W_{p,eff} = W_p$ \\
\midrule
\textbf{Pior Descida} ($t_{PHL,worst}$) & PDN: Caminho por A e C (ou B e C). & $W_{n,eff} = W_n / 2$ \\
\textbf{Melhor Descida} ($t_{PHL,best}$) & PDN: Caminho por (A $\parallel$ B) e C. & $W_{n,eff} = (2W_n) / 3$ \\
\bottomrule
\end{tabular}
\end{table}

\subsection{Relações de Comparação para OAI21}
Usando $W_{p,eff} = r \cdot W_{n,eff}$ (com $r \approx 2.94$ a $3.55$, dependendo do modelo):

\begin{itemize}
    \item \textbf{$t_{PLH,worst} = t_{PHL,worst}$} (Pior subida = Pior descida) 
    \begin{itemize}
        \item $W_{p,eff} = r \cdot W_{n,eff} \implies \dfrac{W_p}{2} = r \cdot \left(\dfrac{W_n}{2}\right)$
        \item $\mathbf{W_p = r \cdot W_n}$
    \end{itemize}
    
    \item \textbf{$t_{PLH,best} = t_{PHL,best}$} (Melhor subida = Melhor descida)
    \begin{itemize}
        \item $W_{p,eff} = r \cdot W_{n,eff} \implies W_p = r \cdot \left(\dfrac{2W_n}{3}\right)$
        \item $\mathbf{W_p = \dfrac{2r}{3} W_n}$
    \end{itemize}
    
    \item \textbf{$t_{PLH,worst} = t_{PHL,best}$} (Pior subida = Melhor descida) 
    \begin{itemize}
        \item $W_{p,eff} = r \cdot W_{n,eff} \implies \dfrac{W_p}{2} = r \cdot \left(\dfrac{2W_n}{3}\right)$
        \item $\mathbf{W_p = \dfrac{4r}{3} W_n}$
    \end{itemize}
    
    \item \textbf{$t_{PLH,best} = t_{PHL,worst}$} (Melhor subida = Pior descida)
    \begin{itemize}
        \item $W_{p,eff} = r \cdot W_{n,eff} \implies W_p = r \cdot \left(\dfrac{W_n}{2}\right)$
        \item $\mathbf{W_p = \dfrac{r}{2} W_n}$
    \end{itemize}
\end{itemize}
\newpage
\section{Dimensionamento de Portas Lógicas Complexas}

O objetivo do dimensionamento é encontrar as larguras ($W_n, W_p$) para que os tempos de subida e descida sejam simétricos. Para isso, igualamos a "força" das redes (PUN e PDN) usando a razão de mobilidade ($r \approx 2.94$ a $3.55$, dependendo do modelo) .

A equação base é: $W_{p,eff} = r \cdot W_{n,eff}$. Assumimos $L_n = L_p$.

\subsection{Porta NAND3 - $Y = \overline{A \cdot B \cdot C}$}
\begin{itemize}
    \item \textbf{PDN (NMOS):} 3 transistores (A, B, C) em \textbf{Série}.
    \item \textbf{PUN (PMOS):} 3 transistores (A, B, C) em \textbf{Paralelo}.
\end{itemize}

\begin{table}[H]
\centering
\caption{Largura Efetiva ($W_{eff}$) para NAND3}
\begin{tabular}{@{}lll@{}}
\toprule
\textbf{Caso} & \textbf{Caminho Ativo} & \textbf{$W_{eff}$} \\
\midrule
Pior Subida ($t_{PLH,worst}$) & PUN: Apenas 1 PMOS ligado & $W_{p,eff} = W_p$ \\
Melhor Subida ($t_{PLH,best}$) & PUN: 3 PMOS ligados em paralelo & $W_{p,eff} = 3 W_p$ \\
\midrule
Pior Descida ($t_{PHL,worst}$) & PDN: 3 NMOS ligados em série & $W_{n,eff} = W_n / 3$ \\
Melhor Descida ($t_{PHL,best}$) & PDN: 3 NMOS ligados em série & $W_{n,eff} = W_n / 3$ \\
\bottomrule
\end{tabular}
\end{table}

\subsection{Porta NOR3 - $Y = \overline{A + B + C}$}
\begin{itemize}
    \item \textbf{PDN (NMOS):} 3 transistores (A, B, C) em \textbf{Paralelo}.
    \item \textbf{PUN (PMOS):} 3 transistores (A, B, C) em \textbf{Série}.
\end{itemize}

\begin{table}[H]
\centering
\caption{Largura Efetiva ($W_{eff}$) para NOR3}
\begin{tabular}{@{}lll@{}}
\toprule
\textbf{Caso} & \textbf{Caminho Ativo} & \textbf{$W_{eff}$} \\
\midrule
Pior Subida ($t_{PLH,worst}$) & PUN: 3 PMOS ligados em série & $W_{p,eff} = W_p / 3$ \\
Melhor Subida ($t_{PLH,best}$) & PUN: 3 PMOS ligados em série & $W_{p,eff} = W_p / 3$ \\
\midrule
Pior Descida ($t_{PHL,worst}$) & PDN: Apenas 1 NMOS ligado & $W_{n,eff} = W_n$ \\
Melhor Descida ($t_{PHL,best}$) & PDN: 3 NMOS ligados em paralelo & $W_{n,eff} = 3 W_n$ \\
\bottomrule
\end{tabular}
\end{table}

\subsection{Porta AOI21 - $Y = \overline{(A \cdot B) + C}$}
\begin{itemize}
    \item \textbf{PDN (NMOS):} (A e B em \textbf{Série}) em \textbf{Paralelo} com C.
    \item \textbf{PUN (PMOS):} (A e B em \textbf{Paralelo}) em \textbf{Série} com C.
\end{itemize}

\begin{table}[H]
\centering
\caption{Largura Efetiva ($W_{eff}$) para AOI21}
\begin{tabular}{@{}lll@{}}
\toprule
\textbf{Caso} & \textbf{Caminho Ativo} & \textbf{$W_{eff}$} \\
\midrule
Pior Subida ($t_{PLH,worst}$) & PUN: Caminho por A e C (ou B e C) & $W_{p,eff} = W_p / 2$ \\
Melhor Subida ($t_{PLH,best}$) & PUN: Caminho por (A$\parallel$B) e C & $W_{p,eff} = (2W_p) / 3$ \\
\midrule
Pior Descida ($t_{PHL,worst}$) & PDN: Caminho por A e B em série & $W_{n,eff} = W_n / 2$ \\
Melhor Descida ($t_{PHL,best}$) & PDN: Caminho por C & $W_{n,eff} = W_n$ \\
\bottomrule
\end{tabular}
\end{table}

\subsection{Porta OAI21 - $Y = \overline{(A + B) \cdot C}$}
\begin{itemize}
    \item \textbf{PDN (NMOS):} (A e B em \textbf{Paralelo}) em \textbf{Série} com C.
    \item \textbf{PUN (PMOS):} (A e B em \textbf{Série}) em \textbf{Paralelo} com C.
\end{itemize}

\begin{table}[H]
\centering
\caption{Largura Efetiva ($W_{eff}$) para OAI21}
\begin{tabular}{@{}lll@{}}
\toprule
\textbf{Caso} & \textbf{Caminho Ativo} & \textbf{$W_{eff}$} \\
\midrule
Pior Subida ($t_{PLH,worst}$) & PUN: Caminho por A e B em série & $W_{p,eff} = W_p / 2$ \\
Melhor Subida ($t_{PLH,best}$) & PUN: Caminho por C & $W_{p,eff} = W_p$ \\
\midrule
Pior Descida ($t_{PHL,worst}$) & PDN: Caminho por A e C (ou B e C) & $W_{n,eff} = W_n / 2$ \\
Melhor Descida ($t_{PHL,best}$) & PDN: Caminho por (A$\parallel$B) e C & $W_{n,eff} = (2W_n) / 3$ \\
\bottomrule
\end{tabular}
\end{table}


\begin{figure}[htb] % 'htb' é uma sugestão de posicionamento (here, top, bottom)
    \centering % Centraliza todo o conteúdo da figura
    
    % --- Primeira Linha ---
    
    \begin{subfigure}{0.48\textwidth}
        \centering
        \includegraphics[width=\linewidth]{nand3.png}
        \caption{Diagrama esquemático da Porta NAND3}
        \label{fig:nand3}
    \end{subfigure}
    \hfill % Adiciona espaço horizontal flexível entre as colunas
    \begin{subfigure}{0.48\textwidth}
        \centering
        \includegraphics[width=\linewidth]{nor3.png}
        \caption{Diagrama esquemático da NOR3}
        \label{fig:nor3}
    \end{subfigure}
    
    \vspace{0.5cm} % Adiciona um pequeno espaço vertical entre as linhas
    
    % --- Segunda Linha ---
    
    \begin{subfigure}{0.48\textwidth}
        \centering
        \includegraphics[width=\linewidth]{aoi21.jpeg}
        \caption{Diagrama esquemático da AOI21}
        \label{fig:aoi21}
    \end{subfigure}
    \hfill % Adiciona espaço horizontal flexível entre as colunas
    \begin{subfigure}{0.48\textwidth}
        \centering
        \includegraphics[width=\linewidth]{oai21.png}
        \caption{Diagrama esquemático da OAI21}
        \label{fig:oai21} % <-- Corrigi este label (estava duplicado)
    \end{subfigure}
    
    % \caption{Caption geral para todas as figuras (opcional)}
    % \label{fig:todos_diagramas}
\end{figure}

\newpage


\subsection[\textcolor{red}{Exemplos Práticos de Provas}]{\textcolor{red}{Exemplos Práticos de Provas}}
(Usando $r = 2.94$ para consistência com os cálculos originais)

\subsubsection[\textcolor{red}{Exemplo 1: Prova 2016b (OAI21)}]{\textcolor{red}{Exemplo 1: Prova 2016b (OAI21)}}
\begin{itemize}
    \item \textbf{Lógica:} $D = \neg((A+B) \cdot C)$ (Porta OAI21).
    \item \textbf{Dado:} $W_n = 3 \mu m$ (para todos os NMOS).
    \item \textbf{Condição:} $t_{PLH,worst} = t_{PHL,worst}$ (pior subida = pior descida).
\end{itemize}

\textbf{Cálculo:}
\begin{enumerate}
    \item \textbf{$W_{n,eff}$ (Pior Descida):} Da tabela OAI21, o pior caminho na PDN é por A-C (ou B-C).
    \[ W_{n,eff} = \frac{W_n}{2} = \frac{3 \mu m}{2} = 1.5 \mu m \]
    
    \item \textbf{$W_{p,eff}$ (Pior Subida):} Da tabela OAI21, o pior caminho na PUN é por A-B.
    \[ W_{p,eff} = \frac{W_p}{2} \]

    \item \textbf{Equacionar:} $W_{p,eff} = r \cdot W_{n,eff}$
    \[ \frac{W_p}{2} = r \cdot \left( \frac{W_n}{2} \right) \]
    \[ W_p = r \cdot W_n \]
    \[ W_p = 2.94 \cdot (3 \mu m) = \mathbf{8.82 \mu m} \]
\end{enumerate}

\newpage

\subsubsection[\textcolor{red}{Exemplo 2: Prova 2022a (AOI21)}]{\textcolor{red}{Exemplo 2: Prova 2022a (AOI21)}}
\begin{itemize}
    \item \textbf{Lógica:} $\neg(a \cdot b + c)$ (Porta AOI21).
    \item \textbf{Dado:} $W_n = 4 \mu m$ (para todos os NMOS).
    \item \textbf{Condição:} $t_{PLH,worst} = t_{PHL,best}$ (pior subida = melhor descida).
\end{itemize}

\textbf{Cálculo:}
\begin{enumerate}
    \item \textbf{$W_{n,eff}$ (Melhor Descida):} Da tabela AOI21, o melhor caminho na PDN é pelo transistor C.
    \[ W_{n,eff} = W_n = 4 \mu m \]
    
    \item \textbf{$W_{p,eff}$ (Pior Subida):} Da tabela AOI21, o pior caminho na PUN é por A-C (ou B-C).
    \[ W_{p,eff} = \frac{W_p}{2} \]

    \item \textbf{Equacionar:} $W_{p,eff} = r \cdot W_{n,eff}$
    \[ \frac{W_p}{2} = r \cdot W_n \]
    \[ W_p = 2 \cdot r \cdot W_n \]
    \[ W_p = 2 \cdot 2.94 \cdot (4 \mu m) = \mathbf{23.52 \mu m} \]
\end{enumerate}

\newpage

\subsubsection[\textcolor{red}{Exemplo 3: Prova 2024 (OAI21)}]{\textcolor{red}{Exemplo 3: Prova 2024 (OAI21)}}
(Nota: A prova pede a função $a(b+c)$. Uma porta CMOS estática é inerentemente inversora, então assumimos a implementação da função OAI21 $Y = \overline{A \cdot (B+C)}$ (que é topologicamente idêntica a $\overline{C \cdot (A+B)}$) e ignoramos a inversão final).
\begin{itemize}
    \item \textbf{Lógica (Assumida):} $Y = \overline{A \cdot (B+C)}$ (Porta OAI21).
    \item \textbf{Dado:} $W_p = 15 \mu m$ (para todos os PMOS).
    \item \textbf{Condição:} $t_{PLH,worst} = t_{PHL,best}$ (pior subida = melhor descida).
\end{itemize}

\textbf{Cálculo:}
\begin{enumerate}
    \item \textbf{$W_{p,eff}$ (Pior Subida):} Da tabela OAI21 (assumindo $Y=\overline{C(A+B)}$), o pior caminho na PUN é A e B em série.
    \[ W_{p,eff} = \frac{W_p}{2} = \frac{15 \mu m}{2} = 7.5 \mu m \]

    \item \textbf{$W_{n,eff}$ (Melhor Descida):} Da tabela OAI21, o melhor caminho na PDN é (A$\parallel$B) e C em série.
    \[ W_{n,eff} = \frac{1}{\frac{1}{W_n + W_n} + \frac{1}{W_n}} = \frac{1}{\frac{1}{2W_n} + \frac{1}{W_n}} = \frac{2W_n}{3} \]

    \item \textbf{Equacionar:} $W_{p,eff} = r \cdot W_{n,eff}$
    \[ 7.5 \mu m = r \cdot \left( \frac{2W_n}{3} \right) \]
    \[ W_n = \frac{3 \cdot 7.5 \mu m}{2 \cdot r} \]
    \[ W_n = \frac{22.5 \mu m}{2 \cdot 2.94} = \frac{22.5 \mu m}{5.88} = \mathbf{3.83 \mu m} \]
\end{enumerate}

\newpage

\subsection[\textcolor{red}{Exemplos Complementares (Topologias Básicas)}]{\textcolor{red}{Exemplos Complementares (Topologias Básicas)}}
Estas questões cobrem topologias fundamentais (NAND e NOR) não presentes explicitamente nos enunciados das provas acima, mas essenciais para o entendimento.

\subsubsection[\textcolor{red}{Exemplo Extra A: Porta NAND3}]{\textcolor{red}{Exemplo Extra A: Porta NAND3}}
\begin{itemize}
    \item \textbf{Lógica:} $Y = \overline{A \cdot B \cdot C}$ (NAND de 3 entradas).
    \item \textbf{Dado:} $W_n = 6 \mu m$ (para todos os NMOS).
    \item \textbf{Condição:} $t_{PLH,worst} = t_{PHL,worst}$ (pior subida = pior descida).
\end{itemize}

\textbf{Cálculo:}
\begin{enumerate}
    \item \textbf{$W_{n,eff}$ (Pior Descida):} A rede PDN da NAND3 possui 3 transistores em série (A-B-C).
    \[ W_{n,eff} = \frac{W_n}{3} = \frac{6 \mu m}{3} = 2.0 \mu m \]

    \item \textbf{$W_{p,eff}$ (Pior Subida):} A rede PUN possui 3 transistores em paralelo. O pior caso ocorre quando apenas 1 transistor conduz (maior resistência).
    \[ W_{p,eff} = W_p \]

    \item \textbf{Equacionar:} $W_{p,eff} = r \cdot W_{n,eff}$
    \[ W_p = r \cdot (2.0 \mu m) \]
    \[ W_p = 2.94 \cdot 2.0 \mu m = \mathbf{5.88 \mu m} \]
\end{enumerate}

\subsubsection[\textcolor{red}{Exemplo Extra B: Porta NOR3}]{\textcolor{red}{Exemplo Extra B: Porta NOR3}}
\begin{itemize}
    \item \textbf{Lógica:} $Y = \overline{A + B + C}$ (NOR de 3 entradas).
    \item \textbf{Dado:} $W_n = 2 \mu m$ (para todos os NMOS).
    \item \textbf{Condição:} $t_{PLH,worst} = t_{PHL,best}$ (pior subida = melhor descida).
\end{itemize}

\textbf{Cálculo:}
\begin{enumerate}
    \item \textbf{$W_{n,eff}$ (Melhor Descida):} A rede PDN da NOR3 possui 3 transistores em paralelo. O melhor caso ocorre quando todos os 3 conduzem.
    \[ W_{n,eff} = 3 \cdot W_n = 3 \cdot 2 \mu m = 6 \mu m \]

    \item \textbf{$W_{p,eff}$ (Pior Subida):} A rede PUN da NOR3 possui 3 transistores em série.
    \[ W_{p,eff} = \frac{W_p}{3} \]

    \item \textbf{Equacionar:} $W_{p,eff} = r \cdot W_{n,eff}$
    \[ \frac{W_p}{3} = r \cdot (6 \mu m) \]
    \[ W_p = 3 \cdot 2.94 \cdot 6 \mu m = \mathbf{52.92 \mu m} \]
    (Nota: O valor elevado de $W_p$ deve-se à condição extrema de comparar 3 PMOS em série com 3 NMOS em paralelo).
\end{enumerate}

\newpage

\section{Schematic (Esquemático)}
Esta seção detalha o processo de criação, verificação e exportação de esquemáticos usando o Design Architect (iniciado a partir do ICStudio).

\subsection{Inicialização do Ambiente}

\subsubsection{Comandos no Terminal}
Primeiro, configure o ambiente Mentor Graphics e abra o projeto .
\begin{lstlisting}[language=bash, caption={Inicialização do shell e abertura do projeto}]
cd /local/users/cad/
source .cshrc-mentor
cd /local/users/cad/workavdl

# Para criar um NOVO projeto
ams_ics -project nome_projeto -t c35b4c3

# Para abrir um projeto EXISTENTE
ams_ics -p nome_projeto
\end{lstlisting}

\subsubsection{No ICStudio (Janela Principal)}
\begin{enumerate}
    \item \textbf{Criar Biblioteca:} No ICStudio, clique em \textbf{File $\rightarrow$ New $\rightarrow$ Library}. Use um nome em letras minúsculas (ex: \texttt{minha\_lib}).
    \item \textbf{Criar Célula:} Selecione a biblioteca recém-criada, clique com o \textbf{botão direito $\rightarrow$ New View}.
    \item \textbf{Configurar Célula:}
    \begin{itemize}
        \item \textbf{Cell Name}: \texttt{nome\_da\_celula} (ex: \texttt{inv}, \texttt{nand3}) 
        \item \textbf{View Type}: Selecione \textbf{Schematic} 
    \end{itemize}
    \item Isso abrirá a ferramenta de esquemático \textbf{Design Architect}.
\end{enumerate}

\subsection{Construção do Esquemático}

\subsubsection{Elementos Básicos (Paleta de Componentes)}
Os componentes são adicionados usando a paleta lateral (aberta no \textit{"penúltimo ícone à esquerda"}  ou com o \textit{"botão do meio"} do mouse ) ou o menu \textbf{Add $\rightarrow$ Instance}.

\begin{itemize}
    \item \textbf{Transistores:}
    \begin{itemize}
        \item \textbf{Biblioteca:} \texttt{HIT-KIT Utilities $\rightarrow$ Devices $\rightarrow$ MOS} 
        \item \textbf{Células:} \textbf{NMOS4} e \textbf{PMOS4} 
        \item \textbf{Configuração:} Após adicionar, selecione o transistor e em \texttt{Properties} (ou na janela pop-up ) configure:
        \item \texttt{Wtot = largura\_em\_microns} (ex: 5u) 
        \item \texttt{L = 0.35} (valor fixo da tecnologia) 
    \end{itemize}
    
    \item \textbf{Fontes de Alimentação:}
    \begin{itemize}
        \item \textbf{Biblioteca:} \texttt{MGC Library $\rightarrow$ Generic Lib} 
        \item \textbf{Células:} \textbf{VDD} e \textbf{VSS} 
    \end{itemize}
    
    \item \textbf{Portas de Entrada/Saída:}
    \begin{itemize}
        \item \textbf{Biblioteca:} \texttt{MGC Library $\rightarrow$ Generic Lib} 
        \item \textbf{Células:} \textbf{Portin} (Entrada) e \textbf{Portout} (Saída) 
    \end{itemize}

    \item \textbf{Resistores (quando necessário):} 
    \begin{itemize}
        \item \textbf{Biblioteca:} \texttt{HIT-KIT Utilities $\rightarrow$ Devices $\rightarrow$ Resistors} 
        \item \textbf{Célula:} \textbf{rpolyh} (Polisilício de alta resistividade) 
    \end{itemize}
\end{itemize}

\subsubsection{Procedimento de Conexão}
\begin{enumerate}
    \item \textbf{Posicionar Componentes:} Adicione todos os transistores, portas e fontes necessários.
    \item \textbf{Conectar com Wires:} Use a ferramenta de "wire" (linha) para conectar os terminais.
    \item \textbf{Nomear Nets (Fios):}
    \begin{itemize}
        \item Selecione um wire $\rightarrow$ \textbf{Botão direito $\rightarrow$ Name Nets} .
        \item Dê nomes claros (ex: \texttt{A}, \texttt{B}, \texttt{OUT}, \texttt{n1}).
    \end{itemize}
    \item \textbf{Verificar Conexões de Bulk (Substrato):} Esta é uma etapa \textbf{crítica}.
    \begin{itemize}
        \item O "bulk" de todos os \textbf{PMOS4} deve ser conectado ao \textbf{VDD}.
        \item O "bulk" de todos os \textbf{NMOS4} deve ser conectado ao \textbf{VSS}.
    \end{itemize}
\end{enumerate}

\subsubsection{Trabalhando com Hierarquia (Células Existentes)}
Você pode instanciar outras células, incluindo bibliotecas padrão ou as que você mesmo criou.

\begin{itemize}
    \item \textbf{Adicionando Células da CORELIB:} 
    \begin{itemize}
        \item \textbf{Add $\rightarrow$ Instance $\rightarrow$ Choose Symbol}.
        \item \textbf{Biblioteca}: \texttt{CORELIB}.
        \item \textbf{Células comuns}: \texttt{inv0} (inversor), \texttt{df1} (flip-flop D), \texttt{dl1} (latch D) .
    \end{itemize}
    
    \item \textbf{Adicionando Seus Próprios Símbolos:}
    \begin{itemize}
        \item \textbf{Add $\rightarrow$ Instance $\rightarrow$ Choose Symbol}.
        \item Selecione o símbolo da célula que você criou e verificou anteriormente.
    \end{itemize}
\end{itemize}

\subsection{Verificações do Esquemático e Geração do Símbolo}
Siga esta sequência para garantir que o esquemático e seu símbolo estejam corretos.

\begin{enumerate}
    \item \textbf{Primeira Verificação (Check Schematic):}
    \begin{itemize}
        \item \textbf{Comando:} \texttt{File $\rightarrow$ Check Schematic}.
        \item \textbf{Resultado Esperado:} 0 Erros, 1 Warning .
        \item \textbf{Warning Esperado:} \textit{"Warning: Interface ... has no pins"}.
    \end{itemize}

    \item \textbf{Geração do Símbolo:}
    \begin{itemize}
        \item \textbf{Comando:} \texttt{Miscellaneous $\rightarrow$ Generate Symbol} .
        \item Marque \textbf{Replace Existing}.
        \item \textbf{Choose Shape} $\rightarrow$ Selecione o formato gráfico adequado (ex: \texttt{And Gate}, \texttt{Buffer}, ou \texttt{Box}) .
        \item O editor de símbolos abrirá.
    \end{itemize}
    
    \item \textbf{ Sequência de Verificação do Símbolo:}
    \begin{itemize}
        \item No editor de símbolo, execute \textbf{File $\rightarrow$ Check Symbol}.
        \item Salve o símbolo: \textbf{File $\rightarrow$ Save Symbol}.
        \item Verifique novamente: \textbf{File $\rightarrow$ Check Symbol} (agora deve dar 0 warnings) .
        \item Feche o editor de símbolo.
    \end{itemize}

    \item \textbf{Verificação Final (Check Schematic):}
    \begin{itemize}
        \item De volta ao esquemático, execute: \texttt{File $\rightarrow$ Check Schematic} (novamente) .
        \item \textbf{Resultado Esperado:} 0 Erros, 0 Warnings .
    \end{itemize}
\end{enumerate}

\subsection{ Criação do ViewPoint e Netlist}
O ViewPoint é uma "visão" do esquemático usada para gerar o netlist de simulação (arquivo `.spi` ou `.cir`).

\subsubsection{Criação do ViewPoint}
\begin{enumerate}
    \item \textbf{Comando:} \texttt{Hit-Kit Utilities $\rightarrow$ Create ViewPoint} .
    \item \textbf{Design Path:} Verifique se o caminho aponta para o seu esquemático \\ (ex: \texttt{\$Cell/default.group/logic.view/gate}).
    \item \textbf{Tipo:} Selecione \texttt{device}.
\end{enumerate}

\subsubsection{Geração do Netlist (ELDO)}
\begin{enumerate}
    \item \textbf{Modo de Simulação:} Entre no modo de simulação (último botão "play" verde na barra esquerda).
    \item \textbf{Selecionar Viewpoint:} Selecione o \texttt{vpt\_c35b4\_device} que você acabou de criar.
    \item \textbf{Gerar Netlist:} Clique no botão "Netlist" (ícone com wires pretos).
    \item \textbf{Visualizar:} Use o comando \texttt{ASCII Results $\rightarrow$ View Netlist} (ícone do "olho") para ver o netlist gerado.
    \item \textbf{Copiar e Editar:} Copie o conteúdo para um arquivo \texttt{.cir}. Você \textbf{precisa} editar o arquivo:
    \begin{itemize}
        \item Mude \texttt{.end} para \texttt{.ends}.
        \item Adicione os nomes dos ports (entradas/saídas) na linha \texttt{.subckt} (ex: \texttt{.subckt GATE A B C OUT VDD VSS}).
    \end{itemize}
\end{enumerate}

\subsubsection{Estrutura do Netlist Gerado}
O arquivo gerado (ex: \texttt{nome\_porta\_vpt\_c35b4\_device.spi}) terá esta aparência:
\begin{lstlisting}[language=pspice]
* Arquivo gerado automaticamente pelo ViewPoint
.global VSS VDD
.subckt NOME_PORTA A B C OUT
* ...
* Definicoes dos transistores M1, M2...
* ...
.ends
\end{lstlisting}

\subsection{Observações e Dicas Críticas}

\subsubsection{Erros Comuns no Esquemático}
\begin{itemize}
    \item \textbf{Bulk não conectado:} Erro mais comum. Causa falha no LVS. Sempre conecte PMOS no VDD e NMOS no VSS.
    \item \textbf{Warning persistente (mesmo após criar símbolo):} Delete o símbolo e gere-o novamente.
    \item \textbf{Dois ports com mesmo nome:} Se as entradas A e B estão em curto, use o mesmo \texttt{Portin} e conecte-o a ambos os gates. Não crie dois \texttt{Portin} com o nome \texttt{A}.
\end{itemize}

\subsubsection{Dicas Importantes}
\begin{itemize}
    \item \textbf{Salvar Frequentemente:} Use \texttt{File $\rightarrow$ Save} ou o ícone do disquete.
    \item \textbf{Nomenclatura:} Use \textbf{apenas letras minúsculas} para nomes de bibliotecas e células.
    \item \textbf{Organização:} Mantenha o esquemático limpo e organizado. Isso facilita muito a depuração e a criação do layout.
\end{itemize}

\subsubsection{Para Circuitos Complexos (AOI/OAI/Sequenciais)}
\begin{itemize}
    \item A metodologia é a mesma.
    \item Siga as regras de dualidade (Série/Paralelo) para as redes PUN/PDN.
    \item Para circuitos sequenciais (Flip-Flops), adicione portas de entrada para \texttt{CLK} (clock) e \texttt{RST} (reset), se necessário.
    \item Use hierarquia: crie e verifique blocos menores (ex: um Latch) e depois use o símbolo desse bloco para construir um circuito maior (ex: um Flip-Flop Master-Slave).
\end{itemize}

\newpage

\newpage
\section[\textcolor{red}{Exemplos de Prova: Esquemático Avançado}]{\textcolor{red}{Exemplos de Prova: Esquemático Avançado}}

\subsection[\textcolor{red}{Contadores Síncronos (Provas 2016b Q5 e 2022a Q5)}]{\textcolor{red}{Contadores Síncronos (Provas 2016b Q5 e 2022a Q5)}}
Ambas as provas solicitam a transposição de um diagrama de blocos (contador) para o esquemático no Mentor Graphics, utilizando células da biblioteca padrão (\texttt{CORELIB}).

\textbf{Células Necessárias:}
\begin{itemize}
    \item \textbf{Prova 2016b (Fig. 2):} \texttt{df1} (Flip-Flop), \texttt{nor21}, \texttt{aoi211}, \texttt{aoi21}, \texttt{nand21} (verifique a lógica exata na figura da prova).
    \item \textbf{Prova 2022a (Fig. 1):} \texttt{df1}, \texttt{xnr21} (XNOR), \texttt{nand21}.
\end{itemize}

\textbf{Passo a Passo de Resolução:}
\begin{enumerate}
    \item \textbf{Criar Nova Célula:} \texttt{File $\rightarrow$ New $\rightarrow$ Cell} (ex: \texttt{contador\_p22}). View: \texttt{Schematic}.
    \item \textbf{Instanciar Células da CORELIB:}
    \begin{itemize}
        \item \texttt{Add $\rightarrow$ Instance $\rightarrow$ Choose Symbol}.
        \item Navegue até a biblioteca \texttt{CORELIB} e selecione as células identificadas acima (ex: \texttt{df1}).
    \end{itemize}
    \item \textbf{Conectar o Clock (CLK):}
    \begin{itemize}
        \item Em circuitos síncronos, o sinal de \texttt{CLK} deve ser conectado à entrada de clock de \textbf{todos} os Flip-Flops.
        \item Adicione um \texttt{Portin} chamado \texttt{CLK}.
    \end{itemize}
    \item \textbf{Realimentação (Feedback):}
    \begin{itemize}
        \item Conecte as saídas $Q$ (ou $\bar{Q}$) dos Flip-Flops de volta às entradas das portas lógicas combinacionais conforme o desenho da prova.
        \item \textit{Dica:} Use \texttt{Name Nets} para nomear fios distantes com o mesmo nome (ex: \texttt{Q1}) em vez de passar fios longos por todo o esquemático.
    \end{itemize}
    \item \textbf{Verificação e Símbolo:}
    \begin{itemize}
        \item \texttt{Check Schematic} (Deve dar 0 Erros).
        \item \texttt{Miscellaneous $\rightarrow$ Generate Symbol} (Substitua o existente).
    \end{itemize}
\end{enumerate}

\subsection[\textcolor{red}{Hierarquia e Blocos Reutilizáveis (Prova 2024 Q5 e Q6)}]{\textcolor{red}{Hierarquia e Blocos Reutilizáveis (Prova 2024 Q5 e Q6)}}
A prova de 2024 exige uma abordagem hierárquica: primeiro cria-se um bloco menor ("BLOCO") e depois ele é usado múltiplas vezes no circuito principal.

\subsubsection[\textcolor{red}{Passo 1: Criar o Sub-bloco (Questão 5)}]{\textcolor{red}{Passo 1: Criar o Sub-bloco (Questão 5)}}
\begin{itemize}
    \item \textbf{Objetivo:} Criar o esquemático interno do bloco mostrado na prova.
    \item \textbf{Componentes:} \texttt{df1} (Flip-Flop), \texttt{nand21} e \texttt{inv1} (para formar AND e OR, já que AND = NAND + INV).
    \item \textbf{Ação:}
    \begin{enumerate}
        \item Crie a célula \texttt{bloco\_q5}.
        \item Monte o circuito interno conforme a figura.
        \item Adicione portas de entrada/saída (ex: \texttt{In}, \texttt{Out}, \texttt{M}, \texttt{CLK}).
        \item \texttt{Check Schematic} e \textbf{Generate Symbol}.
    \end{enumerate}
\end{itemize}

\subsubsection[\textcolor{red}{Passo 2: Circuito Principal (Questão 6)}]{\textcolor{red}{Passo 2: Circuito Principal (Questão 6)}}
\begin{itemize}
    \item \textbf{Objetivo:} Construir o circuito final usando o símbolo criado no Passo 1.
    \item \textbf{Ação:}
    \begin{enumerate}
        \item Crie uma nova célula (ex: \texttt{circuito\_final\_q6}).
        \item \textbf{Instanciar o Bloco:} Use \texttt{Add Instance} e selecione o símbolo \texttt{bloco\_q5} que você acabou de criar. Repita 4 vezes.
        \item \textbf{Configurar Modos ($M_x$):}
        \begin{itemize}
            \item A prova pede conexões específicas como $M_1 = V_{DD}$ e $M_4 = 0V$.
            \item Adicione fontes \texttt{VDD} e \texttt{VSS} da biblioteca \texttt{Generic Lib}.
            \item Conecte o pino $M$ da primeira instância ao \texttt{VDD}.
            \item Conecte o pino $M$ da última instância ao \texttt{VSS} (GND).
        \end{itemize}
        \item \textbf{Interconexões:} Ligue a saída $X_n$ de um bloco à entrada do próximo e adicione a lógica de controle superior (\texttt{nand21}, \texttt{inv1}) conforme a figura.
    \end{enumerate}
\end{itemize}

\newpage
\section{Layout}
Esta seção descreve o fluxo de trabalho para criar o layout físico de uma célula no ICStation, a partir de um esquemático verificado.

\subsection{Criação do Layout a partir do Esquemático}

\subsubsection{Inicialização do Layout}
\begin{enumerate}
    \item No \textbf{ICStudio}: Selecione a célula desejada (ex: \texttt{inv}) na sua biblioteca.
    \item Clique com o \textbf{Botão direito $\rightarrow$ New View}.
    \item \textbf{View Type}: Selecione \textbf{Layout}.
    \item \textbf{Connectivity Source}: Assegure-se de que está selecionado \textbf{Schematic} (ou o \texttt{vpt\_c35b4\_device} se estiver usando ViewPoint).
    \item \textbf{Finish}. A ferramenta \textbf{ICStation} (para layout) será aberta.
\end{enumerate}

\subsubsection{Configuração do Ambiente ICStation}
Configure o ambiente para facilitar o trabalho:
\begin{itemize}
    \item \textbf{ Reserva de Célula:} Garanta que você pode editar o layout.
    \begin{itemize}
        \item \textbf{File $\rightarrow$ Cell $\rightarrow$ Reserve $\rightarrow$ Current Context}.
        \item O status deve mudar de (GE-R-0) (Read-only) para (GE-E-0) (Editável) .
    \end{itemize}
    \item \textbf{Janelas Lado a Lado:} \textbf{MGC $\rightarrow$ Setup $\rightarrow$ LeftRight} (para ver o esquemático e o layout simultaneamente).
    \item \textbf{Habilitar Atalhos:} \textbf{Other $\rightarrow$ Hotkeys $\rightarrow$ Enable}, depois \textbf{Other $\rightarrow$ Hotkeys $\rightarrow$ Load} .
    \item \textbf{Configurar o Grid:} \textbf{Other $\rightarrow$ Window $\rightarrow$ Set Grid} .
    \begin{itemize}
        \item \textbf{X = 0.025}, \textbf{Y = 0.025}
        \item \textbf{Minor = 0.1}, \textbf{Major = 1}
    \end{itemize}
    \item \textbf{Mostrar Softkeys:} \textbf{Setup $\rightarrow$ Session $\rightarrow$ Show Softkeys} (mostra botões de atalho).
    \item \textbf{Mostrar Paleta de Camadas:} \textbf{Other $\rightarrow$ Layers $\rightarrow$ Show layer palette $\rightarrow$ Append $\rightarrow$ all} .
    \item \textbf{ Dica de Estabilidade:} Ao usar a barra de scroll no ICStation, não clique repetidamente, pois isso pode travar a ferramenta.
\end{itemize}

\subsection{Geração Automática e Placement}

\subsubsection{AutoInst (Instanciação Automática)}
\begin{itemize}
    \item \textbf{Comando:} \textbf{DLA Layout $\rightarrow$ AutoInst}.
    \item \textbf{Ação:} Isso gera o layout automático (caixas de transistores e resistores) a partir do esquemático.
    \item \textbf{Visualizar Hierarquia (Peek):} Para ver o conteúdo das células (como os transistores), use \textbf{Context $\rightarrow$ Hierarchy $\rightarrow$ Peek $\rightarrow$ 2-4 levels}.
\end{itemize}

\subsubsection{Placement Automático (para Standard Cells)}
Para projetos que usam células-padrão (ex: da CORELIB), o processo é diferente:
\begin{itemize}
    \item \textbf{Comando:} \textbf{Place \& Route $\rightarrow$ Autofp $\rightarrow$ Ok}.
    \item \textbf{ Aspect Ratio:} Configure \textbf{Upper = 2} em \textbf{Aspect Ratio}.
    \item \textbf{StdCell} $\rightarrow$ Clique e arraste para desenhar a área onde as células serão posicionadas.
    \item Apague as linhas verdes externas (são guias para PADs que não usaremos agora).
    \item \textbf{ Dica de Conexão:} Antes de rotear, selecione todo o esquemático na janela ao lado para garantir que todas as conexões (overflows) apareçam no layout.
\end{itemize}

\subsection{Otimização do Layout (Layout Manual)}

\subsubsection{Merge (Fusão) de Transistores}
O objetivo é otimizar a área fundindo (sobrepondo) terminais de dreno/source.
\begin{itemize}
    \item \textbf{Transistores em Série:} Podem ser juntados diretamente (sobrepondo os terminais).
    \item \textbf{Transistores em Paralelo:} Use o comando \textbf{Flip} (atalho 'f') em um dos transistores antes de juntá-los para alinhar dreno com dreno e source com source.
    \item \textbf{Distância Crítica (DIFF-NTUB):} Mantenha uma distância de \textbf{1.2$\mu$m} entre a difusão (DIFF) dos NMOS e o poço (NTUB) dos PMOS.
    \item \textbf{ Dica de Merge:} O bulk (contato de poço) NUNCA pode estar entre MOS de tamanhos diferentes. Comece o merge por essa parte.
\end{itemize}

\subsubsection{Dobramento (Folding) de Transistores/Resistores}
Usado quando um componente é muito largo e atrapalha a otimização da área.
\begin{enumerate}
    \item \textbf{Dobrar Transistor (Fold):} 
    \begin{itemize}
        \item Desabilite os hotkeys (\textbf{Other $\rightarrow$ Hotkeys $\rightarrow$ Disable}).
        \item Selecione o transistor $\rightarrow$ digite \texttt{fold} no console $\rightarrow$ Enter.
        \item \textbf{Folds: 2} (ou mais, para dividir em múltiplos "dedos").
    \end{itemize}
    \item \textbf{Dobrar Resistor (Bend):} 
    \begin{itemize}
        \item \textbf{Object $\rightarrow$ Change $\rightarrow$ Device $\rightarrow$ Bend}.
    \end{itemize}
\end{enumerate}

\subsection{Adição de Poços e Contatos}

\subsubsection{Contatos de Poço (Bulk/Tap Contacts)}
Esses contatos são \textbf{essenciais} para conectar o substrato (bulk) ao VDD/VSS e evitar latch-up.
\begin{itemize}
    \item \textbf{Método 1 (Change Device):} \textbf{Objects $\rightarrow$ Change $\rightarrow$ Device} $\rightarrow$ Adicionar "t" (tap) onde necessário (ex: \texttt{tcgc} $\rightarrow$ \texttt{tcgct}).
    \item \textbf{Método 2 (Via Manual):} \textbf{DLA Device $\rightarrow$ Via $\rightarrow$ Point Via}.
    \begin{itemize}
        \item \textbf{pdm1}: Contato de bulk para \textbf{PMOS} (conecta NTUB ao VDD).
        \item \textbf{ndm1}: Contato de bulk para \textbf{NMOS} (conecta PTUB ao VSS).
        \item Selecione \textbf{MIN\_SIZE}.
        \item \textbf{} Após colocar o via, desenhe manualmente um \textbf{Shape} da camada (ex: \texttt{NTUB}) ao redor do contato para conectá-lo ao poço.
    \end{itemize}
    \item \textbf{Regra Crítica:} O contato de bulk ("t") não pode estar colado diretamente no gate ("g"). Se necessário, use um contato de difusão ("c") entre eles.
\end{itemize}

\subsubsection{Contatos Poly-Metal1 (p1m1)}
Usados para conectar o gate (Polisilício) ao roteamento de metal (Metal1).
\begin{itemize}
    \item \textbf{Comando:} \textbf{DLA Device $\rightarrow$ Via $\rightarrow$ Point Via $\rightarrow$ p1m1}.
    \item Selecione \textbf{MIN\_SIZE}.
\end{itemize}

\subsection{Roteamento (Routing)}
Processo de desenhar os "fios" (metais) que conectam os componentes.

\subsubsection{Configuração do Roteamento}
\begin{itemize}
    \item \textbf{Direção das Camadas:} \textbf{Route $\rightarrow$ Direction}.
    \begin{itemize}
        \item \textbf{POLY1}: Both (Permite rotear poly em X e Y)
        \item \textbf{MET1}: Both
        \item \textbf{MET2}: Both
        \item \textbf{MET3}: None (Desabilitar camadas superiores)
        \item \textbf{MET4}: None
    \end{itemize}
    \item \textbf{Opções Avançadas:} \textbf{Route $\rightarrow$ Options $\rightarrow$ Advanced}.
    \begin{itemize}
        \item Marcar \textbf{Check same net spacing}.
        \item Marcar \textbf{Center wires on pins}.
    \end{itemize}
    \item \textbf{Setup (Bloqueios):} \textbf{Route $\rightarrow$ Setup}.
    \begin{itemize}
        \item \textbf{Instance blockages}: Marcar \textbf{All data} (impede o roteador de passar por dentro dos transistores).
    \end{itemize}
\end{itemize}

\subsubsection{Roteamento Automático (ARoute)}
\begin{itemize}
    \item \textbf{Comando:} \textbf{ARoute Commands $\rightarrow$ RUN}.
    \item \textbf{Verificar Conexões Faltantes:} Olhe o console ou use o atalho \textbf{S0vrf}.
    \item \textbf{ Repetir/Apagar:} Use \textbf{Rip} (ripup) para refazer conexões falhas, ou \textbf{Routing Results $\rightarrow$ Delete Nets $\rightarrow$ all} para apagar tudo e recomeçar.
\end{itemize}

\subsubsection{Roteamento Manual (IRoute)}
Útil para conexões curtas ou complexas que o ARoute falha.
\begin{itemize}
    \item \textbf{Comando:} \textbf{IRoute Commands $\rightarrow$ RUN}.
    \item \textbf{Barra de espaço} alterna entre as camadas de roteamento (Poly, Metal1, etc.).
    \item \textbf{Conexão de Gate:} O padrão é \texttt{Gate(poly) -- poly -- (p1m1) -- metal(IN)}.
\end{itemize}

\subsubsection{Configuração de Net Classes (VDD/VSS)}
As linhas de alimentação devem ser mais largas para suportar mais corrente.
\begin{enumerate}
    \item \textbf{Comando:} \textbf{ARoute Net Classes $\rightarrow$ Edit $\rightarrow$ New}.
    \item \textbf{Configurar Larguras:} \textbf{MET1: 1.8}, \textbf{MET2: 1.8} $\rightarrow$ \textbf{OK} (Larguras de 1.0 $\mu$m a 1.8 $\mu$m são comuns).
    \item Dê um nome para a classe (ex: "power").
    \item \textbf{Assign} $\rightarrow$ Selecione o net \textbf{VSS} $\rightarrow$ \textbf{Apply} .
    \item Repita o processo para \textbf{VDD}.
\end{enumerate}

\subsection{Ports e Textos}
Ports são necessários para que as ferramentas de verificação (LVS) e extração (PEX) saibam onde estão as entradas e saídas do seu layout.

\subsubsection{Adição de Ports}
\begin{itemize}
    \item \textbf{Comando:} \textbf{DLA Layout $\rightarrow$ Port}.
    \item \textbf{ Se não funcionar:} Se não conseguir colocar ports, tente \textbf{DLA Layout $\rightarrow$ Open}.
    \item \textbf{Camada:} Use a \textbf{Barra de espaço} para mudar a camada para \textbf{MET1} (Metal1) para todos os ports.
    \item Posicione os ports nos locais apropriados (entradas, saídas, VDD, VSS).
\end{itemize}

\subsubsection{Texto nos Ports}
Esta é uma etapa \textbf{crítica} para o LVS.
\begin{itemize}
    \item \textbf{Comando:} \textbf{Botão direito $\rightarrow$ Add $\rightarrow$ Text on Ports}.
    \item \textbf{Text layer}: Mude de \texttt{PIN} para \textbf{M1NET}. (Alguns tutoriais divergem , mas \texttt{M1NET} é a prática recomendada para evitar erros de LVS e DRC ).
    \item \textbf{Text height}: \textbf{1.0}.
\end{itemize}

\subsubsection{Aumentar Área dos Ports (Opcional, mas recomendado)}
Para garantir uma boa conexão com blocos externos.
\begin{enumerate}
    \item \textbf{Comando:} \textbf{Easy Edit $\rightarrow$ Add $\rightarrow$ Shape $\rightarrow$ MET1}.
    \item Desenhe um retângulo de Metal1 sobre a área de dreno/source que serve como port.
    \item \textbf{Comando:} \textbf{Connectivity $\rightarrow$ Port $\rightarrow$ Add to Port}.
\end{enumerate}

\subsection{Camadas Geradas}
Algumas camadas não são desenhadas, mas geradas com base em outras.
\begin{itemize}
    \item \textbf{Comando:} \textbf{HIT-Kit Utilities $\rightarrow$ Generated Layers}.
    \item \textbf{Marcar:} \textbf{NLDD} e \textbf{FIMP}.
    \item \textbf{Ação:} Clique em \textbf{OK}.
    \item \textbf{Importante:} Rode o DRC novamente após gerar essas camadas.
\end{itemize}

\subsection{Ligação com Símbolo (Propriedade phy\_comp)}
Esta etapa "amarra" o seu símbolo de esquemático ao seu layout finalizado.

\begin{enumerate}
    \item \textbf{Copiar Localização do Layout:}
    \begin{itemize}
        \item No \textbf{ICStation}, clique com o \textbf{botão direito} na aba do seu layout (ex: \texttt{inv/layout}).
        \item \textbf{Properties $\rightarrow$ Location}.
        \item Copie o caminho exibido (ex: \texttt{\$minha\_lib/default.group/layout.views/inv}).
    \end{itemize}
    \item \textbf{Adicionar Propriedade ao Símbolo:}
    \begin{itemize}
        \item Volte ao \textbf{Design Architect} (esquemático).
        \item Abra o \textbf{SÍMBOLO} (\texttt{File $\rightarrow$ Open $\rightarrow$ Symbol}).
        \item \textbf{Comando:} \textbf{Add $\rightarrow$ Properties}.
        \item \textbf{Property Name}: \texttt{phy\_comp}.
        \item \textbf{Property Value}: Cole a \textbf{Location} copiada do layout.
        \item Coloque o texto gerado abaixo do símbolo.
    \end{itemize}
    \item \textbf{Ajustar Texto (Opcional):}
    \begin{itemize}
        \item \textbf{ Setup $\rightarrow$ Select Filter $\rightarrow$ Properties} $\rightarrow$ OK .
        \item Selecione o texto \texttt{phy\_comp}.
        \item \textbf{Botão direito $\rightarrow$ Change Height $\rightarrow$ Specified $\rightarrow$ 0.2} .
    \end{itemize}
    \item \textbf{ Verificar Símbolo (A "Parte Mágica"):}
    \begin{itemize}
        \item \textbf{File $\rightarrow$ Check Symbol} (1 warning é esperado: \textit{"Property phy\_comp... not on the interface"}) .
        \item \textbf{Save}.
        \item \textbf{File $\rightarrow$ Check Symbol} (Agora 0 warnings são esperados) .
    \end{itemize}
\end{enumerate}

\subsection{Técnicas Avançadas}

\subsubsection{Adição de PADs}
PADs são as conexões externas do chip.
\begin{enumerate}
    \item \textbf{Comando:} \textbf{Objects $\rightarrow$ Add $\rightarrow$ Cell}.
    \item \textbf{Biblioteca:} \textbf{IOLIB\_4M}.
    \item \textbf{Célula:} \textbf{g\_padonly} $\rightarrow$ OK.
    \item Posicione os PADs (ex: um para VDD, um para VSS) nas bordas do layout.
    \item Conecte-os ao seu circuito usando \textbf{Shapes} de metal largas.
    \item \textbf{ Dica:} Comece grosso, e se der erro de DRC, diminua a largura.
\end{enumerate}

\subsubsection{Determinação do Tamanho da Célula}
\begin{itemize}
    \item \textbf{Comando:} \textbf{Report $\rightarrow$ Windows}.
    \item O console mostrará as coordenadas \textbf{CellExtent} (ex: \texttt{(x1, y1) (x2, y2)}).
    \item \textbf{Área} = (x2 - x1) * (y2 - y1).
\end{itemize}

\subsection{Dicas Críticas e Problemas Comuns}

\subsubsection{Organização do Layout}
\begin{itemize}
    \item \textbf{Minimizar Cruzamentos:} Antes de rotear, posicione os componentes para minimizar o número de "overflows" (linhas amarelas) que se cruzam.
    \item \textbf{Canal de Roteamento:} Tente deixar um caminho livre no meio (entre PMOS e NMOS) para o roteamento.
    \item \textbf{Alinhamento:} Use \textbf{Easy Edit $\rightarrow$ Align $\rightarrow$ Center X} (ou Y) para alinhar componentes.
\end{itemize}

\subsubsection{Problemas Comuns e Soluções}
\begin{itemize}
    \item \textbf{Mouse travado com Ctrl:} Se o mouse se comportar como se o Ctrl estivesse pressionado, pressione \textbf{Ctrl+Shift} e use as setas do teclado; isso geralmente resolve.
    \item \textbf{ICStation Travado:} Evite clicar repetidamente no scroll do mouse.
    \item \textbf{Células Sumiram da Biblioteca:} Crie um novo projeto com o nome antigo e copie a pasta \texttt{default.group} do projeto antigo/backup para a nova pasta do projeto .
    \item \textbf{Não consigo colocar Ports:} Verifique se o layout está aberto com \textbf{DLA Layout $\rightarrow$ Open}.
\end{itemize}

\subsubsection{Distâncias Críticas e Cálculos}
\begin{itemize}
    \item \textbf{POLY-POLY}: 0.45$\mu$m.
    \item \textbf{RES-POLY}: 0.35$\mu$m.
    \item \textbf{DIFF-NTUB}: 1.2$\mu$m.
    \item \textbf{NTUB enclosure} (borda do poço N ao redor do PMOS): 1.2$\mu$m.
    \item \textbf{ Área/Perímetro (AD/PD):} Para simulação, a altura da difusão é estimada como 0.85$\mu$m.
        \item $AD = 0.85 \times W$.
        \item $PD = W + (2 \times 0.85)$.
\end{itemize}

\subsection{Exemplo de Sequência Completa (Resumo)}
\begin{enumerate}
    \item \textbf{AutoInst} $\rightarrow$ Gera layout básico.
    \item \textbf{Merge} transistores $\rightarrow$ Otimiza área (usando Flip se necessário).
    \item \textbf{Adicionar bulks} (t) $\rightarrow$ Contatos de poço (pdm1, ndm1).
    \item \textbf{Configurar roteamento} $\rightarrow$ Direction e Options .
    \item \textbf{Run routing} (ARoute) $\rightarrow$ Automático + manual (IRoute) se necessário.
    \item \textbf{Adicionar ports} $\rightarrow$ Em camada MET1.
    \item \textbf{Text on Ports} $\rightarrow$ Configurar para camada M1NET.
    \item \textbf{Generated Layers} $\rightarrow$ Gerar NLDD + FIMP.
    \item \textbf{Verificações} $\rightarrow$ (DRC e LVS, detalhados na próxima seção).
    \item \textbf{Link com símbolo} $\rightarrow$ Adicionar propriedade \texttt{phy\_comp} ao símbolo.
\end{enumerate}
\newpage

\newpage
\section[\textcolor{red}{Exemplos de Prova: Estratégias de Layout}]{\textcolor{red}{Exemplos de Prova: Estratégias de Layout}}

\subsection[\textcolor{red}{Layout de Portas Lógicas (Manual / AutoInst)}]{\textcolor{red}{Layout de Portas Lógicas (Manual / AutoInst)}}
Aplicável para: \textbf{Prova 2016b (Q2)}, \textbf{Prova 2022a (Q2)} e \textbf{Prova 2024 (Q3)}.

Nestas questões, o objetivo é criar o layout físico de uma única porta (ex: OAI21) otimizando a área.

\textbf{Fluxo de Trabalho Recomendado:}
\begin{enumerate}
    \item \textbf{Geração Inicial:} No ICStation, use \texttt{DLA Layout $\rightarrow$ AutoInst}. Isso trará os transistores com as dimensões definidas no esquemático.
    \item \textbf{Otimização (Merge):}
    \begin{itemize}
        \item Identifique transistores que compartilham o mesmo sinal de dreno/source.
        \item Mova-os para que as áreas de difusão se sobreponham (merge).
        \item Use o comando \textbf{Flip} (atalho 'f') se necessário para alinhar dreno com dreno ou source com source.
    \end{itemize}
    \item \textbf{Contatos de Substrato (Bulks):}
    \begin{itemize}
        \item Adicione manualmente os contatos de poço (taps).
        \item Use \texttt{pdm1} para o poço N (PMOS) conectando ao VDD.
        \item Use \texttt{ndm1} para o substrato P (NMOS) conectando ao VSS.
        \item \textit{Regra Crítica:} Mantenha a distância correta entre a difusão e o poço.
    \end{itemize}
    \item \textbf{Ports e Textos:} Coloque ports em Metal 1 para A, B, C, OUT, VDD e VSS. Adicione \textbf{Text on Ports} na camada \texttt{M1NET} para o LVS.
\end{enumerate}

\subsection[\textcolor{red}{Layout Automático (Standard Cells)}]{\textcolor{red}{Layout Automático (Standard Cells)}}
Aplicável para: \textbf{Prova 2016b (Q7)}, \textbf{Prova 2022a (Q7)} e \textbf{Prova 2024 (Q7)}.

Nestas questões, você está fazendo o layout de um contador ou circuito complexo que usa células prontas da \texttt{CORELIB}. Não desenhamos transistores aqui.

\textbf{Fluxo de Trabalho (ICStation):}
\begin{enumerate}
    \item \textbf{Preparação:} Não use AutoInst.
    \item \textbf{Definição da Área (Floorplanning):}
    \begin{itemize}
        \item Vá em \texttt{Place \& Route $\rightarrow$ Autofp}.
        \item Em \textbf{Aspect Ratio}, defina \textbf{Upper = 2} (ou 1 para quadrado).
        \item Clique em \textbf{StdCell} e desenhe um retângulo na tela onde as células ficarão.
    \end{itemize}
    \item \textbf{Configuração de Alimentação (Crítico):}
    \begin{itemize}
        \item Antes de rotear, configure a largura dos trilhas de energia.
        \item \texttt{ARoute Net Classes $\rightarrow$ Edit $\rightarrow$ New}.
        \item Nome: \texttt{power}. Largura: \textbf{1.8} (para MET1 e MET2).
        \item Atribua os nets \texttt{VDD} e \texttt{VSS} a esta classe.
    \end{itemize}
    \item \textbf{Roteamento:}
    \begin{itemize}
        \item Execute \texttt{ARoute $\rightarrow$ Run}.
        \item Verifique se não sobraram conexões abertas (overflows).
    \end{itemize}
\end{enumerate}

\subsection[\textcolor{red}{Adição de PADs (Prova 2022a Q4)}]{\textcolor{red}{Adição de PADs (Prova 2022a Q4)}}
Requisito específico da prova de 2022, mas útil saber para todas.

\textbf{Procedimento:}
\begin{enumerate}
    \item \textbf{Instanciar PADs:}
    \begin{itemize}
        \item Vá em \texttt{Objects $\rightarrow$ Add $\rightarrow$ Cell}.
        \item Selecione a biblioteca \textbf{IOLIB\_4M}.
        \item Escolha a célula \textbf{g\_padonly} (geralmente a última da lista).
        \item Adicione duas instâncias: uma para VDD e outra para VSS.
    \end{itemize}
    \item \textbf{Posicionamento:} Coloque os PADs nas bordas do seu layout (longe do circuito principal para evitar violações de DRC de poço).
    \item \textbf{Conexão:}
    \begin{itemize}
        \item Desenhe um \textbf{Shape} (retângulo) de Metal 1 ou Metal 2 largo.
        \item Conecte o pino do PAD ao anel/trilha de alimentação correspondente (VDD ou VSS) do seu circuito.
    \end{itemize}
    \item \textbf{Verificação:} Rode o DRC. Erros comuns envolvem espaçamento insuficiente entre o PAD e outros metais ou poços.
\end{enumerate}

\newpage
\section{Calibre (DRC / LVS / PEX)}

\subsection{Introdução ao Calibre}
O Calibre é a ferramenta de verificação e extração utilizada para:
\begin{itemize}
    \item \textbf{DRC (Design Rule Checking)}: Verificação de regras de design (espaçamentos, larguras, etc.).
    \item \textbf{LVS (Layout vs Schematic)}: Comparação entre o layout (físico) e o esquemático (lógico).
    \item \textbf{PEX (Parasitic Extraction)}: Extração de componentes parasitas (R e C) do layout.
\end{itemize}

\subsection{Exemplo de Configuração Completa (Resumo)}
\begin{lstlisting}[language=bash, frame=none, commentstyle=\color{black}]
### DRC (Design Rule Checking) ###
1. No ICStation: Calibre -> Run DRC
2. Na aba "Rules":
   - Load file: /local/users/cad/Calibre_rules/cac35b4rules_all.run
3. Na aba "Inputs", painel "Layout":
   - Top Cell: minha_celula
   - Marcar: Export from layout viewer
4. Clique: Run DRC
5. Resultado: Corrigir erros até restarem apenas os 6-7 warnings
   (ILL_METx, INFO_PROCESS, INFO_TEXT_VDD/VSS).

### LVS (Layout vs Schematic) ###
1. No ICStation: Calibre -> Run LVS
2. Na aba "Rules":
   - Load file: /local/users/cad/Calibre_rules/cac35b4rules_all.run
3. Na aba "Inputs", painel "Layout":
   - Top Cell: minha_celula
   - Marcar: Export from layout viewer
4. Na aba "Inputs", painel "Netlist":
   - Marcar: Export from schematic viewer
5. Na aba "Setup", painel "LVS Options":
   - Sub-aba "Supply": Marcar "Ignore layout and source pins during comparison"
6. Clique: Run LVS
7. Resultado: "Carinha feliz" (The net-lists match).

### PEX (Parasitic Extraction) ###
1. No ICStation: Calibre -> Run PEX
2. Na aba "Rules":
   - Load file: /local/users/cad/Calibre_rules/cac35b4rules_all.run
3. Na aba "Outputs", painel "Netlist":
   - Use Names From: Selecione LAYOUT
4. Na aba "Outputs", painel "Extraction Type":
   - Selecione: C+CC ou R+C+CC (conforme a prova)
5. Clique: Run PEX
6. Resultado: Arquivo .pex.netlist gerado na pasta .cal/

### Preparo do Netlist PEX (Obrigatório!) ###
Edite o arquivo .pex.netlist gerado:
1. Adicionar Ports: Adicione os nomes (VDD, VSS, A, B, OUT) na linha .subckt.
2. Conectar Alimentação: Adicione linhas .connect (ex: .connect VSS N_VSS_M0_s).
3. Corrigir Resistores: Mude "rR0" para "XR0".
4. Corrigir Bipolares: Apague o parâmetro "AREA" se houver (ex: VERT10).
\end{lstlisting}

\subsection{DRC (Design Rule Checking)}

\subsubsection{Execução do DRC}
\begin{enumerate}
    \item No \textbf{ICStation}: \textbf{Calibre $\rightarrow$ Run DRC}.
    \item \textbf{Rules}: Carregar o arquivo de regras (ex: \texttt{/local/users/cad/work/rules/cac35b4rules\_all.run}).
    \item \textbf{Load} (Obrigatório clicar em Load após selecionar o arquivo).
    \item \textbf{Inputs $\rightarrow$ Layout}:
    \begin{itemize}
        \item \textbf{Top Cell}: \texttt{nome\_da\_celula}
        \item \textbf{Export from layout viewer}: Marcado
    \end{itemize}
    \item \textbf{Run DRC}
\end{enumerate}

\subsubsection{Erros DRC Esperados e Aceitáveis}
Em um layout finalizado, é normal e \textbf{aceitável} que restem os seguintes erros (são informativos):
\begin{lstlisting}[language=bash, frame=none]
Check ILL_MET2_DIE_RATIO_M2R1 - 1 Result
Check ILL_MET3_DIE_RATIO_M3R1 - 1 Result
Check ILL_MET4_DIE_RATIO_M4R1 - 1 Result
Check ILL_POLY1_DIE_RATIO_POR1 - 1 Result
Check INFO_PROCESS_C35B4C3 - 1 Result
Check INFO_TEXT_VDD - 1 Result
Check INFO_TEXT_VSS - 1 Result
\end{lstlisting}

\subsubsection{Erros DRC Críticos e Soluções}
\begin{itemize}
    \item \textbf{SPACE DIFF POLY e MET1}: Causa: Dimensões (espaçamentos) incorretas. Solução: Refazer as dimensões na área problemática.
    \item \textbf{NWELL TOO HOT}: Causa: Poço N (NTUB) flutuante ou mal conectado. Solução: Aumentar a largura do metal VDD (fazer "grossão") ou adicionar mais contatos de poço (pdm1). Verificar se o \textbf{Text on Ports} está correto (M1NET).
    \item \textbf{FIMP e NLDD MISSING}: Causa: Camadas geradas não foram criadas. Solução: Rodar \textbf{HIT-Kit Utilities $\rightarrow$ Generated Layers} e marcar \textbf{NLDD} e \textbf{FIMP} . Se não funcionar, feche e reabra o layout.
    \item \textbf{SOFT CONNECTION}: Causa: Conexão inadequada de VDD/VSS (ex: "o VDD grossão só ta encostando a cabecinha"). Solução: Verificar se os metais estão realmente conectados (sobrepostos) e não apenas encostando.
    \item \textbf{DIODE error (gctcg)}: Causa: Tap (contato de bulk) colado diretamente no Gate. Solução: Colocar um contato de difusão ("c") entre o tap ("t") e o gate ("g").
    \item \textbf{ILL\_MISS\_MET4BLOCK\_AMTS1}: Causa: Uso da camada MET4. Solução: Idealmente, não usar MET4, mas pode ser ignorado se necessário .
    \item \textbf{ ILL\_FLOATING\_GATE\_ERC}: Causa: Gate flutuante. Solução: Verifique se escolheu a camada correta para cada port (ex: MET1).
\end{itemize}

\subsection{LVS (Layout vs Schematic)}

\subsubsection{Método Antigo (ICTrace)}
\begin{enumerate}
    \item \textbf{Comando:} \textbf{ICTrace (M) $\rightarrow$ LVS}.
    \item \textbf{Source name}: Apontar para o ViewPoint:
    \begin{itemize}
        \item \texttt{\$celula/default.group/logic.views/nome/vpt\_c35b4\_device} .
    \end{itemize}
    \item \textbf{Abort on Supply Error}: No.
    \item \textbf{OK}
    \item Verificar resultados: \textbf{Report $\rightarrow$ LVS}.
\end{enumerate}

\subsubsection{Método Calibre (Recomendado)}
\begin{enumerate}
    \item \textbf{Comando:} \textbf{Calibre $\rightarrow$ Run LVS}.
    \item \textbf{Rules}: Carregar o arquivo de regras (ex: \texttt{/local/users/cad/Calibre\_rules/cac35b4rules\_all.run}).
    \item \textbf{Load} (Obrigatório).
    \item \textbf{Inputs $\rightarrow$ Layout}:
    \begin{itemize}
        \item \textbf{Files}: \texttt{nome\_da\_celula}.
        \item \textbf{Top Cell}: \texttt{nome\_da\_celula}.
        \item \textbf{Export from layout viewer}: Marcado.
    \end{itemize}
    \item \textbf{Inputs $\rightarrow$ Netlist}:
    \begin{itemize}
        \item \textbf{Export from schematic viewer}: Marcado (puxa o netlist do esquemático aberto).
        \item \textbf{Verificar Path:} Ocasionalmente, confira o caminho do netlist em \textbf{Files}.
    \end{itemize}
    \item \textbf{Setup $\rightarrow$ LVS Options}:
    \begin{itemize}
        \item Marcar: \textbf{Ignore layout and source pins during comparison}.
        \item \textbf{ Desmarcar:} (Prova 2016) \textbf{Ignore layout and source pins during comparison} (alguns tutoriais pedem para desmarcar). Teste as duas opções.
    \end{itemize}
    \item \textbf{Run LVS}
\end{enumerate}

\subsubsection{Erros LVS Comuns e Soluções}
\begin{itemize}
    \item \textbf{Discrepancy: Incorrect Port}: Erro mais comum.
    \begin{itemize}
        \item \textbf{Causa:} Ports não reconhecidos.
        \item \textbf{ Solução (Conflitante):} Os documentos de referência são conflitantes, indicando um "bug" ou trade-off no fluxo:
        \begin{itemize}
            \item \textbf{Solução 1:} Usar \textbf{Text on Ports $\rightarrow$ M1NET}. Isso corrige os erros de DRC (como \texttt{NWELL TOO HOT}).
            \item \textbf{Solução 2:} Se o LVS falhar com M1NET, use \textbf{M1PIN}. Isso pode corrigir o LVS, mas pode reintroduzir os erros \texttt{INFO\_TEXT\_VDD} e \texttt{INFO\_TEXT\_VSS} no DRC.
        \end{itemize}
        \item \textbf{Recomendação:} Tente \texttt{M1NET} primeiro. Se o LVS falhar, tente \texttt{M1PIN}.
    \end{itemize}
    \item \textbf{Supply Errors}:
    \begin{itemize}
        \item \textbf{Causa:} Problemas com VDD/VSS (ex: "soft connection").
        \item \textbf{Solução:} Verificar conexões de alimentação no layout.
    \end{itemize}
    \item \textbf{Model not found (RPOLYH, etc.)}:
    \begin{itemize}
        \item \textbf{Causa:} Modelos de dispositivos (como resistores) faltando no netlist.
        \item \textbf{Solução:} Editar o arquivo netlist de simulação e adicionar os \texttt{.include} necessários (ex: \texttt{restm.mod}).
    \end{itemize}
\end{itemize}

\subsubsection{Debug do LVS}
\begin{enumerate}
    \item \textbf{Comando:} \textbf{ICTrace (M) $\rightarrow$ Discreps} (para o método antigo).
    \item Usar \textbf{first} e \textbf{next} para navegar pelos erros .
    \item \textbf{Unshow $\rightarrow$ All} para deselecionar regiões destacadas.
\end{enumerate}

\subsection{PEX (Parasitic Extraction)}

\subsubsection{Execução do PEX}
\begin{enumerate}
    \item \textbf{Comando:} \textbf{Calibre $\rightarrow$ Run PEX}.
    \item \textbf{Outputs $\rightarrow$ Netlist}:
    \begin{itemize}
        \item \textbf{Use Names From}: Mudar para \textbf{LAYOUT} (importante para que os nomes dos ports (A, B, OUT) sejam mantidos no netlist extraído).
    \end{itemize}
    \item \textbf{Outputs $\rightarrow$ Extraction Type}: 
    \begin{itemize}
        \item \textbf{C+CC}: Capacitâncias parasitas (intrínsecas e acoplamento).
        \item \textbf{R}: Apenas Resistências parasitas.
        \item \textbf{R+C}: Resistências + capacitâncias intrínsecas.
        \item \textbf{R+C+CC}: Todos componentes parasitas (mais completo e pesado).
    \end{itemize}
    \item \textbf{Run PEX}
\end{enumerate}

\subsubsection{Arquivos Gerados pelo PEX}
O PEX cria uma pasta \texttt{.cal/} e gera vários arquivos:
\begin{itemize}
    \item \texttt{nome\_celula.pex.netlist} (O netlist principal que usaremos na simulação).
    \item \texttt{nome\_celula.pex.netlist.pex} 
    \item \texttt{nome\_celula.pex.netlist.nome\_celula.pxi} 
\end{itemize}

\subsubsection{Localização dos Arquivos}
Os arquivos extraídos ficam no diretório da célula, dentro da pasta \texttt{.cal/}:
\begin{lstlisting}[language=bash, frame=none]
/local/users/cad/work/nome_projeto.proj/cell.lib/
  default.group/layout.views/nome_celula/
    nome_celula.cal/
      nome_celula.pex.netlist
\end{lstlisting}

\subsubsection{Preparação do Netlist PEX para Simulação}
O arquivo \texttt{.pex.netlist} gerado \textbf{não} funciona diretamente no ELDO. Ele precisa de edições manuais:

\begin{enumerate}
    \item \textbf{Acrescentar ports na linha .subckt}: A linha \texttt{.subckt} gerada pode não incluir os ports VDD/VSS ou as entradas/saídas. Adicione-os manualmente.
    \begin{lstlisting}[language=pspice]
    * Original: .subckt GATE
    * Corrigido: .subckt GATE VSS VDD A B C OUT
    \end{lstlisting}

    \item \textbf{Conectar nós flutuantes (se houver)}: Às vezes, VDD/VSS são extraídos como nós internos (ex: \texttt{N\_VSS\_M0\_s}). Conecte-os globalmente .
    \begin{lstlisting}[language=pspice]
    .connect VSS N_VSS_MO_s
    .connect VDD N_VDD_MO_d
    \end{lstlisting}

    \item \textbf{Substituir modelos de transistor}: O PEX usa os nomes do layout (\texttt{NMOS4}, \texttt{PMOS4}). O ELDO espera os nomes do \texttt{.defmod} (ex: \texttt{MODN}, \texttt{MODP}).
    
    \item \textbf{Corrigir resistores}: Se houver resistores (rpolyh), o PEX pode instanciá-los como \texttt{rR0} (instância 'r'). O SPICE espera \texttt{XR0} (instância 'X' para subcircuitos).
    
    \item \textbf{Remover parâmetro AREA}: Se houver transistores bipolares (ex: \texttt{VERT10}), o PEX adiciona um parâmetro \texttt{AREA=...} que causa erro no ELDO. Delete-o.
\end{enumerate}

\subsubsection{Includes Necessários}
Para simular o netlist extraído, seu arquivo \texttt{.cir} principal precisará dos modelos corretos:
\begin{lstlisting}[language=pspice]
* Incluir no inicio do arquivo .cir
.include "/local/tools/dkit/ams_3.70_mgc/eldo/c35/modeloWP"
.include "/local/tools/dkit/ams_3.70_mgc/eldo/c35/modeloWS"
.include "/local/tools/dkit/ams_3.70_mgc/eldo/c35/modeloMOD"
\end{lstlisting}

\subsection{Verificação de Resultados}

\subsubsection{Relatórios do Calibre}
\begin{itemize}
    \item \textbf{DRC Report}: Lista todas as violações de regras.
    \item \textbf{LVS Report}: Mostra o status do "matching" (se bate ou não) entre layout e esquemático.
    \item \textbf{PEX Report}: Mostra estatísticas da extração (quantos R e C foram extraídos).
\end{itemize}

\subsubsection{Interpretação dos Relatórios}
\textbf{DRC Bem-sucedido:}
\begin{itemize}
    \item 0 erros críticos.
    \item Apenas os 7 erros "informativos" esperados.
\end{itemize}

\textbf{LVS Bem-sucedido:}
\begin{itemize}
    \item O relatório dirá: \textbf{"The net-lists match."}.
    \item 0 discrepancies.
    \item Todos os dispositivos (devices) e nets estão "matching".
\end{itemize}

\textbf{PEX Bem-sucedido:}
\begin{itemize}
    \item Netlist gerada sem erros de sintaxe (após as correções manuais).
    \item Arquivos .pxi gerados corretamente.
\end{itemize}

\subsection{Configurações Avançadas e Fluxos de Trabalho}

\subsubsection{Configurações do Calibre}
\begin{itemize}
    \item \textbf{Setup $\rightarrow$ DRC Options}: Configurações específicas de DRC.
    \item \textbf{Setup $\rightarrow$ LVS Options}: \textbf{Supply} (configurações de VDD/VSS), \textbf{Ignore} (ignorar certos elementos).
    \item \textbf{Setup $\rightarrow$ PEX Options}: Opções detalhadas de extração.
\end{itemize}

\subsubsection{Problemas de Performance}
\begin{itemize}
    \item \textbf{LVS lento}: Reduzir a complexidade do layout ou verificar hierarquia.
    \item \textbf{PEX com muita memória}: Usar extração mais simples (ex: C+CC ao invés de R+C+CC).
\end{itemize}

\subsubsection{Dicas e Soluções de Problemas}
\begin{itemize}
    \item \textbf{Calibre não carrega rules}: Verifique o caminho do arquivo \texttt{.run} e clique em \textbf{Load}.
    \item \textbf{LVS não encontra schematic}: Verifique se o ViewPoint foi criado corretamente.
    \item \textbf{PEX gera netlist vazia}: Verifique o "Extraction Type" selecionado.
\end{itemize}

\subsubsection{Fluxo Recomendado}
\begin{enumerate}
    \item \textbf{Rodar DRC} $\rightarrow$ Corrigir todos os erros críticos (vermelhos).
    \item \textbf{Rodar LVS} $\rightarrow$ Corrigir "discrepancies" (especialmente "Incorrect Port").
    \item \textbf{Rodar PEX} $\rightarrow$ Preparar o netlist extraído (corrigindo-o manualmente).
    \item \textbf{Simular} o netlist PEX e comparar com a simulação do esquemático.
\end{enumerate}

\subsubsection{Scripts Úteis}
Para converter as imagens de tela (screenshots) do Calibre/ICStation:
\begin{lstlisting}[language=bash]
# Capturar a janela
xwd > layout.xwd

# Converter para TIF/PNG (invertendo cores)
convert -white-threshold 1 -negate layout.xwd layout.tif
convert -white-threshold 1 -negate layout.xwd layout.png
\end{lstlisting}

\newpage

\section{Análise de Circuitos Lógicos e Sequenciais}

A análise precisa de circuitos digitais requer a configuração correta de estímulos transientes, o uso de medições automáticas (\texttt{.meas}) para quantificar desempenho e a interpretação correta das formas de onda no visualizador (EZWave). Esta seção detalha o fluxo de trabalho para caracterizar atrasos, potência e limites de frequência, cobrindo os requisitos das Provas de 2016, 2022 e 2024.

\subsection{Configuração de Estímulos Transientes}

\subsubsection{Onda Quadrada Precisa (Pulse)}
A definição correta da fonte de tensão é o primeiro passo para evitar erros de simulação. Um erro comum é definir a largura do pulso (\texttt{pulse\_width}) simplesmente como $T/2$. Como os tempos de subida ($t_r$) e descida ($t_f$) ocupam tempo, isso resulta em um \textit{duty cycle} incorreto (menor que 50\%), o que altera a medição de potência dinâmica.

\textbf{O Código Robusto:}
O script abaixo calcula a largura do pulso descontando as transições, garantindo simetria perfeita.

\begin{lstlisting}[language=pspice, caption={Configuração Robusta de Onda Quadrada (Copiar e Colar)}]
* --- PARAMETROS DE TEMPO ---
* Ajuste 'period' conforme a frequencia desejada (Ex: 10n = 100MHz)
.Param period = 10n
.Param frequency = '1/period'

* --- NIVEIS DE TENSAO ---
* Ajuste 'high_value' conforme VDD da prova (3.0V ou 3.3V)
.Param low_value = 0 
.Param high_value = 3.0V  
.Param delay = 0

* --- TEMPOS DE TRANSICAO ---
* Provas pedem geralmente 5% a 10% do periodo ou fixo (0.1ns a 0.5ns)
.Param rise_time = 0.2ns 
.Param fall_time = 0.2ns

* --- CALCULO AUTOMATICO DA LARGURA (PULSE WIDTH) ---
* Subtrai a media das bordas para garantir 50% de Duty Cycle exato
.Param pulse_width = 'period/2 - ((rise_time + fall_time)/2)'

* --- FONTE DE TENSAO ---
* Sintaxe: PULSE(Vlow Vhigh delay tr tf pw per)
Va A 0 pulse(low_value high_value delay rise_time fall_time pulse_width period)
\end{lstlisting}

\begin{figure}[H]
    \centering
    
    \caption{Ilustração dos parâmetros da fonte Pulse no ELDO: observe como o PW é ajustado para manter o duty cycle.}
\end{figure}

\newpage

\subsection{Análise de Desempenho (Sweep de Carga e Potência)}

\subsubsection{Teoria e Análise}
Em circuitos CMOS, dois parâmetros são críticos e dependentes da carga:
\begin{enumerate}
    \item \textbf{Atraso ($t_{pd}$):} Aumenta linearmente com a capacitância de carga ($C_L$).
    \item \textbf{Potência ($P_{avg}$):} Aumenta com a frequência e a carga ($P \approx f C V^2$).
\end{enumerate}
Para analisar isso em uma única execução, utilizamos o comando \texttt{SWEEP} dentro da análise \texttt{.tran}.

\textbf{Código Completo:}
Este script carrega o netlist extraído (PEX), aplica o modelo "Worst Power" (comum em provas) e gera os dados de potência e atraso.

\begin{lstlisting}[language=pspice, caption={Script de Varredura de Carga e Potência}]
* 1. INCLUDES E MODELOS (CRITICO)
* Altere o nome do arquivo para o seu .pex.netlist
.include "minha_celula.pex.netlist"

* Modelos de transistor (WP = Worst Power, comum na Q4 2024)
.include "/local/tools/dkit/ams_3.70_mgc/eldo/c35/modeloWP"
.defmod pmos4 modp
.defmod nmos4 modn

* 2. SETUP DE FONTES
.Param VDD_ANALISE = 3.3V
Vdd VDD 0 DC VDD_ANALISE
Vss VSS 0 DC 0

* Capacitor Variavel (Valor inicial dummy, sera sobrescrito pelo SWEEP)
CL OUT 0 50fF  

* 3. ESTIMULO (Onda Quadrada - Inserir parametros do exemplo anterior aqui)
.Param period=10n ... (copiar parametros de tempo)
Va A 0 pulse(...) 

* 4. MEDICOES AUTOMATICAS (.MEAS)
* --- Potencia Media ---
* P = VDD * I_media. O sinal '-' corrige o sentido da corrente no SPICE.
.meas tran corrente_media AVG I(Vdd)
.meas tran potencia_consumida param='VDD_ANALISE * -corrente_media'

* --- Atrasos de Propagacao (50% a 50%) ---
.meas tran atraso_subida trig V(A) val='VDD_ANALISE/2' fall=5 targ V(OUT) val='VDD_ANALISE/2' rise=5
.meas tran atraso_descida trig V(A) val='VDD_ANALISE/2' rise=5 targ V(OUT) val='VDD_ANALISE/2' fall=5

* 5. SIMULACAO COM SWEEP
* Sintaxe: .tran passo final SWEEP var INCR passo inicio fim
* Aqui: Varre CL de 0fF a 300fF com passos de 50fF
.Param sweep_start=0.0fF sweep_end=300.0fF sweep_step=50.0fF
.tran 0.1n '10*period' 0 0.01n SWEEP CL INCR sweep_step sweep_start sweep_end
\end{lstlisting}

\subsubsection{Análise Passo a Passo no EZWave}
Após rodar a simulação (\texttt{eldo arquivo.cir}), siga estes passos para gerar os gráficos exigidos na prova:

\begin{enumerate}
    \item Abra o arquivo de resultados (\texttt{.wdb}).
    \item No menu superior, vá em \textbf{View $\rightarrow$ Measurement Results} (Atalho: \texttt{CTRL+M} em algumas versões, ou procure o ícone de tabela).
    \item Uma janela se abrirá listando as medições: \texttt{potencia\_consumida}, \texttt{atraso\_subida}, etc.
    \item \textbf{Selecionar e Plotar:} Clique na medição desejada e arraste para a área de plotagem.
    \item \textbf{Eixo X Automático:} O EZWave detecta automaticamente que a variável varrida foi $C_L$ e a coloca no eixo X.
    \item \textbf{Destaque dos Pontos:} Para provar que você simulou 5 ou mais pontos (conforme pede a prova), clique com o botão direito na curva $\rightarrow$ \textbf{Properties} $\rightarrow$ ative \textbf{Data Point Symbol}.
\end{enumerate}

\begin{figure}[H]
    \centering
    
    \caption{Exemplo de gráfico Potência vs Carga ($C_L$) gerado. Note a linearidade.}
\end{figure}

\newpage

\subsection{Análise de Frequência Máxima ($F_{max}$)}

\subsubsection{Teoria e Critério de Falha}
A $F_{max}$ é a frequência limite onde o circuito deixa de fornecer níveis lógicos válidos na saída devido ao tempo insuficiente para carga/descarga dos capacitores parasitas.
\begin{itemize}
    \item \textbf{Critério de Aceitação (Degradação):}
    \begin{itemize}
        \item Nível lógico "1" deve chegar a pelo menos 95\% de $V_{DD}$.
        \item Nível lógico "0" deve descer até no máximo 5\% de $V_{DD}$.
    \end{itemize}
\end{itemize}

\textbf{Código Completo:}
Este script varre a frequência e usa a função \texttt{FIND} do ELDO para monitorar os picos de tensão.

\begin{lstlisting}[language=pspice, caption={Script de Varredura de Frequência para Fmax}]
* ... (Includes e Modelos iguais ao anterior) ...

* 1. ESTIMULO COM FREQUENCIA VARIAVEL
* Observe que usamos a variavel 'f' no parametro period
.Param f = 100Meg
.Param period = '1/f'
* Definir Rise/Fall como porcentagem pequena do periodo (ex: 5%)
.Param tr = '0.05 * period'
Va IN 0 pulse(0 3.3V 0 tr tr 'period/2 - tr' period)

* 2. MEDICOES DE DEGRADACAO
* Encontra o valor MAXIMO que o nivel BAIXO atinge (Ideal = 0V)
* Se subir muito (ex: > 0.15V), falhou.
.meas tran minZero find v(OUT) when v(IN)='3.3V * 0.95' fall=5

* Encontra o valor MINIMO que o nivel ALTO atinge (Ideal = 3.3V)
* Se cair muito (ex: < 3.15V), falhou.
.meas tran maxUm find v(OUT) when v(IN)='3.3V * 0.05' rise=5

* 3. SIMULACAO
* Varre 'f' de 100MHz a 2GHz
.tran 0.1n '20*period' SWEEP f INCR 100Meg 100Meg 2G
\end{lstlisting}

\subsubsection{Análise Detalhada no EZWave (Ferramenta Crossing)}
Esta análise requer precisão para encontrar o ponto exato da falha.

\begin{enumerate}
    \item Plote as curvas \texttt{maxUm} e \texttt{minZero} em função da frequência.
    \item Use a ferramenta **Crossing** (\texttt{CTRL + M} $\rightarrow$ Selecione \textbf{All Types} $\rightarrow$ \textbf{Crossing}).
    \item \textbf{Configurar Limites:}
    \begin{itemize}
        \item Para a curva \texttt{minZero}: Defina \textbf{Y Level} = 0.165V (5\% de 3.3V).
        \item Para a curva \texttt{maxUm}: Defina \textbf{Y Level} = 3.135V (95\% de 3.3V).
    \end{itemize}
    \item \textbf{Resultado:} O EZWave marcará a frequência exata onde o sinal cruza esses limites. A $F_{max}$ do circuito é o **menor** valor de frequência entre os dois eventos.
\end{enumerate}

\begin{figure}[H]
    \centering
    
    \caption{Degradação dos níveis lógicos com a frequência. O ponto de cruzamento define a $F_{max}$.}
\end{figure}

\newpage

\subsection{Análise Estatística (Monte Carlo)}

\subsubsection{Teoria e Código}
A análise de Monte Carlo executa a simulação $N$ vezes, variando aleatoriamente os parâmetros dos transistores (como $V_{th}$ e mobilidade) dentro das tolerâncias da fábrica. Isso gera uma distribuição estatística do desempenho.

\textbf{Código Completo:}
Atenção: O comando `.MC` é especial e não deve ser misturado na mesma linha do `.tran`.

\begin{lstlisting}[language=pspice, caption={Configuração Monte Carlo (Prova 2022a)}]
* 1. INCLUDES ESPECIFICOS PARA MC
* Necessario carregar o arquivo de estatisticas (wc53.lib mc)
.INCLUDE /local/tools/dkit/ams_3.70_mgc/eldo/c35/profile.opt
.LIB /local/tools/dkit/ams_3.70_mgc/eldo/c35/wc53.lib mc

.include "minha_porta.pex.netlist"
.defmod pmos4 modp
.defmod nmos4 modn

* 2. PARAMETROS
.Param VDD_VAL = 2.8V
Vdd VDD 0 DC VDD_VAL
Vss VSS 0 0

* 3. COMANDO MONTE CARLO
* 75 rodadas, variacao mismatch (local) e process (global)
.MC 75 NBBINS=20 mismatch process

* 4. MEDICAO
.meas tran tphl trig V(IN) val='VDD_VAL/2' rise=1 targ V(OUT) val='VDD_VAL/2' fall=1

* 5. EXECUCAO
.tran 0 50n 0 0.1n
\end{lstlisting}

\subsubsection{Análise Detalhada no EZWave (Histograma)}
Não se analisa as ondas no domínio do tempo diretamente, pois haverá 75 curvas sobrepostas ("nuvem de curvas").
\begin{enumerate}
    \item Vá em \textbf{Plot $\rightarrow$ Monte Carlo $\rightarrow$ Histogram}.
    \item Selecione a medição desejada (ex: \texttt{tphl}).
    \item O gráfico mostrará barras indicando a frequência de cada valor de atraso.
    \item Observe a legenda do gráfico: ela mostrará a \textbf{Mean} (Média) e \textbf{Sigma} (Desvio Padrão). Estes são os valores que devem ser anotados na prova.
\end{enumerate}

\begin{figure}[H]
    \centering
    
    \caption{Histograma de Monte Carlo. Uma distribuição estreita indica um circuito robusto.}
\end{figure}

\newpage

% --- PARTE DE EXEMPLOS DE PROVAS ---

\subsection[\textcolor{red}{Exemplos Práticos de Provas}]{\textcolor{red}{Exemplos Práticos de Provas}}

\subsubsection[\textcolor{red}{Prova 2024 Q4: Sweep de Carga e Potência (Resolução Completa)}]{\textcolor{red}{Prova 2024 Q4: Sweep de Carga e Potência (Resolução Completa)}}

\textbf{Objetivo:} Determinar como a potência dinâmica e o atraso da porta variam ao aumentar a carga capacitiva na saída. Isso simula o efeito de ligar a porta a vários outros blocos ou a fios longos.

\textbf{Teoria Rápida:}
Em circuitos CMOS digitais, a Potência Dinâmica é dada por:
\[ P_{dyn} = f \cdot C_L \cdot V_{DD}^2 \]
Portanto, esperamos que o gráfico \textbf{Potência vs. $C_L$} seja uma \textbf{reta (linear)} com inclinação positiva.

\textbf{1. Preparação do Netlist (.cir)}
Este script realiza uma simulação transiente (\texttt{.tran}) enquanto varia o valor do capacitor $C_L$ (\texttt{SWEEP}).

\begin{lstlisting}[language=pspice, caption={Script de Varredura de Carga (Q4 2024)}]
* 
* 1. INCLUDES E MODELOS
* Substitua pelo seu arquivo PEX (extraido com R+C ou C+CC)
.include "minha_celula.pex.netlist"

* --- MODELO WORST POWER (WP) ---
* O enunciado exige WP (transistores rapidos, consumo maximo)
.include "/local/tools/dkit/ams_3.70_mgc/eldo/c35/modeloWP"
* Mapeamento dos modelos (Do PEX para o .MOD)
.defmod pmos4 modp
.defmod nmos4 modn

* 2. PARAMETROS GERAIS
.Param VDD_VAL = 3.3V
.Param TEMP_VAL = 27
.Option TEMP = TEMP_VAL

* 3. FONTES DE ALIMENTACAO
Vdd VDD 0 DC VDD_VAL
Vss VSS 0 0

* 4. ESTIMULO (Onda Quadrada precisa)
* Frequencia de operacao (ex: 100MHz)
.Param period = 10n
.Param tr = 0.2n
.Param tf = 0.2n
* Largura do pulso calculada para 50% de Duty Cycle exato
.Param pw = 'period/2 - (tr+tf)/2'

* Fonte de entrada (Injetada no pino A)
Va A 0 PULSE(0 VDD_VAL 0 tr tf pw period)

* 5. CARGA CAPACITIVA VARIAVEL
* Define o capacitor CL no no de saida (OUT)
* O valor '1fF' eh dummy, sera sobrescrito pelo SWEEP
CL OUT 0 1fF

* 6. MEDICOES AUTOMATICAS (.MEAS)

* --- A) Potencia Media ---
* Medimos a corrente media fornecida pelo VDD.
* O sinal negativo (-) corrige o sentido da corrente do SPICE (que sai da fonte).
.meas tran I_media AVG I(Vdd)
.meas tran Potencia_Total param='-I_media * VDD_VAL'

* --- B) Atrasos (Opcional, mas enriquece a resposta "Graficos de Saida") ---
.meas tran tplh trig v(A) val='VDD_VAL/2' fall=1 targ v(OUT) val='VDD_VAL/2' rise=1
.meas tran tphl trig v(A) val='VDD_VAL/2' rise=1 targ v(OUT) val='VDD_VAL/2' fall=1
.meas tran t_medio param='(tplh + tphl)/2'

* 7. SIMULACAO COM SWEEP (O CORACAO DA QUESTAO)
* Sintaxe: .tran passo fim SWEEP var INCR passo inicio fim
* Faixa: 0fF a 300fF (cobre cargas leves e pesadas)
* Passo: 50fF (Gera os pontos: 0, 50, 100, 150, 200, 250, 300 = 7 pontos)
* Garantindo o requisito de "pelo menos 5 valores".

.tran 0.01n '5*period' SWEEP CL INCR 50fF 0fF 300fF
\end{lstlisting}

\textbf{2. Procedimento de Simulação e Análise}

\begin{enumerate}
    \item \textbf{Executar:} Rode o comando \texttt{eldo nome\_arquivo.cir} no terminal.
    \item \textbf{Abrir Resultados:} Abra o EZWave (\texttt{ezwave nome\_arquivo.wdb}).
    \item \textbf{Visualizar a Tabela de Medições:}
    \begin{itemize}
        \item Vá em \textbf{View $\rightarrow$ Measurement Results} (ou use o atalho, ícone de tabela).
        \item Uma janela listará \texttt{Potencia\_Total}, \texttt{t\_medio}, etc.
    \end{itemize}
    \item \textbf{Plotar o Gráfico:}
    \begin{itemize}
        \item Selecione \texttt{Potencia\_Total} na tabela.
        \item Arraste para a área de gráficos.
        \item O eixo X será automaticamente $C_L$ (Load Capacitance) e o eixo Y será a Potência (Watts).
    \end{itemize}
    \item \textbf{Destacar os Pontos (Dica de Ouro):}
    \begin{itemize}
        \item Para provar ao professor que você simulou os 5+ pontos exigidos:
        \item Clique com o botão direito na linha do gráfico $\rightarrow$ \textbf{Properties}.
        \item Vá na aba \textbf{Attributes} (ou Trace) e ative a opção \textbf{Data Point Symbol} (escolha círculos ou xis).
        \item Isso fará com que cada ponto simulado (0, 50, 100...) fique marcado visualmente sobre a reta.
    \end{itemize}
\end{enumerate}



\textbf{3. O que escrever na folha da prova?}
\begin{itemize}
    \item \textbf{Comportamento:} "O gráfico mostra que a potência consumida aumenta linearmente com a capacitância de carga $C_L$."
    \item \textbf{Justificativa:} "Isso está de acordo com a teoria de potência dinâmica em circuitos CMOS ($P = f \cdot C_L \cdot V_{DD}^2$), onde a carga e descarga do capacitor de saída domina o consumo no modelo Worst Power."
    \item \textbf{Valores:} Cite o valor mínimo (em 0fF, que é a capacitância parasita interna) e o máximo (em 300fF) para dar dimensão à resposta.
\end{itemize}

\newpage

\subsubsection[\textcolor{red}{Prova 2024 Q8: Relação $F_{in}/F_{out}$ em Contador (Resolução Completa)}]{\textcolor{red}{Prova 2024 Q8: Relação $F_{in}/F_{out}$ em Contador (Resolução Completa)}}

\textbf{Objetivo:} Determinar experimentalmente a taxa de divisão de um contador ($N = F_{in}/F_{out}$) e encontrar a frequência máxima ($F_{max}$) na qual o circuito para de dividir corretamente.

\textbf{Teoria Rápida:}
Um contador de $k$ bits funciona como um divisor de frequência por $2^k$.
\begin{itemize}
    \item Ex: Contador de 4 bits $\rightarrow$ Divide por $2^4 = 16$.
    \item \textbf{Comportamento esperado:} O gráfico de $F_{in}/F_{out}$ deve ser uma linha reta horizontal no valor de divisão (ex: 16) até atingir a $F_{max}$.
    \item \textbf{Falha:} Acima de $F_{max}$, o atraso de propagação supera o período do clock. O contador perde pulsos e a relação cai ou torna-se caótica.
\end{itemize}

\textbf{1. O Script de Simulação (.cir)}
O segredo aqui é parametrizar o período em função de uma frequência variável \texttt{f} e usar o comando \texttt{SWEEP} no \texttt{.tran}.

\begin{lstlisting}[language=pspice, caption={Script de Varredura de Frequência para Contador}]
* 
* 1. INCLUDES
.include "meu_contador.pex.netlist"
.include "/local/tools/dkit/ams_3.70_mgc/eldo/c35/modeloMOD"
.defmod pmos4 modp
.defmod nmos4 modn

* 2. ALIMENTACAO
.Param VDD_VAL = 3.3V
Vdd VDD 0 DC VDD_VAL
Vss VSS 0 0

* 3. ESTIMULO DE CLOCK (VARIAVEL COM 'f')
* Definimos uma frequencia inicial dummy (sera varrida)
.Param f = 10MEG
.Param period = '1/f'

* Configura bordas rapidas (proporcionais ao periodo para nao falhar em alta freq)
.Param tr = '0.05 * period'
.Param tf = '0.05 * period'
.Param pw = 'period/2 - (tr+tf)/2'

* Clock aplicado na entrada CLK
Vclk CLK 0 PULSE(0 VDD_VAL 0 tr tf pw period)

* Reset (Necessario para contadores assincronos/sincronos iniciarem conhecidos)
* Pulso inicial de Reset e depois fica em 0
Vrst RST 0 PULSE(0 VDD_VAL 0 1n 1n 5n 1000u)

* 4. MEDICOES AUTOMATICAS (.MEAS)
* Objetivo: Medir Periodo de Entrada (T_in) e Saida (T_out)

* --- Medir T_IN (Periodo do Clock) ---
* Mede entre a 10a e 11a borda de subida (estabilidade)
.meas tran T_IN trig v(CLK) val='VDD_VAL/2' rise=10 targ v(CLK) val='VDD_VAL/2' rise=11

* --- Medir T_OUT (Periodo da Saida Q_final) ---
* O periodo da saida eh MUITO maior. Precisamos pegar uma oscilacao completa.
* CUIDADO: Se o contador divide por 16, a saida so sobe na borda 8 e 24 do clock.
* Usamos rise=2 e rise=3 para garantir que o reset ja passou.
.meas tran T_OUT trig v(SAIDA_Q3) val='VDD_VAL/2' rise=2 targ v(SAIDA_Q3) val='VDD_VAL/2' rise=3

* --- Calcular a Relacao (M = Tout / Tin = Fin / Fout) ---
.meas tran RELACAO param='T_OUT/T_IN'

* 5. SIMULACAO COM SWEEP DE FREQUENCIA
* Varredura: 10 MHz ate 2 GHz (ajuste conforme necessidade)
* O tempo total de simulacao ('100*period') deve ser suficiente para 
* haver pelo menos 3 ou 4 oscilacoes completas da SAIDA (que eh lenta).
* Se o contador for de 4 bits (div 16), precisamos de min 50 clocks.

.tran 0 '100*period' SWEEP f DEC 10 10MEG 2G
\end{lstlisting}

\textbf{2. Análise no EZWave}

\begin{enumerate}
    \item \textbf{Rodar:} Execute o ELDO.
    \item \textbf{Tabela de Resultados:} Abra \textbf{View $\rightarrow$ Measurement Results}.
    \item \textbf{Plotar:} Arraste a medição \texttt{RELACAO} para o gráfico.
    \item \textbf{Interpretação Visual:}
    \begin{itemize}
        \item O eixo X será a Frequência ($f$).
        \item O gráfico deve mostrar uma linha reta horizontal perfeita no valor da divisão (ex: 16.0).
        \item Em uma certa frequência alta, a linha começará a cair, oscilar ou desaparecer.
    \end{itemize}
    \item \textbf{Identificar $F_{max}$:} O ponto exato onde a linha deixa de ser constante (sai do patamar de 16.0) é a frequência máxima de operação.
\end{enumerate}



\textbf{3. Dicas Críticas para a Prova}
\begin{itemize}
    \item \textbf{Tempo de Simulação:} Se o gráfico da relação aparecer "zerado" ou vazio, aumente o tempo final do `.tran`. Um contador divide a frequência. Se você simular pouco tempo, o sinal de saída (que é lento) não terá completado um ciclo inteiro ($T_{out}$) para ser medido pelo `.meas`. Use pelo menos \texttt{'100*period'}.
    \item \textbf{Nome da Saída:} Verifique no netlist qual é o nome do bit mais significativo (MSB) do contador (ex: Q3, Q4, OUT). É nele que você mede o $T_{out}$.
    \item \textbf{Resposta Escrita:} "A simulação mostra que a relação $F_{in}/F_{out}$ mantém-se constante em 16 (para 4 bits) até a frequência de $X$ GHz. Após este ponto, as violações de tempo de propagação impedem a comutação correta dos flip-flops."
\end{itemize}

\newpage

\subsubsection[\textcolor{red}{Prova 2022a Q3: Monte Carlo de Timing (Resolução Completa)}]{\textcolor{red}{Prova 2022a Q3: Monte Carlo de Timing (Resolução Completa)}}

\textbf{Objetivo:} Realizar uma análise estatística para determinar a variação esperada nos tempos de propagação do circuito devido a imperfeições do processo de fabricação (mismatch e variações globais).

\textbf{Teoria Rápida:}
Em vez de simular um transistor "ideal", o Monte Carlo roda a simulação $N$ vezes (aqui, 75), sorteando aleatoriamente parâmetros como $V_{th}$ (tensão de limiar) e $\mu$ (mobilidade) baseados em curvas de distribuição reais da fábrica.
\begin{itemize}
    \item \textbf{Resultado:} Um histograma (distribuição Gaussiana).
    \item \textbf{Métrica:} Devemos anotar a \textbf{Média} ($\mu$) e o \textbf{Desvio Padrão} ($\sigma$). Um $\sigma$ alto indica um circuito pouco robusto.
\end{itemize}

\textbf{1. O Script de Simulação (.cir)}
Atenção especial à inclusão das bibliotecas de estatística (`.lib ... mc`) e ao comando `.MC`.

\begin{lstlisting}[language=pspice, caption={Script de Monte Carlo (75 iterações)}]
* 
* 1. INCLUDES E BIBLIOTECAS ESTATISTICAS (CRITICO)
* Para MC, nao usamos os modelos simples (.mod). 
* Precisamos do arquivo de estatisticas (wc53.lib ou similar com flag 'mc')
.INCLUDE /local/tools/dkit/ams_3.70_mgc/eldo/c35/profile.opt
.LIB /local/tools/dkit/ams_3.70_mgc/eldo/c35/wc53.lib mc

* Seu netlist extraido (PEX)
.include "minha_celula.pex.netlist"

* Mapeamento (as vezes necessario dependendo de como o PEX gerou)
.defmod pmos4 modp
.defmod nmos4 modn

* 2. PARAMETROS FIXOS (Do Enunciado)
.Param VDD_VAL = 2.8V
.Param C_LOAD  = 25fF

* Fontes
Vdd VDD 0 DC VDD_VAL
Vss VSS 0 0

* 3. CARGA FIXA
* Diferente do Sweep, aqui o CL eh fixo.
CL OUT 0 C_LOAD

* 4. ESTIMULO
.Param period = 10n
.Param tr = 0.2n
Va IN 0 PULSE(0 VDD_VAL 0 tr tr 'period/2' period)

* 5. CONFIGURACAO DO MONTE CARLO (.MC)
* Sintaxe: .MC <num_runs> [opcoes] <tipo_variacao>
* NBBINS=20 define a resolucao do histograma (barras).
* 'mismatch': Variacao entre transistores do mesmo chip.
* 'process': Variacao global de wafer para wafer.
.MC 75 NBBINS=20 mismatch process

* 6. MEDICOES DE TIMING
* O enunciado pede "tempos de subida e descida". 
* Interpretacao 1: Atraso de Propagacao (tpLH, tpHL) - O mais comum para "timing".
.meas tran tpLH trig V(IN) val='VDD_VAL/2' fall=1 targ V(OUT) val='VDD_VAL/2' rise=1
.meas tran tpHL trig V(IN) val='VDD_VAL/2' rise=1 targ V(OUT) val='VDD_VAL/2' fall=1

* Interpretacao 2: Tempo de Transicao (Rise/Fall Time 10-90%) - Caso o prof. peca a inclinacao.
.meas tran t_rise trig V(OUT) val='VDD_VAL*0.1' rise=1 targ V(OUT) val='VDD_VAL*0.9' rise=1
.meas tran t_fall trig V(OUT) val='VDD_VAL*0.9' fall=1 targ V(OUT) val='VDD_VAL*0.1' fall=1

* 7. ANALISE TRANSIENTE
* O tempo deve ser suficiente para capturar a primeira transicao completa.
.tran 0 20n 0 0.01n
\end{lstlisting}

\textbf{2. Procedimento de Análise no EZWave}

Diferente das outras simulações, você \textbf{não deve} analisar as formas de onda sobrepostas ("nuvem de linhas"), pois é confuso. Você deve gerar o histograma.

\begin{enumerate}
    \item \textbf{Rodar:} Execute o ELDO. Note que demorará 75x mais que uma simulação comum.
    \item \textbf{Abrir EZWave:} Carregue o `.wdb`.
    \item \textbf{Gerar Histograma:}
    \begin{itemize}
        \item No menu superior, vá em \textbf{Plot $\rightarrow$ Monte Carlo $\rightarrow$ Histogram} (ou procure o ícone de gráfico de barras).
        \item Uma janela se abrirá listando suas medições (`tpLH`, `tpHL`, etc.).
        \item Selecione, por exemplo, `tpHL` e clique em OK.
    \end{itemize}
    \item \textbf{Leitura dos Dados:}
    \begin{itemize}
        \item O gráfico mostrará uma distribuição (formato de sino).
        \item Olhe a legenda ou as propriedades do gráfico para encontrar:
        \begin{itemize}
            \item \textbf{Mean (Média):} O atraso típico esperado.
            \item \textbf{Sigma (Desvio Padrão):} A variabilidade.
        \end{itemize}
    \end{itemize}
\end{enumerate}



\textbf{3. O que escrever na folha da prova?}
\begin{itemize}
    \item \textbf{Resultados:} "A simulação de Monte Carlo com 75 pontos indicou um tempo de propagação médio ($t_{pHL}$) de $X$ ps com um desvio padrão ($\sigma$) de $Y$ ps."
    \item \textbf{Análise (Opcional):} "Considerando uma variação de $3\sigma$ (que cobre 99.7\% dos casos), o atraso máximo esperado seria $\mu + 3\sigma$."
    \item \textbf{Gráfico:} Desenhe um esboço simplificado do histograma mostrando onde está a média.
\end{itemize}

\newpage

\subsubsection[\textcolor{red}{Prova 2016b Q4: Cálculo Teórico de $F_{max}$}]{\textcolor{red}{Prova 2016b Q4: Cálculo Teórico de $F_{max}$}}

\textbf{Enunciado:} Suponha que o atraso de cada porta lógica seja $0.2\,\text{ns}$. O atraso ($T_D$), o setup ($T_{set}$) e o holding time ($T_{holding}$) dos D-FFs estão indicados na Figura 1 ($T_D=0.4\,\text{ns}, T_{set}=0.1\,\text{ns}$). Qual será o máximo clock permitido?



\textbf{Solução Detalhada:}

Para encontrar a frequência máxima ($F_{max}$), devemos identificar o **Caminho Crítico**, que é o trajeto lógico mais longo entre a saída ($Q$) de um Flip-Flop e a entrada ($D$) do próximo.

\textbf{1. Análise dos Caminhos Combinacionais ($T_{logic}$):}
Observando a imagem, o sinal pode percorrer dois caminhos distintos:
\begin{itemize}
    \item \textbf{Caminho Superior:} Passa por 2 portas XOR.
    \[ T_{sup} = 2 \times 0.2\,\text{ns} = 0.4\,\text{ns} \]
    \item \textbf{Caminho Inferior:} Passa por 2 portas AND e depois pela porta XOR final (totalizando 3 portas em série).
    \[ T_{inf} = 3 \times 0.2\,\text{ns} = \mathbf{0.6\,\text{ns}} \]
\end{itemize}
O atraso lógico a ser considerado é sempre o maior: $\mathbf{T_{logic} = 0.6\,\text{ns}}$.

\textbf{2. Cálculo do Período Mínimo ($T_{min}$):}
O período do clock deve ser suficiente para cobrir o atraso de propagação do FF inicial, o tempo da lógica e o tempo de preparação (setup) do FF final.
\[ T_{min} \ge T_{clk-q} + T_{logic} + T_{setup} \]

Substituindo os valores dados na figura:
\[ T_{min} \ge 0.4\,\text{ns} + 0.6\,\text{ns} + 0.1\,\text{ns} \]
\[ T_{min} = 1.1\,\text{ns} \]

\textbf{3. Cálculo da Frequência Máxima ($F_{max}$):}
\[ F_{max} = \frac{1}{T_{min}} = \frac{1}{1.1 \times 10^{-9}\,\text{s}} \]
\[ F_{max} \approx 909.09 \times 10^6\,\text{Hz} \]

\[ \boxed{F_{max} \approx 909\,\text{MHz}} \]

\textbf{Nota:} O parâmetro $T_{holding} = 0.08\,\text{ns}$ e $0.075\,\text{ns}$ serve para verificar violações de tempo de retenção (corrida de sinais), mas \textbf{não influencia} o cálculo da frequência máxima de operação.

\newpage

\section{Projeto e Análise de Circuitos Analógicos}

Esta seção fundamenta as escolhas de topologia e dimensionamento para as fontes de corrente exigidas nas provas (2016b, 2022a, 2024).

\subsection{Parte A: Fundamentos Teóricos de Dispositivos}

\subsubsection{Transcondutância ($g_m$) e Regiões de Operação}
A transcondutância ($g_m$) é a figura de mérito fundamental no projeto analógico, representando a variação da corrente de dreno em resposta à tensão de gate ($g_m = \frac{\partial I_D}{\partial V_{GS}}$). O comportamento muda drasticamente dependendo da região de inversão.

\textbf{1. Forte Inversão (Strong Inversion)}
Ocorre quando $V_{GS} > V_{TH}$. O transistor obedece à lei quadrática:
\begin{equation}
I_D = \frac{1}{2} \mu C_{ox} \frac{W}{L} (V_{GS} - V_{TH})^2 (1 + \lambda V_{DS})
\end{equation}
Derivando em relação a $V_{GS}$, obtemos:
\begin{equation}
g_m = \mu C_{ox} \frac{W}{L} (V_{GS} - V_{TH}) = \sqrt{2 \mu C_{ox} \frac{W}{L} I_D} = \frac{2I_D}{V_{OV}}
\end{equation}
\textbf{Consequência de Projeto:} Em forte inversão, para aumentar $g_m$, deve-se aumentar a largura $W$ ou a corrente $I_D$. É a região ideal para espelhos de saída onde se deseja minimizar a área e maximizar a excursão de tensão (swing), como exigido nos PMOS de saída da \textbf{Prova 2024}.

\textbf{2. Fraca Inversão (Weak Inversion / Sub-threshold)}
Ocorre quando $V_{GS} < V_{TH}$. A corrente é dominada por difusão (similar a um BJT) e segue uma relação exponencial:
\begin{equation}
I_D \approx I_{D0} \cdot e^{\frac{V_{GS}}{n U_T}} \cdot \left( 1 - e^{\frac{-V_{DS}}{U_T}} \right)
\end{equation}
Onde $U_T = \frac{kT}{q} \approx 26mV$ (a 300K) e $n$ é o fator de inclinação de sub-limiar ($n \approx 1.2$ a $1.5$).
Derivando $I_D$ em relação a $V_{GS}$:
\begin{equation}
g_m = \frac{\partial}{\partial V_{GS}} \left( I_{D0} e^{\frac{V_{GS}}{n U_T}} \right) = \frac{I_D}{n U_T}
\end{equation}
\textbf{Consequência Crítica para Fontes PTAT:}
\begin{itemize}
    \item A transcondutância \textbf{independe da geometria ($W/L$)}.
    \item A relação $g_m/I_D$ é máxima (máxima eficiência de corrente).
    \item A diferença de tensão $V_{GS}$ entre dois transistores com densidades de corrente diferentes ($J = I_D/W$) é linear com a temperatura ($T$), base para o princípio PTAT.
\end{itemize}



\subsubsection{Espelhos de Corrente e Impedância de Saída}
Uma fonte de corrente ideal possui impedância de saída infinita ($I_{out}$ constante independente de $V_{out}$).

\textbf{Espelho Simples:}
A impedância de saída é dominada pela modulação de comprimento de canal ($\lambda$):
\begin{equation}
r_o = \frac{1}{\lambda I_D} \approx \frac{V_A L}{I_D}
\end{equation}
Para aumentar $r_o$, devemos usar transistores com $L$ longo (ex: $L > 1\mu m$), como feito no dimensionamento do espelho de saída.

\textbf{Espelho Cascode / Wilson:}
Utiliza realimentação para aumentar a impedância efetiva.
\begin{equation}
R_{out} \approx g_m r_o^2
\end{equation}
O ganho intrínseco ($g_m r_o$) multiplica a impedância, tornando a fonte muito mais estável frente a variações de $V_{DD}$ (melhor \textit{Line Regulation}).

\newpage

\subsection{Parte B: Teoria e Projeto da Fonte PTAT}
A fonte de corrente projetada nas provas (2016b, 2024) utiliza a topologia de auto-polarização (\textit{Beta-multiplier} ou similar) operando em fraca inversão.

\subsubsection{Dedução da Fórmula de Projeto}
O circuito força uma diferença de tensão $\Delta V_{GS}$ sobre um resistor $R$.

1. Considere dois transistores NMOS $M_1$ e $M_2$ operando em fraca inversão, onde $M_2$ é $K$ vezes mais largo que $M_1$ (ou o espelho superior força correntes desiguais).

2. As equações de tensão Gate-Source são:
$$ V_{GS1} = n U_T \ln\left(\frac{I_{D1}}{I_{D0} (W/L)_1}\right) $$
$$ V_{GS2} = n U_T \ln\left(\frac{I_{D2}}{I_{D0} (W/L)_2}\right) $$

3. Pela malha do circuito (KVL), a tensão sobre o resistor $R$ é a diferença entre os $V_{GS}$:
$$ V_R = V_{GS1} - V_{GS2} = \Delta V_{GS} $$

4. Substituindo as equações logarítmicas e assumindo $I_{D1} = I_{D2} = I_{REF}$ (copiados pelo espelho PMOS):
$$ V_R = n U_T \ln\left( \frac{I_{REF}}{I_{D0} (W/L)_1} \right) - n U_T \ln\left( \frac{I_{REF}}{I_{D0} (W/L)_2} \right) $$
$$ V_R = n U_T \ln\left( \frac{(W/L)_2}{(W/L)_1} \right) $$

5. Definindo o fator de multiplicidade $M = \frac{(W/L)_2}{(W/L)_1}$:
$$ I_{REF} \cdot R = n U_T \ln(M) $$

\textbf{Equação Final de Projeto (PTAT):}
\begin{equation}
I_{REF} = \frac{n \cdot U_T \cdot \ln(M)}{R}
\end{equation}

\textbf{Por que é PTAT?}
Como $U_T = \frac{kT}{q}$, a corrente $I_{REF}$ é diretamente proporcional à Temperatura absoluta $T$.
$$ I_{REF} \propto T $$
Isso é essencial para compensar o coeficiente negativo ($V_{BE}$) em referências de Bandgap.

\subsubsection{Análise de Simetria do Esquemático ($M4_a$ e $M4_b$)}

Observando o diagrama esquemático (Figura \ref{fig:fonte_esquematico}) e a Tabela \ref{tab:dimensoes_q11_final}, nota-se a divisão do transistor de carga PMOS em duas unidades ($M4_a$ e $M4_b$ ou M41/M42 na imagem). Esta escolha de projeto é fundamentada em três pilares críticos para circuitos analógicos de precisão:

\begin{enumerate}
    \item \textbf{Garantia de Espelhamento (Matching 1:1):}
    A dedução matemática da fonte PTAT assume que a corrente no ramo esquerdo é \textbf{exatamente igual} à corrente no ramo direito ($I_{D1} = I_{D2}$). Qualquer erro nessa cópia introduz um erro logarítmico na tensão de saída. Ao desenhar o layout, dividir o transistor $M4$ em duas metades idênticas ($M4_a$ e $M4_b$) permite o uso de técnicas como \textit{Common Centroid} (Centro Comum), onde os transistores são interdigitados para cancelar gradientes térmicos e de processo.

    \item \textbf{Comprimento de Canal Longo ($L=15\mu m$):}
    Nota-se que os PMOS possuem $L=15\mu m$. Um canal longo aumenta drasticamente a resistência de saída ($r_o \propto L$), reduzindo o efeito de modulação de comprimento de canal ($\lambda$). Isso garante que $I_{D}$ permaneça constante mesmo se houver uma pequena diferença de $V_{DS}$ entre os ramos, melhorando o PSRR e a regulação de linha.

    \item \textbf{Simetria Visual e Física:}
    A estrutura simétrica mostrada no esquemático (Ramo Esquerdo vs. Ramo Direito) facilita a verificação visual de que as cargas são idênticas, o que é um pré-requisito para o funcionamento correto do \textit{Beta-multiplier}.
\end{enumerate}

\begin{table}[H]
    \centering
    \caption{Dimensões finais dos transistores otimizados para o projeto. Observe a simetria entre os blocos funcionais.}
    \label{tab:dimensoes_q11_final}
    % DICA: 'l' alinha a esquerda, 'c' centraliza, 'r' alinha a direita.
    % Usamos booktabs (\toprule, \midrule) para visual profissional.
    \begin{tabular}{l c c c l} 
        \toprule
        \textbf{Bloco Funcional} & \textbf{Transistor} & \textbf{W ($\mu$m)} & \textbf{L ($\mu$m)} & \textbf{Função no Circuito} \\ 
        \midrule
        
        % --- BLOCO NMOS (Nucleo) ---
        \multirow{4}{*}{\textbf{Núcleo NMOS}} 
          & M1 & 95,0 & 1,0 & Gerador $\Delta V_{GS}$ (Lado esquerdo) \\
          & M2 & 95,0 & 1,0 & Gerador $\Delta V_{GS}$ (Lado direito, $M \times W$) \\
          & M6 & 95,0 & 1,0 & Cascode (Aumenta $R_{out}$) \\
          & M7 & 95,0 & 1,0 & Cascode (Aumenta $R_{out}$) \\ 
        
        \midrule
        
        % --- BLOCO PMOS (Carga/Espelho) ---
        \multirow{4}{*}{\textbf{Espelho PMOS}} 
          & M3 & 9,0 & 15,0 & Parte do Espelho Wilson/Cascode \\
          & M4$_a$ & 9,0 & 15,0 & Carga Ativa (Paralelo/Simetria) \\
          & M4$_b$ & 9,0 & 15,0 & Carga Ativa (Paralelo/Simetria) \\
          & M5 & 9,0 & 15,0 & Espelho de Saída \\ 
          
        \bottomrule
    \end{tabular}
    
    % Nota de rodape da tabela para explicar o W/L efetivo
    \vspace{0.2cm}
    \small{\textit{Nota: A razão de aspecto efetiva para os NMOS é $W/L = 95$, garantindo operação em fraca inversão para correntes de $\mu A$.}}
\end{table}

\subsubsection{Script de Dimensionamento (Cálculo dos Componentes)}
O script abaixo aplica as equações de inversão fraca (para o núcleo) e forte (para a saída) para determinar $W/L$.

\begin{lstlisting}[language=python, caption={Script de Dimensionamento PTAT (Baseado na Teoria)}]
import numpy as np
import pandas as pd # Usaremos pandas apenas para visualização bonita, se não tiver, o print normal resolve

# --- FUNÇÃO AUXILIAR: Ajuste para Grade de Layout ---
def round_grid(val, grid=0.05):
    """Arredonda para o múltiplo mais próximo da grade de fabricação (ex: 0.05um)"""
    return np.round(val / grid) * grid

# --- DADOS DE ENTRADA (QUESTÃO 9) ---
Is = 3.00e-6       # 3.0 uA
Vdd = 2.6          # Tensão de alimentação
Vov_pmos = 0.5     # Headroom exigido (Vdd - Vout_max)

# --- PARÂMETROS FÍSICOS (Do Technology File) ---
Vt = 0.026
n = 1.2
un = 475.8; Coxn = 4.55e-7  # NMOS
up = 148.2; Coxp = 4.55e-7  # PMOS

# --- CÁLCULOS DE ALVO (TARGETS) ---

# 1. PMOS (Saída): Forte Inversão, Saturado
# I = 1/2 * up * Coxp * (W/L) * Vov^2  --> Isolando W/L:
WL_pmos_target = (2 * Is) / (up * Coxp * (Vov_pmos**2))

# 2. NMOS (Núcleo): Fraca Inversão
# Para garantir fraca inversão, W/L tem que ser GRANDE para a densidade de corrente ser baixa.
# Fator de segurança = 5x longe da inversão moderada
WL_nmos_min_target = 5.0 * (Is / (un * Coxn * (n * Vt)**2))

print(f"--- ALVOS DE PROJETO ---")
print(f"Razão W/L PMOS necessária: {WL_pmos_target:.4f} (Para Vov=0.5V)")
print(f"Razão W/L NMOS mínima:     {WL_nmos_min_target:.4f} (Para garantir Fraca Inversão)")
print("-" * 60)

# --- GERADOR DE CENÁRIOS INTELIGENTES ---

# Lista de L (Comprimentos) sugeridos para teste. 
# Em analógico, evitamos L mínimo. Usamos L maiores para ganhar impedância de saída (ro).
L_choices = [1.0, 2.0, 4.0, 5.0] # em micrometros (um)
M_choices = [4, 8] # Fatores de multiplicação do espelho (Inteiros pares facilitam centroide)

scenarios = []

for M in M_choices:
    # Resistor depende apenas de M
    R = (n * Vt * np.log(M)) / Is
    R_kohm = R / 1000.0
    
    for L_val in L_choices:
        L_microns = L_val
        
        # --- Dimensionamento PMOS ---
        # W = Razão_Alvo * L
        W_pmos_raw = WL_pmos_target * L_microns
        W_pmos_final = round_grid(W_pmos_raw)
        
        # Check de sanidade: Se W ficar muito pequeno (< 0.22u), descartar ou avisar
        if W_pmos_final < 0.22: 
            note_p = "W muito pequeno (Viol. DRC)"
        else:
            note_p = "OK"

        # --- Dimensionamento NMOS (Espelho 1:M) ---
        # W_nmos = Razão_Minima * L
        # O transistor M1 tem tamanho (W/L). O transistor M2 tem tamanho M*(W/L).
        # Aqui calculamos o tamanho unitário base.
        W_nmos_raw = WL_nmos_min_target * L_microns
        W_nmos_final = round_grid(W_nmos_raw)
        
        # Adicionando ao relatório
        scenarios.append({
            "M (Fator)": int(M),
            "R (kΩ)": round(R_kohm, 2),
            "L (µm)": L_microns,
            "PMOS W (µm)": W_pmos_final,
            "NMOS W (µm)": W_nmos_final,
            "Status PMOS": note_p
        })

# --- EXIBIÇÃO DOS RELATÓRIOS ---

df = pd.DataFrame(scenarios)

print("\n### RELATÓRIO 1: SELEÇÃO DE VALORES ###")
print("Legenda: L é o comprimento do canal. W é a largura total.")
print("Nota: O NMOS deve operar em fraca inversão, por isso o W é grande.")
print("-" * 80)
# Filtrando apenas colunas relevantes e imprimindo
print(df.to_string(index=False))

print("\n\n### RECOMENDAÇÃO 'PRONTA PARA A PROVA' ###")

# Lógica de seleção automática do "Melhor" cenário
# Critério: M=8 (Resistor maior, menos erro de processo) e L=2.0um (Bom matching sem ser gigante)
best_choice = df[(df["M (Fator)"] == 8) & (df["L (µm)"] == 2.0)].iloc[0]

print(f"Topologia: Beta Multiplier (NMOS em Fraca Inversão, PMOS em Forte Inversão)")
print(f"Resistor Escolhido: {best_choice['R (kΩ)']} kΩ (Resultante de M={int(best_choice['M (Fator)'])} e Is=3uA)")
print(f"Transistor PMOS (Saída): W = {best_choice['PMOS W (µm)']} µm / L = {best_choice['L (µm)']} µm")
print(f"   -> Justificativa: Dimensionado para Vov = 0.5V (atendendo Vdd-0.5V).")
print(f"Transistor NMOS (Núcleo): W = {best_choice['NMOS W (µm)']} µm / L = {best_choice['L (µm)']} µm")
print(f"   -> Justificativa: W/L > {WL_nmos_min_target:.1f} para garantir operação em Fraca Inversão.")
print(f"   -> Obs: O transistor M2 será composto por {int(best_choice['M (Fator)'])} dedos ou unidades deste tamanho.")
\end{lstlisting}

\begin{lstlisting}[language=c, caption={Script de Dimensionamento PTAT (Baseado na Teoria)}]
--- ALVOS DE PROJETO ---
Razão W/L PMOS necessária: 0.3559 (Para Vov=0.5V)
Razão W/L NMOS mínima:     71.1781 (Para garantir Fraca Inversão)
------------------------------------------------------------

### RELATÓRIO 1: SELEÇÃO DE VALORES ###
Legenda: L é o comprimento do canal. W é a largura total.
Nota: O NMOS deve operar em fraca inversão, por isso o W é grande.
--------------------------------------------------------------------------------
 M (Fator)  R (kΩ)  L (µm)  PMOS W (µm)  NMOS W (µm) Status PMOS
         4   14.42     1.0         0.35        71.20          OK
         4   14.42     2.0         0.70       142.35          OK
         4   14.42     4.0         1.40       284.70          OK
         4   14.42     5.0         1.80       355.90          OK
         8   21.63     1.0         0.35        71.20          OK
         8   21.63     2.0         0.70       142.35          OK
         8   21.63     4.0         1.40       284.70          OK
         8   21.63     5.0         1.80       355.90          OK


### RECOMENDAÇÃO 'PRONTA PARA A PROVA' ###
Topologia: Beta Multiplier (NMOS em Fraca Inversão, PMOS em Forte Inversão)
Resistor Escolhido: 21.63 kΩ (Resultante de M=8 e Is=3uA)
Transistor PMOS (Saída): W = 0.7000000000000001 µm / L = 2.0 µm
   -> Justificativa: Dimensionado para Vov = 0.5V (atendendo Vdd-0.5V).
Transistor NMOS (Núcleo): W = 142.35 µm / L = 2.0 µm
   -> Justificativa: W/L > 71.2 para garantir operação em Fraca Inversão.
   -> Obs: O transistor M2 será composto por 8 dedos ou unidades deste tamanho.


** Process exited - Return Code: 0 **

\end{lstlisting}

\begin{figure}[H]
    \centering
        \caption{Esquemático da fonte PTAT. O núcleo M1-M2 opera em fraca inversão para gerar $\Delta V_{GS}$.}
\end{figure}

\begin{figure}[H]
    \centering
    \includegraphics[width=0.9\textwidth]{fonte_esquematico.png}
    \caption{Esquemático do circuito gerador de corrente reprojetado, com espelho de corrente de Wilson.}
    \label{fig:fonte_esquematico}
\end{figure}


\newpage

\subsection{Verificação e Simulação}
Após desenhar o esquemático com os valores calculados, é obrigatório validar o circuito com simulações DC, Transiente e AC.

\subsubsection{Verificação DC: Estabilidade (Line Regulation)}
Verifica se a corrente de saída permanece constante (estável) mesmo com variações na tensão de alimentação ($V_{DD}$).

\textbf{O que esperar:} A corrente deve subir rapidamente a partir de $\approx 1V$ e estabilizar num patamar plano (ex: $3\mu A$) até $V_{DD}$ máximo.

\begin{lstlisting}[language=pspice, caption={Script DC para Regulação de Linha}]
* 1. INCLUDES
.include "fonte_corrente.pex.netlist"
.include "/local/tools/dkit/ams_3.70_mgc/eldo/c35/modeloMOD"

* 2. FONTES
vdd_s VDD 0 3.3V   * Valor nominal
vss_s VSS 0 0V
vout_s OUT 0 0V    * Fonte de 0V na saida para medir corrente

* 3. SIMULACAO (SWEEP VDD)
* Varre VDD de 0V ate 5V
.DC vdd_s 0 5 0.01

.probe DC IS(vout_s)
\end{lstlisting}

\textbf{Análise no EZWave:}
\begin{enumerate}
    \item Plote \texttt{IS(vout\_s)}.
    \item Verifique se existe uma região "plana" (constante) em torno da tensão de operação ($3.3V$ ou $3.0V$).
    \item Se a corrente continuar subindo linearmente (efeito Early), aumente o comprimento ($L$) dos transistores de saída ou use uma topologia cascode.
\end{enumerate}



\subsubsection{Verificação de Temperatura (PTAT)}
Verifica se a corrente aumenta linearmente com a temperatura, requisito fundamental para compor referências Bandgap.

\begin{lstlisting}[language=pspice, caption={Script DC para Temperatura}]
* Varre Temperatura de -10C a 100C
.DC TEMP -10 100 1

.probe DC IS(vout_s)
\end{lstlisting}

\textbf{Análise no EZWave:}
\begin{enumerate}
    \item Plote \texttt{IS(vout\_s)}.
    \item O gráfico deve ser uma reta com inclinação positiva.
    \item Se for uma curva parabólica ou tiver inclinação negativa, verifique se os transistores NMOS estão realmente em fraca inversão (aumente $W$ se necessário).
\end{enumerate}

\newpage

\subsubsection{Verificação de Start-Up (Transiente)}
Fontes de auto-polarização possuem um ponto de equilíbrio estável onde $I = 0A$. O teste de start-up força o circuito para esse estado "morto" e verifica se ele consegue "acordar" sozinho.

\textbf{Código Robusto:}
Usamos o comando \texttt{.IC} (Initial Condition) para travar os nós internos em tensões que desligam os transistores.

\begin{lstlisting}[language=pspice, caption={Script Transiente de Start-Up}]
* 1. CONDICAO INICIAL (FORCAR OFF)
* Se M3/M4 sao PMOS, Gate=VDD mantem eles desligados.
* Descubra o nome do no do gate no seu netlist (ex: N_Gate_PMOS)
.ic V(N_Gate_PMOS)=3.3V

* 2. SIMULACAO TRANSIENTE
* Simula por 100us para dar tempo de estabilizar
.tran 1u 100u 0 1u

.probe tran IS(vout_s)
\end{lstlisting}

\textbf{Análise no EZWave:}
\begin{enumerate}
    \item Plote \texttt{IS(vout\_s)}.
    \item \textbf{Cenário Correto:} A corrente começa em 0A, dá um salto abrupto e estabiliza no valor nominal ($3\mu A$) em poucos microsegundos.
    \item \textbf{Cenário de Falha:} A corrente permanece em 0A indefinidamente.
    \item \textit{Correção:} Se falhar, adicione um circuito de start-up (transistor diodo ou divisor resistivo) para injetar uma pequena corrente inicial.
\end{enumerate}



\subsubsection{Análise de Frequência (PSRR)}
O PSRR (\textit{Power Supply Rejection Ratio}) mede o quanto ruídos na linha $V_{DD}$ "vazam" para a saída de corrente.

\begin{lstlisting}[language=pspice, caption={Script AC para PSRR}]
* 1. FONTE VDD COM RUIDO AC
* DC=3.3V (ponto de operacao), AC=1V (estimulo para analise)
Vdd VDD 0 DC 3.3V AC 1.0

* 2. SIMULACAO AC
* Varredura logaritmica de 1kHz a 100MHz
.AC DEC 10 1K 100MEG

* 3. MEDICAO
* Como Vin_AC = 1V, Iout_AC ja representa o ganho
.probe AC I(vout_s)
\end{lstlisting}

\textbf{Análise no EZWave:}
\begin{enumerate}
    \item Plote a magnitude de \texttt{I(vout\_s)}.
    \item Converta para dB: Clique direito na escala Y $\rightarrow$ Logarithmic ou use a calculadora de onda \texttt{20*log10(...)}.
    \item Valores baixos (ex: -60dB ou nA) indicam boa rejeição. O PSRR costuma piorar (subir) em altas frequências.
\end{enumerate}

\newpage

\subsection[\textcolor{red}{Exemplos Práticos de Provas}]{\textcolor{red}{Exemplos Práticos de Provas}}

\subsubsection[\textcolor{red}{Prova 2024 Q8/Q9: Fonte de $3\mu A$ (Projeto e Validação)}]{\textcolor{red}{Prova 2024 Q8/Q9: Fonte de $3\mu A$ (Projeto e Validação)}}

\textbf{Objetivo:} Projetar uma Fonte de Corrente Autorreferenciada (Topologia Beta Multiplier) que forneça $3\mu A$ estáveis com $V_{DD}=3.3V$.
\textbf{Requisitos:} Transistores PMOS em Forte Inversão (para economizar área/headroom) e validação de Start-up.

\textbf{1. Dimensionamento (Cálculo dos Componentes)}

Antes de ir para o Mentor Graphics, você deve definir $W/L$ e $R$. Use o script Python (Seção 7.1.2) com os seguintes ajustes:

\begin{itemize}
    \item \textbf{Corrente Alvo:} \texttt{Is = 3.0e-6}
    \item \textbf{Tensão:} \texttt{Vdd = 3.3}
    \item \textbf{Headroom PMOS:} Como pede Forte Inversão, assuma $V_{ov} \approx 0.2V$ a $0.3V$.
    \item \textbf{Multiplicador:} Escolha $M=4$ ou $M=8$ (facilitam o layout em centroide).
\end{itemize}

\textit{Exemplo de Resultado Típico (M=8):}
\begin{itemize}
    \item $R \approx 21.6 k\Omega$.
    \item $W/L_{NMOS} \approx 70$ (Grande, para garantir Fraca Inversão).
    \item $W/L_{PMOS} \approx 2$ a $5$ (Pequeno, para garantir Forte Inversão).
\end{itemize}



\textbf{2. O Script de Simulação Unificado (.cir)}

Este script realiza duas análises essenciais: **DC** (para verificar a estabilidade da corrente em relação ao VDD) e **Transiente** (para verificar se o circuito liga).

\begin{lstlisting}[language=pspice, caption={Script de Validação da Fonte de Corrente}]
* 
* 1. INCLUDES
* O netlist deve conter o circuito Beta Multiplier + Resistor
.include "minha_fonte.pex.netlist"
.include "/local/tools/dkit/ams_3.70_mgc/eldo/c35/modeloMOD"
.defmod pmos4 modp
.defmod nmos4 modn
* Modelo do resistor (se usar rpolyh)
.include "/local/tools/dkit/ams_3.70_mgc/eldo/c35/restm.mod"

* 2. FONTES
* VDD nominal e VSS
Vdd VDD 0 DC 3.3V
Vss VSS 0 0
* Fonte Dummy na saida para medir a corrente copiada (se houver espelho de saida)
Vout_meas OUT 0 1.65V

* 3. ANALISE 1: SWEEP DC (LINE REGULATION)
* Verifica se a corrente estabiliza em 3uA ao variar VDD
* Descomente a linha abaixo para rodar esta analise
* .DC Vdd 0 4 0.01

* 4. ANALISE 2: TRANSIENTE DE START-UP (CRITICO)
* O Beta Multiplier tem dois pontos de operacao: 3uA (Desejado) e 0A (Indesejado).
* Precisamos forcar o circuito para 0A no inicio para ver se ele "acorda".

* Condicao Inicial (.IC):
* Forcamos o Gate dos PMOS para 3.3V (Vgs=0 -> Desligado)
* Descubra o nome do no do gate no seu netlist (ex: N_Gate_PMOS_1)
.ic V(N_Gate_PMOS_1)=3.3V

* Simula por 50us para ver a estabilizacao
.tran 0 50u 0 10n

* 5. MEDICOES
.probe DC I(Vout_meas)
.probe tran I(Vout_meas) V(N_Gate_PMOS_1)
\end{lstlisting}

\textbf{3. Análise no EZWave e Circuito de Start-up}

\begin{enumerate}
    \item \textbf{Verificação DC (Line Regulation):}
    \begin{itemize}
        \item A curva de corrente deve subir e ficar "plana" em $3\mu A$ quando $V_{DD} > 1.5V$.
        \item Se a corrente continuar subindo (efeito Early), aumente o $L$ dos transistores de saída.
    \end{itemize}

    \item \textbf{Verificação de Start-up (Transiente):}
    \begin{itemize}
        \item Plote a corrente.
        \item \textbf{Cenário A (Sucesso):} A corrente começa em 0, dá um "salto" abrupto em poucos $\mu s$ e estabiliza em $3\mu A$. O circuito é robusto.
        \item \textbf{Cenário B (Falha - Ponto Morto):} A corrente permanece em $0A$ (ou picos de ruído femto-ampere) durante toda a simulação.
    \end{itemize}

    \item \textbf{Como corrigir a Falha (Cenário B):}
    \begin{itemize}
        \item Se o gráfico ficar em 0A, você deve desenhar um **Circuito de Start-up** no Mentor.
        \item \textbf{Solução Simples:} Adicione um transistor NMOS "vazando" corrente.
        \begin{itemize}
            \item Dreno: Ligado ao Gate dos NMOS (ou nó interno da fonte).
            \item Gate: Ligado ao VDD.
            \item Source: Ligado ao VSS.
            \item Dimensionamento: Use um $L$ muito grande (ex: $W=1\mu m, L=20\mu m$) para que ele funcione como um resistor de valor altíssimo, injetando apenas uma corrente minúscula ("leakage") para tirar o circuito do zero, mas sem afetar a precisão de $3\mu A$.
        \end{itemize}
    \end{itemize}
\end{enumerate}



\textbf{Dica de Prova:} Se a questão pedir para "mostrar a necessidade do start-up", rode a simulação com `.ic` \textbf{sem} o circuito de start-up primeiro, tire o print do gráfico em 0A (falha), e depois adicione o circuito e mostre o gráfico funcionando.

\newpage


\subsubsection[\textcolor{red}{Prova 2016b Q9/Q10: Fonte de $0.7\mu A$ (Baixa Potência e Start-up)}]{\textcolor{red}{Prova 2016b Q9/Q10: Fonte de $0.7\mu A$ (Baixa Potência e Start-up)}}

\textbf{Objetivo:} Projetar uma fonte de corrente de referência de apenas $0.7\mu A$ com tensão de alimentação reduzida ($V_{DD}=2.6V$).
\textbf{Desafios:} O valor alto do resistor necessário, a dependência com a temperatura (PTAT) e a tendência do circuito travar no estado desligado (Start-up).

\textbf{1. Dimensionamento e Cuidados Especiais}

\begin{itemize}
    \item \textbf{Resistor Elevado:} Para gerar uma corrente tão baixa ($0.7\mu A$), a lei de Ohm ($V=IR$) exige um resistor grande.
    \item \textbf{Cálculo Rápido:} Usando o script Python com $I_s = 0.7e-6$ e $M=8$, você encontrará um resistor $R \approx 80k\Omega$ a $100k\Omega$.
    \item \textbf{Dica de Layout:} No esquemático/layout, é obrigatório usar o resistor de alta resistividade (\texttt{rpolyh}) para economizar área, e dobrá-lo ("snake pattern").
\end{itemize}



\textbf{2. O Script de Simulação (Temperatura e Start-up)}

Este script está configurado para atender aos dois pedidos da prova: mostrar o comportamento com a temperatura e falhar propositalmente o start-up.

\begin{lstlisting}[language=pspice, caption={Script de Análise de Temperatura e Start-up (2016b)}]
* 
* 1. INCLUDES
.include "minha_fonte_07uA.pex.netlist"
.include "/local/tools/dkit/ams_3.70_mgc/eldo/c35/modeloMOD"
.defmod pmos4 modp
.defmod nmos4 modn
.include "/local/tools/dkit/ams_3.70_mgc/eldo/c35/restm.mod"

* 2. ALIMENTACAO (BAIXA TENSAO)
* Enunciado especifica 2.6V
.Param VDD_VAL = 2.6V
Vdd VDD 0 DC VDD_VAL
Vss VSS 0 0

* 3. ANALISE A: COMPORTAMENTO COM TEMPERATURA (Q9)
* O Beta Multiplier eh naturalmente PTAT (Proporcional a Temp).
* A corrente DEVE aumentar com a temperatura.
* Descomente a linha abaixo para gerar este grafico:
* .DC TEMP -20 100 1

* 4. ANALISE B: PROVA DE NECESSIDADE DE START-UP (Q10)
* Para provar que o circuito *precisa* de ajuda, vamos força-lo
* a ficar desligado no t=0. Se ele for robusto, ele acorda.
* Se ele for dependente, ele fica morto (0A).

* Condicao Inicial (.IC):
* Forca o Gate do PMOS para 2.6V (Vgs=0 -> PMOS OFF)
* E forca o Gate do NMOS para 0V (Vgs=0 -> NMOS OFF)
* Substitua pelos nomes reais dos nos do seu netlist
.ic V(N_Gate_PMOS)=2.6V V(N_Gate_NMOS)=0V

* Simula o transiente
.tran 0 100u 0 10n

* 5. MEDICOES
* Monitoramos a corrente de uma fonte dummy ou do proprio VDD
.probe DC I(Vdd)
.probe tran I(Vdd) V(N_Gate_PMOS)
\end{lstlisting}

\textbf{3. Passo a Passo da Análise (EZWave)}

\textbf{Parte A: Estabilidade com Temperatura (Questão 9)}
\begin{enumerate}
    \item Rode o \texttt{.DC TEMP}.
    \item Plote a corrente de saída.
    \item \textbf{Resultado Esperado:} Uma reta com inclinação positiva (a corrente sobe conforme esquenta).
    \item \textbf{Justificativa na Prova:} "O circuito apresenta comportamento PTAT (Proportional To Absolute Temperature), pois $I_{ref} \propto V_T/R$ e a tensão térmica $V_T = kT/q$ aumenta linearmente com T. Isso é desejável para compensar referências de tensão, mas mostra que a corrente não é constante com a temperatura."
\end{enumerate}

\textbf{Parte B: O "Show" do Start-up (Questão 10)}
A prova pede para \textit{"mostrar que há necessidade"}. Isso se faz em dois tempos:

\begin{enumerate}
    \item \textbf{Passo 1: A Falha (Sem Circuito de Start-up)}
    \begin{itemize}
        \item Desenhe a fonte básica (apenas o núcleo Beta Multiplier).
        \item Rode o script Transiente com o comando \texttt{.ic}.
        \item \textbf{Gráfico:} A corrente ficará travada em $0A$ (ou ruído de fA) por todos os $100\mu s$.
        \item \textbf{Ação:} Tire um print e escreva: "Com condições iniciais desfavoráveis, o circuito possui um ponto de operação estável em corrente zero, provando a necessidade de um circuito de partida."
    \end{itemize}

    \item \textbf{Passo 2: A Solução (Com Circuito de Start-up)}
    \begin{itemize}
        \item Adicione o circuito de start-up no esquemático.
        \item \textbf{Sugestão Simples:} Um divisor resistivo (ou dois transistores diodo invertidos) que injeta uma tensão no gate do NMOS apenas quando a fonte está desligada.
        \item Rode a mesma simulação (mantendo o \texttt{.ic}).
        \item \textbf{Gráfico:} A corrente começa em 0, mas o circuito de start-up "vence" a condição inicial e a corrente salta para $0.7\mu A$.
    \end{itemize}
\end{enumerate}



\textbf{Dica de Ouro:} Se você não quiser desenhar um circuito complexo de start-up na hora, use o "Start-up via Leakage": Coloque um transistor NMOS muito comprido ($W=0.5u, L=50u$) com Gate em VDD, Source em VSS e Dreno no nó do Gate dos NMOS da fonte. Ele age como um resistor gigante injetando uma corrente minúscula que é suficiente para "acordar" a fonte na simulação.

\newpage

\subsubsection[\textcolor{red}{Prova 2022a Q9: Otimização de Resistor (Trimming via Sweep)}]{\textcolor{red}{Prova 2022a Q9: Otimização de Resistor (Trimming via Sweep)}}

\textbf{Objetivo:} Projetar uma fonte de corrente precisa de $0.7\mu A$ ($V_{DD}=3.0V$).

\textbf{Problema:} Os cálculos teóricos (Python) usam modelos simplificados. Na simulação com modelos reais (BSIM3/4), um resistor calculado teoricamente como $80k\Omega$ pode resultar em uma corrente de $0.62\mu A$ ou $0.75\mu A$.

\textbf{Solução:} Em vez de "chutar" valores manualmente, faremos uma varredura (Sweep) no valor do resistor para encontrar o ponto exato.

\textbf{1. Preparação do Netlist para Otimização}

O segredo é definir o valor do resistor como um **parâmetro** (`.PARAM`) e não como um número fixo.

\begin{lstlisting}[language=pspice, caption={Script de Otimização de Resistor (Sweep Paramétrico)}]
* 
* 1. INCLUDES
.include "minha_fonte.pex.netlist"
.include "/local/tools/dkit/ams_3.70_mgc/eldo/c35/modeloMOD"
.defmod pmos4 modp
.defmod nmos4 modn
* Incluir modelo de resistor se necessario
.include "/local/tools/dkit/ams_3.70_mgc/eldo/c35/restm.mod"

* 2. DEFINICAO DE PARAMETRO (VARIAVEL)
* Chute inicial calculado pelo Python (ex: 80k)
.Param R_TRIM = 80k

* 3. COMPONENTE COM VALOR VARIAVEL
* Aqui reside o truque: Se o resistor R1 estiver dentro do netlist PEX,
* voce nao consegue mudar o valor dele facilmente aqui fora.
* SOLUCAO: No esquematico, coloque o resistor com valor '{R_TRIM}' 
* (entre chaves) ou edite o netlist manualmente para:
* XR1 no_a no_b rpolyh R=R_TRIM

* 4. FONTES
Vdd VDD 0 DC 3.0V
Vss VSS 0 0
* Fonte Dummy para medir a corrente de saida
Vout_meas OUT 0 1.5V

* 5. SIMULACAO DE VARREDURA (.DC PARAM)
* Vamos varrer o valor de R_TRIM em vez de varrer uma fonte de tensao.
* Faixa: De 50k a 120k (uma margem segura ao redor do valor teorico)
* Passo: 0.1k (100 ohms de precisao)

.DC param R_TRIM 50k 120k 0.1k

* 6. MEDICAO
.probe DC I(Vout_meas)
\end{lstlisting}

\textbf{2. Procedimento de "Trimming" no EZWave}

\begin{enumerate}
    \item \textbf{Executar:} Rode a simulação.
    \item \textbf{Análise do Gráfico:}
    \begin{itemize}
        \item O eixo X agora é a **Resistência ($k\Omega$)**.
        \item O eixo Y é a **Corrente ($A$)**.
        \item A curva será decrescente (quanto maior o resistor, menor a corrente).
    \end{itemize}
    \item \textbf{Encontrar o Valor Ideal:}
    \begin{itemize}
        \item Use a ferramenta **Cursor** ou **Measurement Tool**.
        \item Mova o cursor até que o valor de Y seja exatamente \texttt{700.0nA} ($0.7\mu A$).
        \item Leia o valor correspondente no eixo X (ex: $84.3 k\Omega$).
    \end{itemize}
    
    

    \item \textbf{Atualização do Projeto (Back-Annotation):}
    \begin{itemize}
        \item Volte ao **Design Architect** (Esquemático).
        \item Selecione o resistor.
        \item Altere a propriedade \texttt{R} para o valor encontrado (ex: $84.3k$).
        \item Gere o netlist novamente e rode uma simulação simples (`.op` ou `.tran`) para confirmar que agora cravou em $0.7\mu A$.
    \end{itemize}
\end{enumerate}

\textbf{Dica de Prova:} Escreva na folha: "Devido às não-linearidades dos modelos BSIM que não são capturadas pela equação quadrática simples, o valor teórico de $R$ resultou em um erro de corrente. Foi realizada uma análise paramétrica (.DC SWEEP) para ajustar o resistor para $84.3k\Omega$, garantindo $I_{out}=0.7\mu A$ com precisão." Isso demonstra domínio da ferramenta.
\newpage

\section{Comandos e Configurações Gerais}

\subsection{Configuração do Ambiente}

\subsubsection{Inicialização dos Ambientes}
\begin{lstlisting}[language=bash, caption={Terminal 1 - Para ICStation (Mentor)}]
cd /local/users/cad/
source .cshrc-mentor
\end{lstlisting}

\begin{lstlisting}[language=bash, caption={Terminal 2 - Para ELDO (Anacad)}]
cd /local/users/cad/
source .cshrc-anacad
\end{lstlisting}

\textbf{Importante}: Executar em shells separados para evitar conflitos.

\subsubsection{Criação e Acesso a Projetos}
\begin{lstlisting}[language=bash]
# Novo projeto
ams_ics -project nome_projeto -t c35b4c3

# Projeto existente (ex: ams_ics -p nome)
ams_ics -p nome_projeto

# Exemplo de prova
ams_ics -p prova24brbtv -t -c35b4c3
\end{lstlisting}

\subsubsection{Diretórios de Trabalho}
\begin{lstlisting}[language=bash, frame=none]
# Diretorio padrao de trabalho
cd /local/users/cad/workavdl

# Estrutura de projetos
nome_projeto.proj/
|-- cell.lib/
|   `-- default.group/
|       |-- logic.views/ (Esquematicos, Simbolos)
|       `-- layout.views/ (Layouts, pasta .cal com PEX)
`-- ...
\end{lstlisting}

\subsection{Comandos de Terminal Essenciais}

\subsubsection{Gerenciamento de Arquivos}
\begin{lstlisting}[language=bash]
# Criar arquivo de simulacao
touch circuito.cir

# Ver conteudo
cat circuito.cir

# Editar
vim circuito.cir
# ou
nano circuito.cir

# Copiar seguranca
cp projeto.proj projeto_backup.proj
\end{lstlisting}

\subsubsection{Execução de Simulações}
\begin{lstlisting}[language=bash]
# Simulacao ELDO
eldo circuito.cir

# Abrir resultados
ezwave circuito.wdb

# Executar em background
eldo circuito.cir > log.txt 2>&1 &
\end{lstlisting}

\subsubsection{Processamento de Imagens (Screenshots)}
\begin{lstlisting}[language=bash]
# Capturar a janela (mouse vira uma mira)
xwd > layout.xwd

# Converter para formato util (inverte e limpa)
convert -white-threshold 1 -negate layout.xwd layout.tif
convert -white-threshold 1 -negate layout.xwd layout.png
\end{lstlisting}

\subsection{Paths e Diretórios Importantes}

\subsubsection{Modelos e Regras}
\begin{lstlisting}[language=bash, frame=none]
# Pasta de Modelos de transistores
/local/tools/dkit/ams_3.70_mgc/eldo/c35/

# Modelo Tipico (exemplo de arquivo)
/local/tools/dkit/ams_3.70_mgc/eldo/c35/cmos53tm.mod

#  Modelo de Resistor
/local/tools/dkit/ams_3.70_mgc/eldo/c35/restm.mod

# Regras Calibre
/local/users/cad/work/rules/cac35b4rules_all.run
/local/users/cad/Calibre_rules/cac35b4rules_all.run

# Manuais
/local/tools/dkit/ams_3.70_mgc/docs/ENG-182_rev2.pdf
/local/tools/dkit/ams_3.70_mgc/docs/ENG-183_rev2.pdf
/local/tools/mentor/shared/pdfdocs/eldo_ur.pdf
\end{lstlisting}

\subsubsection{Arquivos de Projeto}
\begin{lstlisting}[language=bash, frame=none]
# Netlists PEX (localizacao)
/local/users/cad/work/<projeto>.proj/cell.lib/default.group/
layout.views/<celula>/<celula>.cal/

# ViewPoints (localizacao)
/local/users/cad/work/<projeto>.proj/cell.lib/default.group/
logic.views/<celula>/vpt_c35b4_device
\end{lstlisting}

\subsection{Configurações do ICStation}

\subsubsection{Hotkeys e Softkeys}
\begin{itemize}
\item \textbf{Other $\rightarrow$ Hotkeys $\rightarrow$ Enable} 
\item \textbf{Other $\rightarrow$ Hotkeys $\rightarrow$ Load} 
\item \textbf{Other $\rightarrow$ Hotkeys $\rightarrow$ Report} (ver hotkeys ativas) 
\item \textbf{MGC $\rightarrow$ Setup $\rightarrow$ Show Softkeys} 
\end{itemize}

\subsubsection{Configurações de Grid}
\begin{itemize}
\item \textbf{Other $\rightarrow$ Window $\rightarrow$ Set Grid} 
\item \textbf{X = 0.025}, \textbf{Y = 0.025} 
\item \textbf{Minor = 0.1}, \textbf{Major = 1} 
\end{itemize}

\subsubsection{Camadas e Visualização}
\begin{itemize}
\item \textbf{Other $\rightarrow$ Layers $\rightarrow$ Show layer palette $\rightarrow$ Append $\rightarrow$ all} 
\item \textbf{View $\rightarrow$ Visible Layers} (configurar visibilidade) 
\item \textbf{Setup $\rightarrow$ Select Filter $\rightarrow$ Properties} (para selecionar textos) 
\end{itemize}

\subsubsection{Reserva de Células}
\begin{itemize}
\item \textbf{Verificar status}: "Context: nome\_celula(GE-R-0)" = Não reservado 
\item \textbf{Reservar}: \textbf{File $\rightarrow$ Cell $\rightarrow$ Reserve $\rightarrow$ Current Context} 
\item \textbf{Liberar}: \textbf{File $\rightarrow$ Cell $\rightarrow$ Unreserve}
\end{itemize}

\newpage

\subsection{Configurações ELDO/SPICE}

\subsubsection{ Sintaxe de Componentes (ELDO)}
\begin{itemize}
    \item \textbf{Transistor (MOSFET):} 
    \begin{lstlisting}[language=pspice, frame=none]
M<nome> <dreno> <gate> <source> <bulk> <modelo>
+ w=<width> l=<length>
+ Ad=<area_dreno> Pd=<perim_dreno>
+ As=<area_source> Ps=<perim_source>
    \end{lstlisting}
    \item \textbf{Resistor:} \texttt{R<nome> <no1> <no2> <valor>} 
    \item \textbf{Capacitor:} \texttt{C<nome> <no1> <no2> <valor>} 
    \item \textbf{Subcircuito:} \texttt{x<nome> <nos...> <nome\_modelo>} 
    \item \textbf{Fonte de Tensão (V):} 
    \begin{lstlisting}[language=pspice, frame=none]
V<nome> <no+> <no-> <valor_dc>
V<nome> <no+> <no-> PULSE(<v_min> <v_max> <delay> <t_rise> <t_fall> <t_pulse> <periodo>)
V<nome> <no+> <no-> SIN(<offset> <amplitude> <freq> <delay> <amortec> <fase>)
    \end{lstlisting}
\end{itemize}

\subsubsection{Parâmetros Comuns}
\begin{lstlisting}[language=pspice]
* Tensao de alimentacao
.param VDD_value='3.3V'
VDD VDD 0 DC VDD_value

* Parametros de tempo e frequencia
.param freq='10MEG'
.param T='1/freq'

* Capacitancia de carga
.param Cload='50f'
CL OUT 0 Cload
\end{lstlisting}

\subsubsection{Includes de Modelos}
\begin{lstlisting}[language=pspice]
* Modelos para diferentes condicoes
.include "/local/tools/dkit/ams_3.70_mgc/eldo/c35/modeloWP" * Worst Power
.include "/local/tools/dkit/ams_3.70_mgc/eldo/c35/modeloWS" * Worst Speed
.include "/local/tools/dkit/ams_3.70_mgc/eldo/c35/modeloMOD" * Typical

* Modelos de resistor
.include "/local/tools/dkit/ams_3.70_mgc/eldo/c35/restm.mod"
\end{lstlisting}

\subsubsection{Comandos de Análise e Medição}
\begin{lstlisting}[language=pspice]
* Ponto de Operacao DC
.op

* Analise DC (varredura)
.DC <fonte> <inicio> <fim> <passo>

* Analise transiente
.tran <passo_print> <tempo_final> <tempo_inicial> <passo_max>

* Analise com varredura (ex: varrer T)
.tran 0 80n 0 0.01n sweep T 0.4n 0.6n 0.01n

* Condicoes iniciais
.ic V(no)=valor

* Salvar formas de onda
.probe DC V(no1) V(no2)
.probe tran V(no1) I(Vfonte)

* Medir tempos de propagacao (tphl/tplh)
.meas tran <nome> trig V(<in>) val='VDD/2' fall=1 
+                  targ V(<out>) val='VDD/2' rise=1

* Medir potencia media
.meas tran <current> AVG I(Vdd) FROM=<inicio> TO=<fim>
.meas tran <pot> param='<current> * VDD_value'
\end{lstlisting}

\subsubsection{ Dicas Críticas de Simulação (ELDO)}
\begin{itemize}
    \item \textbf{Primeira Linha:} Comente a primeira linha do arquivo \texttt{.cir} (título), pois ela pode não ser lida corretamente.
    \item \textbf{Nomes de Modelo:} Use \texttt{MODN} e \texttt{MODP} (definidos por \texttt{.defmod}) ao invés de \texttt{NMOS4} e \texttt{PMOS4} no netlist, a menos que esteja incluindo os arquivos de modelo completos (como \texttt{cmos53tm.mod}).
    \item \textbf{Netlist PEX:} Lembre-se de editar o netlist PEX para corrigir a linha \texttt{.subckt} e os nomes dos modelos.
\end{itemize}

\subsection{Scripts e Automação}

\subsubsection{Script de Inicialização Rápida}
\begin{lstlisting}[language=bash]
#!/bin/csh
# init_project.csh
cd /local/users/cad/
source .cshrc-mentor
cd /local/users/cad/workavdl
ams_ics -p $1
\end{lstlisting}

\subsubsection{Script de Simulação Automática}
\begin{lstlisting}[language=bash]
#!/bin/csh
# run_simulation.csh
cd /local/users/cad/
source .cshrc-anacad
cd /local/users/cad/workavdl/$1
eldo circuito.cir
ezwave circuito.wdb
\end{lstlisting}

\subsubsection{Comandos TCL para EZWave}
\begin{lstlisting}[language=bash]
# config_ezwave.tcl
wave add V(IN) V(OUT)
xaxis limit 0 100n
yaxis limit -0.5 3.5
cursor add
cursor add
\end{lstlisting}

\subsection{Constantes e Parâmetros Técnicos}

\subsubsection{Mobilidades e Constantes}
\begin{itemize}
\item $\mu_n = 370\ cm^2/V\cdot s$ 
\item $\mu_p = 126\ cm^2/V\cdot s$ 
\item $r = \mu_n/\mu_p \approx 2.94$
\item $C_{ox} \approx 4.6\ fF/\mu m^2$ 
\item $\epsilon_{ox} = 3.5 \times 10^{-13}\ F/cm$ 
\item $t_{ox} = 7.6\ nm$ 
\end{itemize}

\subsubsection{Modelos de Transistores (Valores U0)}
\begin{itemize}
\item \textbf{Típico (tm)}:
 \begin{itemize}
 \item NMOS: U0 = 4.758e+02 
 \item PMOS: U0 = 1.482e+02 
 \item $r \approx 3.21$
 \end{itemize}
\item \textbf{Worst Case Power (wp)}:
 \begin{itemize}
 \item NMOS: U0 = 5.002e+02 
 \item PMOS: U0 = 1.581e+02 
 \item $r \approx 3.16$
 \end{itemize}
\item \textbf{Worst Case Speed (ws)}:
 \begin{itemize}
 \item NMOS: U0 = 4.671e+02 
 \item PMOS: U0 = 1.314e+02 
 \item $r \approx 3.55$
 \end{itemize}
\end{itemize}

\subsubsection{Distâncias Críticas e Cálculos}
\begin{itemize}
\item \textbf{POLY-POLY}: 0.45 $\mu m$ 
\item \textbf{RES-POLY}: 0.35 $\mu m$ 
\item \textbf{DIFF-NTUB}: 1.2 $\mu m$ 
\item \textbf{NTUB enclosure}: 1.2 $\mu m$ 
\item \textbf{MET1 largura VDD/VSS}: 1.8 $\mu m$ (recomendado) 
\item \textbf{ Área/Perímetro (AD/PD):} 
    \begin{itemize}
        \item $AD = 0.85 \times W$
        \item $PD = W + (2 \times 0.85)$ (ou $PD = W + 1.7$)
    \end{itemize}
\end{itemize}

\subsection{Solução de Problemas Comuns}

\subsubsection{Problemas de ICStation}
\begin{itemize}
\item \textbf{Mouse comporta-se como Ctrl pressionado}:
 \begin{itemize}
 \item Solução: Pressione \textbf{Ctrl+Shift} e use as setas do teclado; isso deve normalizar .
 \end{itemize}
\item \textbf{ICStation Travado}:
 \begin{itemize}
 \item Causa: Clicar repetidamente no scroll do mouse.
 \item Solução: Fechar e reabrir.
 \end{itemize}
\item \textbf{Células desaparecem da library}:
 \begin{itemize}
 \item Solução: Crie um novo projeto com o nome antigo e copie a pasta \texttt{default.group} do projeto antigo/backup para a nova pasta do projeto .
 \end{itemize}
\item \textbf{Não consegue colocar ports}:
 \begin{itemize}
 \item Solução: Tente \textbf{DLA Layout $\rightarrow$ Open}.
 \end{itemize}
\end{itemize}

\subsubsection{Problemas de ELDO}
\begin{itemize}
\item \textbf{Erro de modelo não encontrado}:
 \begin{itemize}
 \item Verificar includes dos modelos.
 \item Substituir \texttt{NMOS4/PMOS4} por \texttt{MODN/MODP} ou vice-versa.
 \end{itemize}
\item \textbf{Netlist PEX com erros}:
 \begin{itemize}
 \item Verificar conexões: \texttt{.connect VSS N\_VSS\_...} .
 \item Corrigir resistores: \texttt{rR0} $\rightarrow$ \texttt{XR0}.
 \item Remover parâmetro \texttt{AREA} de bipolares.
 \end{itemize}
\item \textbf{Simulação não converge}:
 \begin{itemize}
 \item Aumentar HMAX.
 \item Adicionar \texttt{.ic} com condições iniciais.
 \item Verificar fontes de alimentação.
 \end{itemize}
\end{itemize}

\subsubsection{Problemas de Calibre}
\begin{itemize}
\item \textbf{Rules não carregam}:
 \begin{itemize}
 \item Verificar path completo do arquivo \texttt{.run}.
 \item \textbf{Load} obrigatório após selecionar.
 \end{itemize}
\item \textbf{LVS não encontra schematic}:
 \begin{itemize}
 \item Verificar se ViewPoint foi criado .
 \item Verificar path do \texttt{vpt\_c35b4\_device} .
 \end{itemize}
\item \textbf{Erros de port}:
 \begin{itemize}
 \item Executar \textbf{Add Text on Ports} novamente .
 \item Verificar camada \texttt{M1NET} .
 \end{itemize}
\end{itemize}

\subsection{Comandos de Debug e Verificação}

\subsubsection{Verificação de Layout}
\begin{lstlisting}[language=bash, frame=none]
# Relatorio de camadas
Report -> Layer -> Design Layers

# Relatorio de selecao
Report -> Selected

# Tamanho da celula
Report -> Windows
\end{lstlisting}

\subsubsection{Verificação de Hierarquia}
\begin{itemize}
\item \textbf{Context $\rightarrow$ Hierarchy $\rightarrow$ Peek $\rightarrow$ 2-4 levels} 
\item \textbf{Edit $\rightarrow$ Flatten} (para desmontar célula) 
\item \textbf{Objects $\rightarrow$ Make $\rightarrow$ Cell} (para definir célula) 
\end{itemize}

\subsubsection{Comandos de History}
\begin{lstlisting}[language=bash]
# Ver comandos recentes
history

# Reexecutar comando
!numero

# Encontrar arquivos
find -name "*.cir"
\end{lstlisting}

\subsection{Fluxos de Trabalho Otimizados}

\subsubsection{Fluxo Completo do Projeto}
\begin{enumerate}
\item \textbf{Terminal}: \texttt{source .cshrc-mentor} + \texttt{ams\_ics}
\item \textbf{Schematic}: Criar e verificar esquemático
\item \textbf{Symbol}: Gerar e configurar símbolo
\item \textbf{ViewPoint}: Criar para netlist
\item \textbf{Layout}: AutoInst + otimizacao + routing
\item \textbf{DRC}: Verificar e corrigir erros
\item \textbf{LVS}: Comparar com schematic
\item \textbf{PEX}: Extrair parasitas
\item \textbf{ELDO}: Preparar e executar simulacao
\item \textbf{EZWave}: Analisar resultados
\end{enumerate}

\subsubsection{Fluxo Rapido para Modificacoes}
\begin{enumerate}
\item Modificar schematic
\item Check schematic
\item Atualizar layout (se necessario)
\item Run DRC + LVS
\item Run PEX
\item Simular e analisar
\end{enumerate}

\subsection{Backup e Versionamento}

\subsubsection{Estrategias de Backup}
\begin{lstlisting}[language=bash]
# Backup completo do projeto
cp -r projeto.proj projeto_backup_date.proj

# Backup incremental
tar -czf projeto_$(date +%Y%m%d).tar.gz projeto.proj

# Backup seletivo (apenas celulas importantes)
cp -r projeto.proj/cell.lib/default.group/logic.views/celula_chave
\end{lstlisting}

\subsubsection{Controle de Versões Manual}
\begin{itemize}
\item \textbf{Nomear versões}: projeto\_v1, projeto\_v2, etc.
\item \textbf{Documentar mudanças}: Arquivo CHANGES.txt
\item \textbf{Backup antes de modificações grandes}
\item \textbf{Testar cada versão} antes de prosseguir
\end{itemize}

\subsection{Dicas de Produtividade}

\subsubsection{Atalhos e Boas Práticas}
\begin{itemize}
\item \textbf{Salvar frequentemente} em todas as ferramentas
\item \textbf{Usar nomes descritivos} para células e sinais
\item \textbf{Documentar} configurações importantes
\item \textbf{Testar incrementalmente} cada parte do circuito
\item \textbf{Manter organização} nos diretórios e arquivos
\end{itemize}

\subsubsection{Otimização de Performance}
\begin{itemize}
\item \textbf{Limitar complexidade} do layout quando possível
\item \textbf{Usar extrações mais simples} (C+CC ao invés de R+C+CC) para debug
\item \textbf{Fechar ferramentas} não utilizadas para liberar memória
\item \textbf{Limitar número de sinais} no EZWave para melhor performance
\end{itemize}

\subsubsection{Workflow em Equipe}
\begin{itemize}
\item \textbf{Comunicar mudanças} em células compartilhadas
\item \textbf{Reservar células} quando estiver trabalhando
\item \textbf{Documentar interfaces} entre diferentes blocos
\item \textbf{Manter padrões consistentes} de nomenclatura
\end{itemize}

\subsection{Comandos de Emergência}

\subsubsection{Recuperação de Projeto}
\begin{lstlisting}[language=bash]
# Se cElulas desaparecerem 
ams_ics -p projeto_copy -t c35b4c3
# Copiar default.group do projeto antigo para o novo

# Se arquivo corrompido
cp backup/projeto_backup.proj ./
\end{lstlisting}

\subsubsection{Limpeza de Arquivos Temporários}
\begin{lstlisting}[language=bash]
# Limpar arquivos de simulaCAOo temporArios
rm -f *.wdb *.log *.out *.err

# Limpar netlists antigos
rm -rf *.cal/
\end{lstlisting}

\subsubsection{Reinicialização de Ambiente}
\begin{lstlisting}[language=bash]
# Se problemas com variaveis de ambiente
exit
# Re-login e re-executar source commands
\end{lstlisting}
\newpage
\newpage
\section{Tópicos Específicos}

\subsection{Células da CORELIB}

\subsubsection{Células Disponíveis (Exemplos)}
\begin{itemize}
\item \textbf{inv0, inv1}: Inversores básicos.
\item \textbf{df1, df3}: Flip-flops D.
\item \textbf{dl1}: Latch D.
\item \textbf{nand2, nand21, nand22, nand23}: Portas NAND .
\item \textbf{nor2, nor21, nor23}: Portas NOR .
\item \textbf{xor2}: Porta XOR.
\item \textbf{and2}, \textbf{or2}.
\end{itemize}

\subsubsection{Uso no Schematic}
\begin{enumerate}
\item \textbf{Add $\rightarrow$ Instance $\rightarrow$ Choose Symbol} .
\item \textbf{Biblioteca}: CORELIB.
\item Selecionar célula desejada (ex: \texttt{df1}).
\item Conectar conforme necessário.
\item \textbf{Verificar}: Check Schematic (0 errors, 0 warnings).
\end{enumerate}

\subsubsection{Uso no Layout}
\begin{itemize}
\item \textbf{ICStudio $\rightarrow$ Library $\rightarrow$ CORELIB $\rightarrow$ célula}.
\item Já inclui layout pré-validado (célula-padrão).
\item \textbf{DRC/LVS}: Geralmente já verificado.
\end{itemize}

\subsection{Adição de PADs}

\subsubsection{Procedimento Completo}
\begin{enumerate}
\item \textbf{Objects $\rightarrow$ Add $\rightarrow$ Cell}.
\item \textbf{Biblioteca}: \textbf{IOLIB\_4M} .
\item \textbf{Célula}: \textbf{g\_padonly} (último item).
\item Colocar um PAD para VDD e um para VSS.
\item \textbf{Conectar com shapes}: 
 \begin{itemize}
 \item Puxar shapes de metal dos PADs até os pontos de conexão.
 \item \textbf{Grossura}: O mais grosso possível (reduzir se der erro DRC).
 \item \textbf{Materiais}: Usar MET1 ou MET2.
 \end{itemize}
\item \textbf{Verificar DRC}: Corrigir erros de espaçamento.
\end{enumerate}

\subsubsection{Configuração de PADs}
\begin{itemize}
\item \textbf{Orientação}: Posicionar nas bordas do chip.
\item \textbf{Alimentação}: PAD VDD e VSS devem ser robustos.
\item \textbf{Sinais}: PADs para entradas/saídas importantes.
\item \textbf{ESD}: Proteção geralmente incluída nos PADs.
\end{itemize}

\subsection{Criação de Hierarquia}

\subsubsection{Hierarquia no Schematic}
\begin{enumerate}
\item Criar schematic de bloco inferior (ex: \texttt{inversor}).
\item Gerar símbolo e configurar a propriedade \textbf{phy\_comp} (ver abaixo).
\item No schematic superior: \textbf{Add $\rightarrow$ Instance $\rightarrow$ Choose Symbol}.
\item Selecionar símbolo do bloco inferior (ex: \texttt{inversor}).
\item Conectar ports e alimentação.
\item \textbf{Check Schematic} (0 errors, 0 warnings).
\end{enumerate}

\subsubsection{ Ligação Símbolo-Layout (phy\_comp)}
Esta é a etapa crucial que conecta o símbolo do esquemático ao seu layout físico.
\begin{enumerate}
    \item \textbf{Copiar Localização do Layout:}
    \begin{itemize}
        \item No \textbf{ICStation}, clique com o \textbf{botão direito} na \textit{view} do seu layout.
        \item \textbf{Properties $\rightarrow$ Location}.
        \item Copie o caminho exibido (ex: \texttt{\$minha\_lib/default.group/layout.views/inv}) .
    \end{itemize}
    \item \textbf{Adicionar Propriedade ao Símbolo:}
    \begin{itemize}
        \item Volte ao \textbf{Design Architect} e abra o \textbf{SÍMBOLO} (\texttt{File $\rightarrow$ Open $\rightarrow$ Symbol}).
        \item \textbf{Comando:} \textbf{Add $\rightarrow$ Properties}.
        \item \textbf{Property Name}: \texttt{phy\_comp}.
        \item \textbf{Property Value}: Cole a \textbf{Location} copiada .
    \end{itemize}
    \item \textbf{Ajustar Texto e Verificar (A "Parte Mágica"):}
    \begin{itemize}
        \item \textbf{Setup $\rightarrow$ Select Filter $\rightarrow$ Properties} $\rightarrow$ OK .
        \item Mude a altura do texto (ex: \textbf{Change Height $\rightarrow$ Specified $\rightarrow$ 0.2}) .
        \item \textbf{File $\rightarrow$ Check Symbol} (1 warning é esperado) .
        \item \textbf{File $\rightarrow$ Save Symbol}.
        \item \textbf{File $\rightarrow$ Check Symbol} (Agora 0 warnings são esperados) .
    \end{itemize}
\end{enumerate}

\subsubsection{Hierarquia no Layout}
\begin{enumerate}
\item Criar layout do bloco inferior (com DRC/LVS limpos).
\item No layout superior: \textbf{Objects $\rightarrow$ Add $\rightarrow$ Cell}.
\item Selecionar célula do bloco inferior (que agora está ligada ao símbolo).
\item Posicionar e conectar (usando \textbf{Place \& Route} para Standard Cells ou manualmente).
\item \textbf{Verificar}: DRC e LVS hierárquicos.
\end{enumerate}

\subsubsection{Contexto Hierárquico}
\begin{itemize}
\item \textbf{Context $\rightarrow$ Hierarchy $\rightarrow$ Peek $\rightarrow$ N levels}: Ver "dentro" do layout do bloco.
\item \textbf{Edit $\rightarrow$ Flatten}: Desmontar hierarquia (cuidado).
\item \textbf{Objects $\rightarrow$ Make $\rightarrow$ Cell}: Criar célula a partir de shapes.
\end{itemize}

\subsection{Resistores em Poly}

\subsubsection{Adição no Schematic}
\begin{enumerate}
\item \textbf{HIT-KIT Utilities $\rightarrow$ Devices $\rightarrow$ Resistors}.
\item \textbf{rpolyh}: Resistor poly de alta resistividade.
\item \textbf{Parâmetros}:
 \begin{itemize}
 \item \textbf{R}: Valor da resistência.
 \item \textbf{L}: Comprimento (calculado automaticamente).
 \item \textbf{W}: Largura.
 \item \textbf{Bends}: Número de dobras (para layout compacto).
 \end{itemize}
\end{enumerate}

\subsubsection{Layout de Resistores}
\begin{itemize}
\item \textbf{AutoInst}: Gera layout automaticamente.
\item \textbf{Dobramento}: \textbf{Object $\rightarrow$ Change $\rightarrow$ Device $\rightarrow$ Bend}.
\item \textbf{Distâncias}: Respeitar espaçamento POLY-RES (0.35$\mu$m).
\item \textbf{Contatos}: Usar MET1 para conexões.
\end{itemize}

\subsubsection{Modelo ELDO}
\begin{lstlisting}[language=pspice]
* Incluir modelo de resistor
.include "/local/tools/dkit/ams_3.70_mgc/eldo/c35/restm.mod"

* Instancia no PEX (vem como 'rR0', trocar para 'XR0')
XR0 n1 n2 rpolyh r=1k
\end{lstlisting}

\subsection{Fontes de Corrente e Espelhos}

\subsubsection{Implementação Básica}
\begin{itemize}
\item \textbf{Transistores em diodo}: Para referência.
\item \textbf{Espelho simples}: Copiar corrente.
\item \textbf{Espelho cascode} ou \textbf{Wilson}: Maior impedância de saída, melhor topologia para fontes de corrente (conforme Prova 2016b, Q9) .
\item \textbf{Polarização}: Usar resistores ou divisores.
\end{itemize}

\subsubsection{Dimensionamento}
\begin{itemize}
\item \textbf{Corrente de referência}: Definida por resistor ou fonte.
\item \textbf{Relação W/L}: Define razão de espelhamento.
\item \textbf{Tensão Early}: Considerar para precisão.
\item \textbf{Matching}: Transistores próximos no layout.
\end{itemize}

\subsubsection{Exemplo ELDO}
\begin{lstlisting}[language=pspice]
* Espelho de corrente simples 1:2
Mref ref ref VSS VSS MODN w=10u l=1u
Mout out ref VSS VSS MODN w=20u l=1u
Iref VDD ref DC 100uA
\end{lstlisting}

\subsection{Circuitos C2MOS}

\subsubsection{Características}
\begin{itemize}
\item \textbf{Latch dinâmico}: Mantém estado quando clock desabilitado.
\item \textbf{Clock = Alto (3V)}: $M_{P1}$ e $M_{N1}$ estão conduzindo e o circuito se comporta como um inversor.
\item \textbf{Clock = Baixo (0V)}: $M_{P1}$ e $M_{N1}$ estão cortados e a carga é mantida no capacitor de saída (ex: gate de outra porta).
\item \textbf{Aplicações}: Registradores, pipelines.
\end{itemize}

\subsubsection{Implementação}
\begin{itemize}
\item \textbf{Transistores de passagem} (em série): $M_{P1}$ e $M_{N1}$ controlados pelo clock.
\item \textbf{Inversor principal}: $M_{P2}$ e $M_{N2}$ controlados pela entrada \texttt{In}.
\item \textbf{Bulk}: Conectar corretamente (PMOS em VDD, NMOS em VSS).
\end{itemize}

\subsubsection{Dimensionamento}
\begin{itemize}
\item \textbf{Transistores de passagem}: L mínimo para reduzir carga.
\item \textbf{Inversor}: Dimensionar para tempos equivalentes (subida/descida).
\item \textbf{Clock}: Considerar sobreposição (non-overlap).
\end{itemize}

\subsection{Osciladores em Anel}

\subsubsection{Configuração Básica}
\begin{itemize}
\item \textbf{N estágios}: Número \textbf{ímpar} de inversores.
\item \textbf{Frequência}: $f = \frac{1}{2 \cdot N \cdot t_{pd}}$ (onde $t_{pd}$ é o atraso médio do inversor).
\item \textbf{Condição de oscilação}: Ganho > 1 por estágio.
\item \textbf{Alimentação}: Estável para frequência constante.
\end{itemize}

\subsubsection{Implementação}
\begin{itemize}
\item \textbf{Célula básica}: Inversor ou porta CMOS (ex: NAND com uma entrada em VDD).
\item \textbf{Controle de frequência}: Tensão de alimentação ou capacidades de carga.
\item \textbf{Saída}: Usar um buffer para evitar que a carga de medição afete a oscilação.
\end{itemize}

\subsubsection{Simulação}
\begin{itemize}
\item \textbf{Condições iniciais}: \texttt{.ic} pode ser necessário para "chutar" a oscilação.
\item \textbf{Tempo de simulação}: Múltiplos períodos.
\item \textbf{Medição}: Período e frequência.
\end{itemize}

\subsection{Camadas Especiais}

\subsubsection{HRES (High Resistance)}
\begin{itemize}
\item \textbf{Função}: Usada para construir resistores de polisilício.
\item \textbf{Processo}: Define uma área que isola o \texttt{POLY1} e impede que ele seja dopado (aumentando sua resistividade) .
\item \textbf{Uso}: Definir o corpo do resistor \texttt{rpolyh}.
\item \textbf{DRC}: Verificar espaçamentos (ex: \texttt{RES-POLY}).
\end{itemize}

\subsubsection{NLDD e FIMP}
\begin{itemize}
\item \textbf{NLDD (N-Lightly Doped Drain)}: Reduz campos elétricos na junção dreno-bulk.
\item \textbf{FIMP}: Implantação para ajuste fino de $V_{th}$ (Threshold Voltage).
\item \textbf{Geração}: \textbf{HIT-Kit Utilities $\rightarrow$ Generated Layers} .
\item \textbf{Verificação}: Rodar DRC novamente após a geração .
\end{itemize}

\subsubsection{ Camadas Físicas (Exemplos)}
Os nomes das camadas no processo :
\begin{itemize}
    \item \textbf{ntub}: Poço N (Well) para PMOS.
    \item \textbf{diff}: Região ativa (onde dreno/source são formados).
    \item \textbf{nplus / pplus}: Implantações N+ e P+ para Dreno/Source.
    \item \textbf{poly1}: Primeira camada de Polisilício (Gates).
    \item \textbf{cont}: Contato entre \texttt{diff} ou \texttt{poly1} e \texttt{met1}.
    \item \textbf{met1}: Primeira camada de Metal.
    \item \textbf{via1}: Contato entre \texttt{met1} e \texttt{met2}.
    \item \textbf{pad}: Abertura na passivação para contato externo (Bonding pad).
\end{itemize}

\subsection{Técnicas de Layout Avançadas}

\subsubsection{Match de Transistores (Casamento)}
Essencial para circuitos analógicos (espelhos de corrente, pares diferenciais).
\begin{itemize}
\item \textbf{Orientação comum}: Todos os transistores na mesma direção.
\item \textbf{Proximidade}: Transistores pareados devem ser colocados próximos.
\item \textbf{Interdigitação}: Alternar "dedos" de transistores (ex: A-B-A-B).
\item \textbf{Comum centroid (Centro Comum)}: Layout simétrico ao redor de um ponto central.
\end{itemize}

\subsubsection{Redução de Parasitas}
\begin{itemize}
\item \textbf{Metais curtos}: Minimizar comprimento de interconexões.
\item \textbf{Shielding}: Blindar sinais sensíveis (ex: analógicos) com linhas de VDD/VSS.
\item \textbf{Vias múltiplas}: Usar várias vias para conexões de alta corrente (reduz R e melhora confiabilidade).
\item \textbf{Orientação}: Evitar cruzamentos desnecessários; se inevitável, cruzar em camadas diferentes (ex: MET1 e MET2).
\end{itemize}

\subsubsection{Otimização de Área}
\begin{itemize}
\item \textbf{Merge agressivo}: Juntar drenos/sources sempre que possível .
\item \textbf{Dobramento (Fold)}: Usar \texttt{fold} para transistores largos .
\item \textbf{Compactação}: Mover blocos para o mais próximo possível (respeitando o DRC).
\item \textbf{Roteamento inteligente}: Planejar o roteamento para minimizar a área de "fios".
\end{itemize}

\subsection{Fluxos Especiais}

\subsubsection{Fluxo Standard Cells}
\begin{enumerate}
\item \textbf{Place \& Route $\rightarrow$ Autofp}.
\item \textbf{Aspect Ratio}: 2 em Upper.
\item \textbf{StdCell}: Selecionar área.
\item \textbf{Route}: Configurar net classes para VDD/VSS (ex: 1.8$\mu$m) .
\item \textbf{Verificação}: DRC e LVS hierárquico.
\end{enumerate}

\subsubsection{Fluxo com Células Personalizadas}
\begin{enumerate}
\item Criar células personalizadas (ex: porta lógica) com DRC/LVS limpos.
\item Adicionar ao schematic superior como instâncias.
\item Gerar layout hierárquico (os blocos aparecerão como caixas).
\item Conectar as células com roteamento (manual ou automático).
\item Verificação completa (DRC, LVS).
\end{enumerate}

\subsubsection{Fluxo Analógico-Digital}
\begin{enumerate}
\item \textbf{Blocos analógicos}: Layout manual com \textit{matching} (interdigitação, etc.).
\item \textbf{Blocos digitais}: Standard cells ou auto place \& route.
\item \textbf{Interface}: Cuidado com acoplamento de ruído digital para o analógico.
\item \textbf{Alimentação}: Idealmente, separar VDD/VSS analógico e digital.
\item \textbf{Verificação}: DRC, LVS e extração PEX completa.
\end{enumerate}

\subsection{Problemas Específicos e Soluções}

\subsubsection{Latch-up}
\begin{itemize}
\item \textbf{Causa}: Acionamento de um SCR (tiristor) parasitário (PNPN) entre VDD e VSS, causando um curto-circuito.
\item \textbf{Prevenção}:
 \begin{itemize}
 \item \textbf{Guard rings}: Anéis de proteção (N+ em poço N, P+ em substrato P) ao redor de blocos.
 \item \textbf{Espaçamento}: Distância adequada entre NMOS (PTUB) e PMOS (NTUB).
 \item \textbf{Substrate contacts}: Usar muitos contatos de poço (bulks) para "amarrar" o potencial do substrato ao VDD/VSS.
 \end{itemize}
\item \textbf{Verificação}: DRC para espaçamentos críticos (regra NTUB/PTUB).
\end{itemize}

\subsubsection{Electromigration}
\begin{itemize}
\item \textbf{Causa}: Corrente excessiva "empurra" os átomos de metal em trilhas finas, criando aberturas (falhas).
\item \textbf{Prevenção}:
 \begin{itemize}
 \item \textbf{Largura de metal}: Usar trilhas largas para VDD/VSS (ex: 1.8$\mu$m) .
 \item \textbf{Vias múltiplas}: Usar várias vias para conexões de alta corrente.
 \end{itemize}
\item \textbf{Verificação}: DRC verifica regras de largura mínima de metal (width rules).
\end{itemize}

\subsubsection{Cross-talk (Acoplamento)}
\begin{itemize}
\item \textbf{Causa}: Acoplamento capacitivo entre sinais (trilhas) adjacentes.
\item \textbf{Prevenção}:
 \begin{itemize}
 \item \textbf{Shielding}: Blindar sinais sensíveis (ex: analógicos, clock) com linhas de VDD/VSS ao lado.
 \item \textbf{Espaçamento}: Aumentar distância entre sinais críticos.
 \item \textbf{Orientação}: Cruzar sinais em ângulos retos (usando camadas diferentes, ex: MET1 vs MET2).
 \end{itemize}
\item \textbf{Verificação}: PEX com extração \textbf{C+CC} (Capacitância de Acoplamento) .
\end{itemize}

\subsection{Exemplos de Células Complexas}

\subsubsection{Flip-flop D Master-Slave}
\begin{itemize}
\item \textbf{Composição}: 2 latches (ex: C2MOS ou com realimentação) em série.
\item \textbf{Clock}: Clocks $\phi$ e $\bar{\phi}$ (não sobrepostos) para evitar \textit{race conditions}.
\item \textbf{Reset}: Pode ser assíncrono ou síncrono.
\item \textbf{Layout}: Cuidado com o roteamento e distribuição do(s) sinal(is) de clock.
\end{itemize}

\subsubsection{Buffer de Potência (Driver)}
\begin{itemize}
\item \textbf{Função}: Dirigir cargas capacitivas grandes (ex: pinos de I/O, linhas de clock longas).
\item \textbf{Estágios múltiplos}: Uma cadeia de inversores com W/L progressivamente maiores.
\item \textbf{Dimensionamento}: Razão de "tapering" constante (fator 'f') entre os estágios.
\item \textbf{Layout}: Transistores de saída muito largos (usando \textit{fold}) e múltiplas vias para VDD/VSS.
\end{itemize}

\subsubsection{Comparador Analógico}
\begin{itemize}
\item \textbf{Amplificador diferencial}: Estágio de entrada (alta impedância, rejeição de modo comum).
\item \textbf{Estágio de ganho}: Amplificação do pequeno sinal de diferença.
\item \textbf{Latch de saída}: Converte o sinal analógico amplificado em um sinal digital (0 ou 1).
\item \textbf{Layout}: \textit{Matching} (casamento) é crítico no par diferencial de entrada.
\end{itemize}

\subsection{Técnicas de Verificação Avançada}

\subsubsection{LVS com Células Personalizadas (Hierarquia)}
\begin{itemize}
\item \textbf{phy\_comp}: A propriedade é essencial. O LVS usará o layout apontado por \texttt{phy\_comp} em vez de procurar transistores dentro do símbolo .
\item \textbf{Hierarquia}: Verificar todos os níveis (LVS hierárquico).
\item \textbf{Interfaces}: Conferir se os ports (VDD, VSS, A, B) do bloco correspondem ao esquemático.
\item \textbf{Problemas comuns}: Bulk não conectado dentro do bloco, ports trocados.
\end{itemize}

\subsubsection{ERC (Electrical Rule Checking)}
O DRC moderno (Calibre) já inclui muitas verificações de ERC:
\begin{itemize}
\item \textbf{Gates flutuantes} (ex: erro \texttt{ILL\_FLOATING\_GATE\_ERC}).
\item \textbf{Bulk connections} (ex: erro \texttt{NWELL TOO HOT}).
\item \textbf{Short circuits} (curtos entre VDD e VSS).
\item \textbf{Open circuits} (nets não conectados).
\end{itemize}

\subsubsection{Análise de Performance}
\begin{itemize}
\item \textbf{Atraso}: Extrair $t_{pd}$ (média de $t_{phl}$ e $t_{plh}$).
\item \textbf{Potência}: Consumo estático (vazamento) e dinâmico ($C \cdot V^2 \cdot f$).
\item \textbf{Margem de ruído}: Imunidade a interferências.
\item \textbf{Temperatura}: Efeito na performance (simulação \texttt{.DC TEMP}).
\end{itemize}

\subsection{Scripts e Automação Avançada}

\subsubsection{Batch DRC/LVS}
\begin{lstlisting}[language=bash]
# Script para verificacao em lote
foreach cell (cell1 cell2 cell3)
  calibre -drc $cell
  calibre -lvs $cell
end
\end{lstlisting}

\subsubsection{Extração Automática de Parâmetros}
\begin{lstlisting}[language=bash]
# Extrair W/L de todos os transistores
grep "M.*MOD" schematic.netlist | awk '{print $8, $9}'
\end{lstlisting}

\subsubsection{Geração de Relatórios}
\begin{lstlisting}[language=bash, frame=none]
# (Exemplo de scripts hipoteticos)
report_area.tcl
report_performance.tcl
\end{lstlisting}

\subsection{Migração de Tecnologia}

\subsubsection{Ajustes Necessários}
\begin{itemize}
\item \textbf{Regras de design}: Novas distâncias mínimas.
\item \textbf{Modelos de transistor}: Novos parâmetros (ex: $V_{th}$, $C_{ox}$).
\item \textbf{Capacitâncias}: Valores diferentes.
\item \textbf{Resistências}: Novos valores de \textit{sheet resistance}.
\end{itemize}

\subsubsection{Procedimento}
\begin{enumerate}
\item \textbf{Revisar regras}: Novas regras DRC.
\item \textbf{Ajustar dimensões}: W/L se necessário.
\item \textbf{Re-rotear}: Se necessário para novas regras.
\item \textbf{Re-verificar}: DRC, LVS, simulação.
\end{enumerate}

\subsection{Documentação e Manutenção}

\subsubsection{Documentação de Células}
\begin{itemize}
\item \textbf{Esquemático}: Diagrama limpo e legível.
\item \textbf{Símbolo}: Pins organizados logicamente.
\item \textbf{Layout}: Com medidas críticas anotadas.
\item \textbf{Performance}: Parâmetros extraídos (atrasos, potência).
\end{itemize}

\subsubsection{Versionamento}
\begin{itemize}
\item \textbf{Nomenclatura}: celula\_v1, celula\_v2.
\item \textbf{Log de mudanças}: O que foi modificado e porquê.
\item \textbf{Backup}: Versões estáveis arquivadas.
\item \textbf{Testes}: Resultados de simulação documentados.
\end{itemize}

\subsubsection{Manutenção}
\begin{itemize}
\item \textbf{Atualização}: Ajustes para novas versões das ferramentas.
\item \textbf{Otimização}: Melhorias contínuas de performance.
\item \textbf{Bug fixes}: Correção de problemas identificados.
\item \textbf{Documentação}: Manter documentação atualizada.
\end{itemize}

\end{document}
